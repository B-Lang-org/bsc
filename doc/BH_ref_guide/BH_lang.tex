% $Id$
%
% This document was forked from the internal Bluespec repository
% bsc-doc.git/doc/language-manual_old/lang.tex
% at commit 5a70486924b35cbd3344750debd4f4828e8e8e54.

\documentclass[twoside,letterpaper]{article}
% \usepackage{a4}
\usepackage[latin1]{inputenc}
\usepackage[T1]{fontenc}
\usepackage{latexsym}
\usepackage{makeidx}
\usepackage{verbatim}
\usepackage{moreverb}
\usepackage{index}
\usepackage{dingbat}
\usepackage{fancyhdr}

\usepackage{ifpdf}

% \newif\ifpdf
% \ifx\pdfoutput\undefined
% \else
%   \ifx\pdfoutput\relax
%   \else
%     \ifcase\pdfoutput
%     \else
%       \pdftrue
%     \fi
%   \fi
% \fi

\ifpdf
  \usepackage[pdftex,colorlinks=true,bookmarksopen, pdfstartview=FitH,
              linkcolor=blue, citecolor=blue, urlcolor=blue]{hyperref}
  \pdfcompresslevel=9
  \usepackage[pdftex]{graphicx}
\else
  \usepackage[dvips]{graphicx}
\fi

\usepackage{ae}
\usepackage{aecompl}

\usepackage{hyperref}

% HORIZONTAL MARGINS
% Left margin, odd pages: 1.25 inch (0.25 + 1)
\setlength{\oddsidemargin}{0.25in}
% Left margin, even pages: 1.25 inch (0 + 1)
\setlength{\evensidemargin}{0.25in}
% Text width 6 inch (so other margin is 1.25 inch).
\setlength{\textwidth}{6in}
% ----------------
% VERTICAL MARGINS
% Top margin 0.5 inch (-0.5 + 1)
\setlength{\topmargin}{-0.5in}
% Head height 0.25 inch (where page headers go)
\setlength{\headheight}{0.25in}
% Head separation 0.25 inch (between header and top line of text)
\setlength{\headsep}{0.25in}
% Text height 9 inch (so bottom margin 1 in)
\setlength{\textheight}{9in}
% ----------------
% PARAGRAPH INDENTATION
\setlength{\parindent}{0in}
% SPACE BETWEEN PARAGRAPHS
\setlength{\parskip}{\medskipamount}
% ----------------
% STRUTS
% HORIZONTAL STRUT.  One argument (width).
\newcommand{\hstrut}[1]{\hspace*{#1}}
% VERTICAL STRUT. Two arguments (offset from baseline, height).
\newcommand{\vstrut}[2]{\rule[#1]{0in}{#2}}
% ----------------
% HORIZONTAL LINE ACROSS PAGE:
\newcommand{\hdivider}{\noindent\mbox{}\hrulefill\mbox{}}
% ----------------
% EMPTY BOXES OF VARIOUS WIDTHS, FOR INDENTATION
\newcommand{\hm}{\hspace*{1em}}
\newcommand{\hmm}{\hspace*{2em}}
\newcommand{\hmmm}{\hspace*{3em}}
\newcommand{\hmmmm}{\hspace*{4em}}
% ----------------
% ``TIGHTLIST'' ENVIRONMENT (no para space betwee items, small indent)
\newenvironment{tightlist}%
{\begin{list}{$\bullet$}{%
    \setlength{\topsep}{0in}
    \setlength{\partopsep}{0in}
    \setlength{\itemsep}{0in}
    \setlength{\parsep}{0in}
    \setlength{\leftmargin}{1.5em}
    \setlength{\rightmargin}{0in}
    \setlength{\itemindent}{0in}
}
}%
{\end{list}
}

% ----------------

\newcommand{\ie}{\emph{i.e.,}}

%----------------
% Added to get special characters in index

\newcommand{\Caret}{\char94}
\newcommand{\Tilde}{\char126}

% ----------------
% BSV names

\newcommand{\LibRefGuide}{\emph{Libraries Reference Guide}}
\newcommand{\LibRefGuideFullName}{\emph{Bluespec Compiler (BSC) Libraries Reference Guide}}

\newcommand{\BS}{Bluespec}
\newcommand{\BSInc}{Bluespec, Inc.}

\newcommand{\BH}{BH}
\newcommand{\BHFull}{Bluespec Haskell/Classic}

\newcommand{\BSVFull}{Bluespec SystemVerilog}
\newcommand{\BSV}{BSV}

\newcommand{\bsc}{\emph{bsc}}

\newcommand{\SV}{SystemVerilog}
\newcommand{\SVThreeOneA}{SystemVerilog 3.1a}
\newcommand{\SC}{SystemC}

\newcommand{\V}{Verilog}
\newcommand{\veri}{Verilog}
\newcommand{\VOrig}{Verilog 1995}
\newcommand{\VTwoK}{Verilog 2001}

% ----------------------------------------------------------------
% CODE DISPLAYS.
% Bluespec code displays are enclosed between \BBS and \EBS
% Most characters are taken verbatim, in typewriter font,
% Except:
%  Commands are still available (beginning with \)
%    but use ` and ' instead of { and }
%  Math mode is still available (beginning with $)
%    but use ~ and ! for ^ and _

\outer\def\BBS{%
  \begin{list}{$\bullet$}{%
    \setlength{\topsep}{0in}
    \setlength{\partopsep}{0in}
    \setlength{\itemsep}{0in}
    \setlength{\parsep}{0in}
    \setlength{\leftmargin}{1em}
    \setlength{\rightmargin}{0in}
    \setlength{\itemindent}{0in}
  }\item[]
%  \catcode`\{=12
%  \catcode`\}=12
  \catcode`\&=12
  \catcode`\#=12
  \catcode`\%=12
  \catcode`\~=12
%  \catcode`\_=12
  \catcode`\^=12
%  \catcode`\~=7
%  \catcode`\!=7      % superscript
%  \catcode`\'=2
%  \catcode`\`=1
  \obeyspaces
  \obeylines \tt}

\outer\def\EBS{%
  \end{list}
  }

{\obeyspaces\gdef {\ }}

% ----------------------------------------------------------------
% Editorial notes are enclosed between \begin{NOTE} and \end{NOTE}

\newenvironment{NOTE}{%
  \hm{\bf{Note}}
  \begin{list}{$\bullet$}{%
    \setlength{\topsep}{0in}
    \setlength{\partopsep}{0in}
    \setlength{\itemsep}{0in}
    \setlength{\parsep}{0in}
    \setlength{\leftmargin}{2em}
    \setlength{\rightmargin}{0in}
    \setlength{\itemindent}{0in}
  }\item[]
  \sf
 }
 {\end{list}\hm{\bf{End of Note}}}

% ----------------------------------------------------------------
% The following hack is from Mark Tuttle
\newcommand{\ttsymbol}[1]{%
  % print character at position #1 in the tt font family in current font size
  \begingroup\fontfamily{cmtt}\selectfont\symbol{#1}\endgroup
}
% ----------------
% BANG (tt font exclamation mark), can be used inside \fbox environment
\newcommand{\BANG}{\ttsymbol{33}}
% ----------------
% HASH (tt font hash), can be used inside \fbox environment
\newcommand{\HASH}{\ttsymbol{35}}
% ----------------
% BSL (tt font backslash), can be used inside \fbox environment
\newcommand{\BSL}{\ttsymbol{92}}
% ----------------
% HAT (tt font hat), can be used inside \fbox environment
\newcommand{\HAT}{\ttsymbol{94}}
% ----------------
% UNDERSCORE of standard char width (normal tt font \_ is narrower)
\newcommand{\US}{\ttsymbol{95}}
% ----------------
% TILDE (tt font tilde), can be used inside \fbox environment
\newcommand{\TILDE}{\ttsymbol{126}}
% ----------------
% LBRACE (tt font left brace), can be used inside \fbox environment
\newcommand{\LBRACE}{\ttsymbol{123}}
% ----------------
% BAR (tt font vertical bar), can be used inside \fbox environment
\newcommand{\BAR}{\ttsymbol{124}}
% ----------------
% RBRACE (tt font right brace), can be used inside \fbox environment
\newcommand{\RBRACE}{\ttsymbol{125}}
% ----------------------------------------------------------------

% Library environment.  Used by generated code.
\newenvironment{libverbatim}
  {\vspace*{-1.0em}
   \verbatim}
  {\endverbatim
  }

\newenvironment{smcenterboxverbatim}
  {\center
   \small
   \boxedverbatim}
  {\endboxedverbatim 
  {\endcenter} }

\newenvironment{centerboxverbatim}
  {\center
   \boxedverbatim}
  {\endboxedverbatim 
  {\endcenter }}


% ----------------------------------------------------------------

\newcommand\lineup{\vspace*{-0.6em}}

\newcommand\com[1]{}
\newcommand{\te}[1]{\texttt{#1}}
\newcommand{\nterm}[1]{\emph{#1}}
\newcommand{\term}[1]{{\tt{#1}}}
\newcommand{\many}[1]{\{ #1 \}}
\newcommand{\opt}[1]{[ #1 ]}
\newcommand{\alt}{{$\mid$}}
\newcommand{\gram}[2]{    \hm\makebox[10em][l]{\it #1}\makebox[1.5em][l]{::=}    #2}
\newcommand{\grammore}[1]{\hm\makebox[10em][l]{      }\makebox[1.5em][l]{}       #1}
\newcommand{\gramalt}[1]{ \hm\makebox[10em][l]{      }\makebox[1.5em][l]{\alt}   #1}

\newcommand{\tbd}[1]{{\sf TBD: #1}}
\newcommand{\note}[1]{\vspace*{2mm}{\sf {\large \bf Note\\} #1}\vspace{2mm}}

\newcommand{\begindescrlist}[1]{
\begin{list}{\arabic{enumi}}{
                \settowidth{\labelwidth}{#1}
                \setlength{\leftmargin}{\labelwidth} % {#1}
                \addtolength{\leftmargin}{\labelsep}
                \setlength{\parsep}{0ex}
                \setlength{\itemsep}{0ex}
                \usecounter{enumi}
        }
}
\newcommand{\litem}[1]{\item[#1\hfill]}

% ``Quoted'' inline bluespec
\newcommand{\qbs}[1]{``\mbox{\te{#1}}''}

\newcommand{\obsolete}[1]{}

\makeindex

\title{
\resizebox{2in}{!}{\includegraphics[width=\textwidth]{../common/B-Lang}} \\
\vspace{0.3in}
{\BH}$^{\rm{TM}}$ ({\BS} Haskell/Classic) \\
\mbox{} \\
Language Reference Guide \\
\vspace*{2in}
\mbox{}
}

% Revision id, major copyrights
\author{Revision: 3 February 2022}

\date{
Copyright {\copyright}
\begin{tabular}[t]{ll}
2000 -- January 2020: & Bluespec, Inc. \\
January 2020 onwards: & various open-source contributors
\end{tabular}
}


\ifpdf
\hypersetup{
pdfauthor = {Bluespec, Inc.},
pdftitle = {Bluespec Haskell (BH) (TM) Reference Guide},
pdfsubject = {Bluespec},
pdfkeywords = {Bluespec},
pdfcreator = {Bluespec}}
\else
\fi

\begin{document}

% ----------------

\maketitle

% ================================================================

\pagestyle{fancy}

\lhead[Reference Guide]{BH}
\rhead[BH]{Reference Guide}


%\lfoot[\thepage]{}
\cfoot{\thepage}
%\rfoot[]{\thepage}

% ----------------

\newpage

{\large\bf Trademarks and copyrights}

Verilog is a trademark of IEEE (the Institute of Electrical and
Electronics Engineers).  The Verilog standard is copyrighted, owned
and maintained by IEEE.

VHDL is a trademark of IEEE (the Institute of Electrical and
Electronics Engineers).  The VHDL standard is copyrighted, owned and
maintained by IEEE.

SystemVerilog is a trademark of IEEE.  The SystemVerilog standard is
owned and maintained by IEEE.

SystemC is a trademark of IEEE.  The SystemC standard is owned and
maintained by IEEE.

Bluespec is a trademark of Bluespec, Inc.

% ================================================================

\newpage

\clearpage
\phantomsection
\addcontentsline{toc}{section}{Table of Contents}

\tableofcontents

% The following two commands are a work-around for some bug
% seemingly introduced by the fancyhdr package.  Without this,
% the entries on the last page of the table of are spread
% vertically across the page, i.e., the linespacing is
% screwed up.  This work-around seems to fix it.

\vfill

\hm

\newpage

% ================================================================

\section{Introduction}

{\BH} ({\BHFull}) is a language for hardware design.  The language borrows its
notation, type and package system from an existing general-purpose
functional programming language called Haskell
{\cite{haskell12}}{\index{Haskell}} where those constructs have been
well tested for over a decade.  Unlike Haskell, {\BH} is meant
solely for hardware design--- a {\BH} program represents a circuit.
The abstract model for these circuits is a Term Rewriting System
(TRS); details about using TRSs for describing circuits, and compiling
these descriptions to real hardware, may be found in James Hoe's
thesis {\cite{jhoe}}.  {\BH} has several restrictions and extensions
relative to Haskell, arising out of this hardware focus.

This document is not meant as a tutorial on {\BH} (separate
documents exist for that purpose).  Nevertheless, this document has
numerous small examples to explicate {\BH} notation.

% ----------------------------------------------------------------

\subsection*{Meta notation}

The grammar
\index{grammar}
\index{meta notation|see{grammar}}
rules in the
presentation below mostly follow the usual EBNF (Extended BNF)
structure.  Grammar alternatives are separated by ``{\alt}''.  Items
enclosed in \opt{} are optional.  Items enclosed in {\many{}} can be
repeated zero or more times.  The last piece of notation is used
sloppily; sometimes there must be at least one item, and also, the
last terminal inside the {\many{}} is sometimes a separator rather
than terminator.

% ----------------------------------------------------------------

\subsection*{Identifiers and the r\^ole of upper and lower case}

An identifier {\index{identifiers|textbf}}in {\BH} consists of a letter
followed by zero or more letters, digits, underscores
\index{underscore|see{\qbs{\US}}}\index{_@\te{\US}!in identifiers}and single quotes.
\index{=squote@\te{'} (character, in identifiers)}
Identifiers are case sensitive: {\tt{glurph}},
{\tt{gluRph}} and {\tt{Glurph}} are three distinct identifiers.

The case of the first letter in an identifier
\index{identifiers!case of first letter}
is very important.  If the first letter is lower
case, the identifier is a ``variable identifier'', referred to in the
grammar rules as a {\nterm{varId}}.
\index{varId@\te{varId} (grammar terminal)|textbf}
If the first letter is upper case,
the identifier is a ``constructor
identifier'',
\index{identifiers!constructor}
referred to in the
grammar rules as a
{\nterm{conId}}.
\index{conId@\nterm{conId} (grammar non-terminal)|textbf}

In {\BH}, package names (\nterm{packageId}),
\index{packageId@{\nterm{packageId}} (grammar non-terminal)}
type names (\nterm{tycon})
\index{tycon@\nterm{tycon} (grammar non-terminal)}
and value constructor names
\index{value constructor names}
are all constructor identifiers.  (Ordinary)
variables,
\index{variables}
field names
\index{field names}
and type variables
\index{type variables}
are all variable identifiers.
\index{identifiers!variable}

A lone underscore, {\qbs{\US}},
\index{=underscore@\te{\US}!``don't care'' pattern}
\index{=underscore@\te{\US}!``don't care'' expression}
is treated as a
special identifier--- it is used as a ``don't care''
pattern
\index{don't care (patterns and expressions)|see{\qbs{\US}}}
or expression (more details in Sections \ref{sec-exprs-wildcard} and
\ref{sec-patterns-var}).

% ----------------------------------------------------------------

\subsection*{The Standard Prelude}

{\index{Prelude@{\te{Prelude}}}}
The Standard Prelude is a predefined package that is imported
implicitly into every {\BH} package.  It contains a number of useful
predefined entities (types, values/functions, classes, instances,
etc.).  It is somewhat analogous to the combination of various ``.h''
files and standard libraries in C, except that in {\BH} no special
action is needed to import the prelude or to link it in.  We will
refer to the prelude periodically in the following sections, and there
are more details in appendix {\ref{sec-prelude}}.

% ----------------------------------------------------------------

\subsection*{Lexical syntax/layout}

\index{layout}
\index{indentation|see{layout}}
In {\BH}, there are various syntactic constructs that involve zero
or more items enclosed in braces and separated by semicolons:
\BBS
    \{  {\rm\emph{item}} ; {\rm\emph{item}} ; $\cdots$ ; {\rm\emph{item}} \}
\EBS
These braces and semicolons
\index{braces and semicolons|see{layout}}
can be omitted entirely if the components are laid out with proper
indentation.

Suppose the parser discovers a missing open brace (e.g., after the
keywords {\te{where}}, {\te{let}}, {\te{do}} and {\te{of}}).  Then,
the indentation of the next lexical element is remembered (and the
missing open brace is implicitly inserted before it).  For each
subsequent line, if it contains only whitespace or is indented more,
then it is treated as a continuation of the current item.  If it is
indented the same amount, it is treated as the beginning of the next
item ({\ie} a semicolon is inserted implicitly before the item).  If
it is indented less, then the list of items is considered to be
complete ({\ie} a closing brace is implicitly inserted).  An explicit
brace is never matched against an implicit one.  Thus, while using the
layout rule, if the parser encounters an explicit open brace, then it
does not resume using the layout rule for this list of items until it
has ``emerged'' past the explicit corresponding closing brace (a
construct nested inside this list of items may still use the layout
rule).

% ----------------------------------------------------------------

\subsection*{Comments in {\BH} programs}

In a {\BH} program, a {\emph{comment}} is legal as whitespace, and
may be introduced in two ways.  An {\emph{ordinary
comment}}{\index{comment!ordinary}}{\index{=minusminus@{\te{{-}{-}}} (ordinary
comment)}} is introduced by a lexical token consisting of two or more
consecutive dashes followed by a non-symbol, and extends up to and
including the end of the line.  (See Section
{\ref{sec-infix-applications}} for the list of symbols.)  Note: the
lexical token {\te{--->}} is a legal token in {\BH}, and since it
contains three consecutive dashes followed by a symbol, it does not
begin a comment.

A {\emph{nested comment}}{\index{comment!nested}} is introduced by the
lexeme {\qbs{\{-}}
{\index{=lbraceminus@\texttt{\LBRACE{}-} (open nested comment)}}
and extends until the next matching
{\qbs{-\}}},
{\index{=minusrbrace@\texttt{-\RBRACE{}} (close nested comment)}}
possibly spanning multiple lines.  A nested comment can itself contain
another nested comment; this nesting can be repeated to any depth.

In an ordinary comment, the character sequences {\qbs{\{-}} and
{\qbs{-\}}} have no special significance, and, in a nested comment, a
sequence of dashes has no special significance.

% ----------------------------------------------------------------

\subsection*{General organization of this document}

A concept that is pervasive in {\BH} is the notion of a
{\emph{type}}.  Every value expression in {\BH}, even a basic value
identifier, has a type, and the compiler does extensive static type
checking to rule out absurd use of values (such as taking the square
root of an IP address).  Types are discussed in section
{\ref{sec-type}}.

A {\BH} program consists of one or more packages.  These outermost
constructs are described in section {\ref{sec-packages}}.  As
explained later, a {\BH} package is a linguistic
namespace-management mechanism and does not have any direct
correlation with any hardware module being described by the program.
Hardware modules correspond to {\emph{modules}}, a particular type of
value in {\BH}.

Within each package is a collection of top-level definitions.  These
are described in section {\ref{sec-top-level-defs}}.

Amongst the top-level definitions are {\emph{value definitions}}
(section {\ref{sec-val-defs}}), which constitute the actual meat of
the code.  Value definitions are built around {\emph{expressions}},
which are described in section {\ref{sec-exprs}}.

% ================================================================

\section{Types}

\label{sec-type}

\index{types|textbf}
Every value expression and, in particular, every value identifier in
{\BH} has a {\emph{type}}.  In some cases the programmer must supply
a {\emph{type signature}}{\index{type signature}} specifying this and in many cases the
compiler infers it automatically.  The {\BH} programmer should be
aware of types at all times.

\gram{type}{\nterm{btype} \opt{ \term{->} \nterm{type} } } \\
\gram{btype}{\opt{\nterm{btype}} \nterm{atype}} \\
\gram{atype}{\nterm{tycon} {\alt} \nterm{tyvar} {\alt} \term{(} \many{\nterm{type} \term{,}} \term{)}} \\
\gram{tycon}{\nterm{conId}}
\index{tycon@\nterm{tycon} (grammar non-terminal)|textbf}

Most type expressions have the form:
\index{type constructor}
\BBS
{\rm\emph{TypeConstructor}}  $t_1$ $\cdots$ $t_n$
\EBS
where $t_1$ $\cdots$ $t_n$ are themselves type expressions, and $n
{\geq} 0$.  The $t_1$ $\cdots$ $t_n$ are referred to as the
{\emph{type arguments}} to the type constructor.  $n$ is also called
the {\emph{arity}}
\index{arity, of type constructor}
of the type constructor.

Familiar basic types have zero-arity type constructors (no type
arguments, $n = 0$).  Examples:
\begin{verbatim}
 Integer
 Bool
 String
 Action
\end{verbatim}
Other type constructors have arity $n > 0$; these are also known as
{\emph{parameterized types}}.
\index{types!parameterized}
Examples:
\BBS
List   Bool
List  (List Bool)
Array  Integer  String
Maybe  Integer
\EBS
These represent the types of lists of Booleans, lists of lists of
Booleans, arrays indexed by integers and containing strings, and
an optional result possibly containing an integer.

A type can be {\emph{polymorphic}},
\index{types!polymorphic}
indicated using type variables.
\index{type variables}
Examples:
\begin{verbatim}
 List   a
 List  (List b)
 Array  i  (List  String)
\end{verbatim}
These represent lists of things of some unknown type {\qbs{a}}, lists
of lists of things of some unknown type {\qbs{b}}, and arrays indexed
by some unknown type {\qbs{i}} and containing lists of strings.

One type constructor is given special status in the syntax.  The type
of functions from arguments of type $t_1$ to results of type $t_2$
could have been written as:
\BBS
Function  $t_1$  $t_2$
\EBS
but in {\BH} we write the constructor as an infix arrow:
\index{function types|textbf}
\index{arrow types|see{function types}}
\index{=minusgt@{\te{->}} (infix function type constructor)|textbf}
\BBS
$t_1$ -> $t_2$
\EBS
These associate to the right, {\ie}
\BBS
$t_1$ -> $\cdots$ -> $t_{n-1}$ -> $t_n$  \hmm  $\equiv$  \hmm  $t_1$ -> ($\cdots$ -> ($t_{n-1}$ -> $t_n$))
\EBS

There is one particular set of niladic type constructors that look
like numbers.
\index{size types}
These are used to represent certain ``sizes''.  For
example, the type:
\begin{verbatim}
 Bit 16
\end{verbatim}
consists of the unary type constructor {\te{Bit}} applied to type
represented by the niladic type constructor {\qbs{16}}.  The type as a
whole represents bit vectors of length 16 bits.  Similarly the type
\begin{verbatim}
 UInt 32
\end{verbatim}
represents the type of unsigned integers that can be represented in 32
bits.
These numeric types are said to have kind \texttt{\#},
rather than kind \texttt{*} for value types.

\index{string types}
Strings can also be used as type, having kind \texttt{\$}.
This is less common, but string types are quite useful in the generics library,
described in the \LibRefGuide.
Examples:
\begin{verbatim}
 MetaData#("Prelude","Maybe",PrimUnit,2)
 MetaConsNamed#("Valid",1,1)
\end{verbatim}

% ----------------------------------------------------------------

\subsection{Type classes and overloading}

\label{sec-overloading}

\index{overloading, of type}
{\BH}'s {\te{class}} and {\te{instance}} mechanisms form a
systematic way to do {\emph{overloading}} (the approach has been well
tested in Haskell).

Overloading is a way to use a common name to refer to a set of
operations at different types.  For example, we may want to use the
{\qbs{<}} operator name for the integer comparison operation, the
floating-point comparison operation, the vector comparison operation
and the matrix comparison operation.  Note that this is not the same
as polymorphism: a polymorphic function is a {\emph{single}} function
that is meaningful at an infinity of types ({\ie} at every possible
instantiation of the type variables in its type).  An overloaded
identifier, on the other hand, usually uses a common name to refer to
a (usually) small set of distinct operations.

Further, it may make sense to have {\qbs{<=}}, {\qbs{>}} and
{\qbs{>=}} operations wherever there is a {\qbs{<}} operation, on
integers, floating points numbers, vectors and matrices.  Rather than
handle these separately, we say:
\begin{tightlist}
\item
there is class of types which we will call {\te{Ord}} (for ``ordered
types''),
\index{type class|textbf}
\index{class|see{type class}}

\item
that the integer, floating point, vector and matrix types are members
(or ``instances'') of this class, and
\index{instance (of type class)|textbf}

\item
that all types that are members of this class have appropriate
definitions for the {\qbs{<}}, {\qbs{<=}}, {\qbs{>}} and {\qbs{>=}}
operations.  We also say that these operations are {\emph{overloaded}}
across these instance types, and we refer to these operations as the
{\emph{methods}} of this class.
\index{methods!of a type class|textbf}

\end{tightlist}
Another example: we could use a class {\te{Hashable}} with an
operation called {\te{hash}} to represent those types $T$ for which we
can and do define a hashing function.  Each such type $T$ has to
specify how to compute the {\te{hash}} function at that type.

Classes, and the membership of a type in a class, do not come into
existence by magic.  Every class is created explicitly using a class
declaration, described in section {\ref{sec-classes}}.  A type must
explicitly be made an instance of a class and the corresponding class
methods have to be provided explicitly; this is described in
{\ref{sec-instances}}.

% ----------------

\subsubsection{Context-qualified types}

\index{context-qualified types|textbf}
\index{types!context-qualified|see{context-qualified types}}
Consider the following type declaration:
\begin{verbatim}
 sort  ::  (Ord a) => List a -> List a
\end{verbatim}
It expresses the idea that a sorting function takes an (unsorted)
input list of items and produces a (sorted) output list of items, but
it is only meaningful for those types of items (\qbs{a}) for which the
ordering functions (such as {\qbs{<}}) are defined.  Thus, it is ok to
apply {\te{sort}} to lists of {\te{Integer}}'s or lists of
{\te{Bool}}'s, because those types are instances of {\te{Ord}}, but it
is not ok to apply {\te{sort}} to a list of, say, {\te{Counter}}'s
(assuming {\te{Counter}} is not an instance of the {\te{Ord}} class).

In the type of {\te{sort}} above, the part before {\qbs{=>}} is called
a {\emph{context}}.
\index{contexts|see{context-qualified types}}
A context expresses constraints on one or more
type variables--- in the above example, the constraint is that any
actual type {\qbs{a}} must be an instance of the {\te{Ord}} class.

A context-qualified type has the following grammar:

\gram{ctxType}{\opt{\nterm{context} \term{=>}} \nterm{type}}
\index{=eqgt@\te{=>} (in context-qualified types)|textbf}

\gram{context}{\term{(} \many{\nterm{classId} \many{ \nterm{varId} } \term{,} } \term{)}}

\gram{classId}{\nterm{conId}}

In the above example, the class {\te{Ord}} had only one type parameter
({\ie} it constrains a single type) but, in general, a type class can
have multiple type parameters.  For example, in {\BH} we frequently
use the class {\mbox{\qbs{Bits a n}}} which constrains the type
represented by {\te{a}} to be representable in bit strings of length
represented by the type {\te{n}}.

\begin{NOTE}
When using an overloaded identifier {\te{x}} there is always a
question of whether or not there is enough type information available
to the compiler to determine which of the overloaded {\te{x}}'s you
mean.  For example, if {\te{read}} is an overloaded function that
takes strings to integers or Booleans, and {\te{show}} is an
overloaded function that takes integers or Booleans to strings, then
the expression {\mbox{\te{show (read s)}}} is ambiguous--- is the
thing to be read an integer or a Boolean?

In such ambiguous situations, the compiler will so notify you, and you
may need to give it a little help by inserting an explicit type
signature, e.g.,
\begin{verbatim}
 show ((read s) :: Bool)
\end{verbatim}
\end{NOTE}

% ================================================================

\section{Packages}

\label{sec-packages}

\index{package|textbf}
Packages are the outermost constructs in {\BH}--- all {\BH} code
must be inside packages.  There should be one package per file.  A
{\BH} package is a linguistic device for namespace control, and is
particularly useful for programming-in-the-large.  A package does not
directly correspond to hardware modules.  (Hardware modules correspond
to {\BH} modules, described in section {\ref{sec-modules}}.)

A {\BH} package consists of the package header, import declarations,
and top level definitions.  The package header indicates which names
defined in this package are exported, {\ie} available for import into
other packages.

\gram{packageDefn}{ \term{package} \nterm{packageId} \term{(} \nterm{exportDecl} \term{)} \term{where} \term{\{} }
\index{package@\te{package} (keyword)}
\index{where@\te{where} (keyword)}
\\
\grammore{\many{ \nterm{importDecl} \term{;} }} \\
\grammore{\many{ \nterm{fixityDecl} \term{;} }} \\
\grammore{\many{ \nterm{topDefn} \term{;} }} \\
\grammore{\term{\}}}

\gram{exportDecl}{ \nterm{varId} {\alt} \nterm{typeId} \opt{\nterm{conList}} }

\gram{conList}{ \term{(..)} }

\gram{importDecl}{ \term{import} \opt{\term{qualified}} \nterm{packageId} }
\index{import@\te{import} (keyword)}

\gram{fixityDecl}{ \nterm{fixity} \nterm{integer} \nterm{varId} }

\gram{fixity} { \term{infix} {\alt} \term{infixl} {\alt} \term{infixr} }
\index{infix@\te{infix} (keyword)}
\index{infixl@\te{infixl} (keyword)}
\index{infixr@\te{infixr} (keyword)}

\gram{packageId}{\nterm{conId}}

Example:
\BBS
package Foo (x, y) where
\null
import Bar
import Glurph
\null
{\rm\emph{... top level definition ...}}
{\rm\emph{... top level definition ...}}
{\rm\emph{... top level definition ...}}
\EBS
Here, {\te{Foo}} is the name of this package, {\te{x}} and {\te{y}} are
names exported from this package (they will be defined amongst the top
level definitions in this package), and {\te{Bar}} and {\te{Glurph}}
are the names of package being imported (for use in this package).

\index{export, identifiers from a package|textbf}
\index{identifiers!export from a package}
\index{=lparendotdotrparen@\te{(..)} (exporting constructors and field names)}
The export list is a list of identifiers, each optionally followed by
\te{(..)}.  Each identifier in the list will be visible outside the
package.  If the exported identifier is the name of \te{data},
\te{struct}, or \te{interface}, then the constructors or fields of the
type will be visible only if \te{(..)}  is used.  Otherwise, if you
export only the name of a type without the \te{(..)} suffix, the type
is an abstract (opaque) data type outside the package.  The list of
identifiers may include identifiers defined in the package as well as
identifiers imported from other packages.

If the keyword \te{qualified} is present in the import declaration
all the imported entities from that package must be referred to
by a qualified name.

The fixity declaration can be used to give a precedence level to
a user-defined infix operator.  The \te{infixl} specifies a left
associative operator, \te{infixr} a right associative operator, and
\te{infix} a non-associative operator.

% ----------------------------------------------------------------

\subsection{Name clashes and qualified names}

\index{identifiers!qualified|textbf}
When used in any scope, a name must have an unambiguous meaning.  If
there is name clash for a name $x$ because it is defined in the
current package and/or it is available from one or more imported
packages, then the ambiguity can be resolved by using a qualified name
of the form $M.x$ to refer to the version of $x$ contained in package
$M$.

% ================================================================

\section{Top level definitions}

\label{sec-top-level-defs}

Top level definitions can be used only on the top level within a package.

% ----------------------------------------------------------------

\subsection{\te{data}}

\label{sec-data}

A {\te{data}} definition defines a brand new type, which is different
from every primitive type and every other type defined using a
{\te{data}} definition, even if they look structurally similar.  The
new type defined by a {\te{data}} definition is a ``sum of products'',
or a ``union of products''.

\gram{topDefn}{ \term{data} \nterm{typeId} \many{\nterm{tyVarId}} \term{=} \many{\nterm{summand} \term{|}} \opt{\nterm{derive}}}

\gram{summand}{\nterm{conId} \many{\nterm{type}}}

\gram{summand}{\nterm{conId} \term{\{} \many{ \nterm{fieldDef} \term{;} } \term{\}} }

\index{deriving@\te{deriving} (keyword)|textbf}
\gram{derive}{\term{deriving} \term{(} \many{ \nterm{classId} \term{,} } \term{)} }

\gram{fieldDef}{ \nterm{fieldId} \term{::} \nterm{type} }

The {\nterm{typeId}} is the name of this new type.  If the
{\nterm{tyVarId}}'s exist, they are type parameters, thereby making
this new type polymorphic.  In each {\nterm{summand}}, the
{\nterm{conId}} is called a ``constructor''.  You can think of them as
unique {\emph{tag}}'s that identify each summand.  Each
{\nterm{conId}} is followed by a specification for the fields involved
in that summand ({\ie} the fields are the ``product'' within the
summand).  In the first way of specifying a summand, the fields are
just identified by position, hence we only specify the types of the
fields.  In the second way of specifying a summand, the fields are
named, hence we specify the field names ({\nterm{fieldId}}'s) and
their types.

The same constructor name may occur in more than one type.  The same
field name can occur in more than one type.  The same field name can
occur in more than one summand within the same type, but the type of
the field must be the same in each summand.

The optional {\nterm{derive}} clause is used as a shorthand to make
this new type an instance of the {\nterm{classId}}'s, instead of using
a separate, full-blown {\te{instance}} declaration.  This can only be
done for certain predefined {\nterm{classId}}'s: {\te{Bits}},
{\te{Eq}}, and {\te{Bounded}}.  The compiler automatically derives the
operations corresponding to those classes (such as {\te{pack}} and
{\te{unpack}} for the {\te{Bits}} class).  Type classes, instances,
and {\te{deriving}} are described in more detail in sections
{\ref{sec-overloading}}, {\ref{sec-classes}} and
{\ref{sec-instances}}.

To construct a value corresponding to some {\te{data}} definition $T$,
one simply applies the constructor to the appropriate number of
arguments (see section {\ref{sec-exprs-constrs}}); the values of those
arguments become the components/fields of the data structure.

To extract a component/field from such a value, one uses pattern
matching (see section {\ref{sec-patterns}}).

Example:
\begin{verbatim}
 data Bool = False | True
\end{verbatim}
This is a ``trivial'' case of a {\te{data}} definition.  The type is
not polymorphic (no type parameters); there are two summands with
constructors {\te{False}} and {\te{True}}, and neither constructor has
any fields.  It is a 2-way sum of empty products.  A value of type
{\te{Bool}} is either the value {\te{False}} or the value {\te{True}}
Definitions like these correspond to an ``enum'' definition in C.

Example:
\begin{verbatim}
 data Operand = Register (Bit 5)
              | Literal  (Bit 22)
              | Indexed  (Bit 5) (Bit 5)
\end{verbatim}
Here, the first two summands have one field each; the third has two
fields.  The fields are positional (no field names).  The field of a
{\te{Register}} value must have type {\mbox{Bit 5}}.  A value of type
{\te{Operand}} is either a {\te{Register}} containing a 5-bit value,
or a {\te{Literal}} containing a 22-bit value, or an {\te{Indexed}}
containing two 5-bit values.

Example:
\begin{verbatim}
 data Maybe a = Nothing | Just a
              deriving (Eq, Bits)
\end{verbatim}
\index{Maybe@\te{Maybe} (type)|textbf}
\index{Nothing@\te{Nothing}|see{\te{Maybe}}}
\index{Just@\te{Just}|see{\te{Maybe}}}
This is a very useful and commonly used type.  Consider a function
that, given a key, looks up a table and returns some value associated
with that key.  Such a function can return either {\te{Nothing}}, if
the table does not contain an entry for the given key, of {\te{Just
$v$}}, if the table contains $v$ associated with the key.  The type is
polymorphic (type parameter {\qbs{a}}) because it may be used with
lookup functions for integer tables, string tables, IP address tables,
etc., {\ie} we do not want here to over-specify the type of the value
$v$ at which it may be used.

Example:
\begin{verbatim}
 data Instruction = Immediate { op::Op; rs::Reg; rt::CPUReg; imm::UInt16; }
                  | Jump      { op::Op; target::UInt26; }
\end{verbatim}

An {\te{Instruction}} is either an {\te{Immediate}} or a {\te{Jump}}.
In the former case, it contains
a field called {\te{op}} containing a value of type {\te{Op}},
a field called {\te{rs}} containing a value of type {\te{Reg}},
a field called {\te{rt}} containing a value of type {\te{CPUReg}}, and
a field called {\te{imm}} containing a value of type {\te{UInt16}}.
In the latter case, it contains
a field called {\te{op}} containing a value of type {\te{Op}}, and
a field called {\te{target}} containing a value of type {\te{UInt26}}.

\begin{NOTE}
Error messages involving data type definitions sometimes show traces
of how they are handled internally.  Data type definitions are
translated into a data type where each constructor has exactly one
argument.  The types above translate
to:
\begin{verbatim}
 data Bool = False PrimUnit | True PrimUnit

 data Operand = Register (Bit 5)
              | Literal (Bit 22)
              | Indexed Operand_$Indexed
 struct Operand_$Indexed = { _1 :: Reg 5; _2 :: Reg 5 }

 data Maybe a = Nothing PrimUnit | Just a

 data Instruction = Immediate Instruction_$Immediate
                  | Register Instruction_$Register
 struct Instruction_$Immediate = { op::Op; rs::Reg; rt::CPUReg; imm::UInt16; }
 struct Instruction_$Register = { op::Op; target::UInt26; }
\end{verbatim}
\end{NOTE}

% ----------------------------------------------------------------

\subsection{\te{struct}}

\label{sec-struct-type}

\index{structs!type definition|textbf}
Defines a record type (a ``pure product'').
\index{records|see{structs}}
\index{product types|see{structs}}
This is a
specialized form of a {\te{data}} definition. The same field name may
occur in more than one type.

\gram{topDefn}{ \term{struct} \nterm{typeId} \many{\nterm{tyVarId}} \term{=} \term{\{} \many{ \nterm{fieldDef} \term{;} } \term{\}}  \opt{\nterm{derive}}}
\index{struct@\te{struct} (keyword)}

\gram{fieldDef}{ \nterm{fieldId} \term{::} \nterm{type} }

Example:
\begin{verbatim}
 struct Proc = { pc :: Addr; rf :: RegFile; mem :: Memory }
 struct Coord = { x :: Int; y :: Int }
\end{verbatim}

Section {\ref{sec-struct-val}} describes how to construct values of a
{\te{struct}} type.  A field of a {\te{struct}} type can be extracted
either directly using ``dot'' notation (section
{\ref{sec-struct-select}}) or using pattern matching (section
{\ref{sec-patterns-struct}}).

% ----------------

\subsubsection{Tuples}

\label{sec-tuple-type}

\index{tuples!type definition|textbf}
One way to group multiple values together is to use a {\te{data}}
definition in which a constructor has multiple fields.

However, there is a built-in notation for a common form of grouping,
called ``tuples''.  To group two (or more) values together the Prelude
contains a type,
\te{PrimPair}, for which there is syntactic sugar for type expressions,
value expressions, and patterns.

The type has the following definition
\begin{verbatim}
 struct PrimPair a b = { fst :: a; snd :: b } deriving (Eq, Bits, Bounded)
\end{verbatim}
For type expressions the following shorthand can be used:
\BBS
(a, b)  \hmm  $\equiv$  \hmm  PrimPair a b
\EBS
Or, more generally,
\BBS
($t_1$, $t_2$, $\cdots$, $t_n$)  \hmm  $\equiv$  \hmm  PrimPair $t_1$ (PrimPair $t_2$ ($\cdots$ $t_n$))
\EBS

There is a corresponding shorthand for value expressions and patterns:
\index{tuples!values}
\index{tuples!patterns}
\BBS
(a, b)  \hmm  $\equiv$  \hmm  PrimPair \{ fst = a; snd = b \}
\EBS

There is also special syntax for the empty tuple.  It is written \qbs{()} for
types, expressions, and patterns.  The real type has the following definition
\begin{verbatim}
 struct PrimUnit = { } deriving (Eq, Bits, Bounded)
\end{verbatim}

% ----------------------------------------------------------------

\subsection{\te{type}}

Defines a type synonym.  These are used purely for readability, {\ie}
a type synonym can always be ``expanded out'' to its definition at any
time.

\gram{topDefn}{ \term{type} \nterm{typeId} \many{\nterm{tyVarId}} \term{=} \nterm{type} }

Examples:
\begin{verbatim}
 type Byte     = Bit  8
 type Word     = Bit 16
 type LongWord = Bit 32
\end{verbatim}
These provide commonly used names for certain bit lengths.  In a
specification of a processor:
\begin{verbatim}
 data RegName = R0 | R1 | ... | R31
 type Rdest = RegName
 type Rsrc  = RegName
 data ArithInstr = Add Rdest Rsrc Rsrc
                 | Sub Rdest Rsrc Rsrc
\end{verbatim}
the last two lines suggest the roles of the registers in the
instructions, and is more readable than:
\begin{verbatim}
 data ArithInstr = Add RegName RegName RegName
                 | Sub RegName RegName RegName
\end{verbatim}

% ----------------------------------------------------------------

\subsection{\te{interface}}

\index{interfaces|textbf}
Defines an interface for a hardware module (see section
\ref{sec-modules}).  An interface is essentially a {\te{struct}},
but its components are restricted to those things that have a physical
interpretation as wires in and out of a circuit.  The types of fields
in an interface are more likely to involve {\te{Action}}'s (see
section {\ref{sec-actions}}), which are typically interpreted as
``enable signals'' into a circuit.  The fields of an interface are
also known as {\emph{methods}}
\index{methods!of an interface|textbf}
(not to be confused with methods of a class, described in Sections
{\ref{sec-overloading}} and {\ref{sec-classes}}).

\gram{topDefn}{ \term{interface} \nterm{typeId} \many{\nterm{tyVarId}} \term{=} \term{\{} \many{ \nterm{fieldDef} \term{;} } \term{\}} }
\index{interface@\te{interface} (keyword)!in interface definitions|textbf}

Example:
\begin{verbatim}
 interface Stack a =
     push :: a -> Action
     pop :: Action
     top :: Maybe a
\end{verbatim}
This describes a circuit that implements a stack (a LIFO) of items.
This polymorphic definition does not specify the type of the contents
of the stack, just that they have some type {\qbs{a}}.  Corresponding
to the {\te{push}} method, the circuit will have input wires to carry
a value of type {\qbs{a}}, and a ``push-enable'' input wire that
specifies when the value present on the input wires should be pushed
on the stack.  Corresponding to the {\te{pop}} component, the circuit
will have a ``pop-enable'' input wire that specifies when a value
should be popped off the stack.  Corresponding to the {\te{top}}
component, the circuit will have a set of output wires: if the stack
is empty, the wires will represent the value {\te{Nothing}}, and if
the stack is non-empty and $v$ is the value at the top of the stack,
the wires will represent {\te{Maybe $v$}}.

% ----------------------------------------------------------------

\subsection{{\te{class}} declarations}

\label{sec-classes}

The general concepts behind classes, instances, overloading etc.
were introduced in section {\ref{sec-overloading}}.  A new class is
declared using the following:

\gram{topDefn}{\term{class} \opt{\nterm{context} \term{=>}} \nterm{classId} \many{\nterm{tyVarId}} \opt{\term{|} \nterm{funDep}}
        \term{where} \term{\{}} \\
\grammore{ \many{\nterm{varId} \term{::} \nterm{ctxType} \term{;}} } \\
\grammore{ \term{\}} }

{\nterm{classId}} is the newly declared class.  It can be polymorphic,
if {\nterm{tyVarId}}'s exist; these are called the {\emph{parameters}}
of the type class.  The {\nterm{tyVarId}}'s may themselves be
constrained by {\nterm{context}}, in which case the classes named in
{\nterm{context}} are called the ``super-classes'' of this class.  The
``{\nterm{varId}\term{::}\nterm{ctxType}}'' list declares the class
method names and their types.

Example (from the Prelude):
\begin{verbatim}
 class Literal a where
     fromInteger :: Integer -> a
\end{verbatim}
This defines the class {\te{Literal}}.  It says that any type {\te{a}}
in this class must have a method (a function) called
{\te{fromInteger}} that converts an {\te{Integer}} value into the type
{\te{a}}.  In fact, this is the mechanism the {\BH} uses to
interpret literal constants, e.g., to resolve whether a literal like
{\te 6847} is to be interpreted as a signed integer, an unsigned
integer, a floating point number, a bit value of 10 bits, a bit value
of 8 bits, etc.  (This is described in more detail in Section
\ref{sec-exprs-constrs}.)

Example (from the Prelude):
\begin{verbatim}
 class (Literal a) => Arith a where
     (+) :: a -> a -> a
     (-) :: a -> a -> a
     negate :: a -> a
     (*) :: a -> a -> a
\end{verbatim}
This defines the class {\te{Arith}} with super-class {\te{Literal}}.
It says that for any type {\te{a}} that is a member of the class
{\te{Arith}}, it must also be a member of the class {\te{Literal}},
and it must have four methods with the given names and types.  Said
another way, an {\te{Arith}} type must have a way to convert integer
literals into that type, and it must have addition, subtraction,
negation and multiplication defined on it.

The optional {\nterm{funDep}} section specifies {\emph{functional
dependencies}} between the parameters of the type class:

\gram{funDep}{\many{ \many{\nterm{tyVarId}} \term{->} \many{\nterm{tyVarId}} \term{,} }}

These declarations specify that a type parameter may be determined
uniquely by certain other type parameters.  For example:
\begin{verbatim}
 class Add x y z | x y -> z, y z -> x, z x -> y
\end{verbatim}
Here, the class declaration says that for any triple of types
{\te{x}}, {\te{y}} and {\te{z}} that are in the class {\te{Add}}, any
two of the types uniquely determines the remaining type, {\ie}
\begin{tightlist}
\item
{\te{x}} and {\te{y}} uniquely determine {\te{z}},

\item
{\te{y}} and {\te{z}} uniquely determine {\te{x}}, and

\item
{\te{z}} and {\te{z}} uniquely determine {\te{y}}.
\end{tightlist}

See section {\ref{sec-primitives-size-types}} for more detailed
insights into the use of functional dependencies.

\begin{NOTE}
Functional dependencies are not currently checked by the compiler.
\end{NOTE}

% ----------------------------------------------------------------

\subsection{{\te{instance}} declarations}

\label{sec-instances}

A type can be declared as an instance of a class in two ways.  The
general mechanism is the {\te{instance}} declaration; a convenient
shortcut that can sometimes be used is the {\te{deriving}} mechanism.

The general {\te{instance}} declaration grammar is the following:

\gram{topDefn}{\term{instance} \nterm{context} \term{=>} \nterm{classId} \many{\nterm{type}}
        \term{where}} \\
\grammore{\term{\{} \many{\nterm{localDefn} \term{;}} \term{\}} }

This can be read as saying that the type {\nterm{type}} is an instance
of class {\nterm{classId}}, provided the constraints of
{\nterm{context}} hold, and where the {\nterm{localDefn}}'s specify
the implementation of the methods of the class.

Sometimes, when a new type is defined using a {\te{data}} declaration,
it can simultaneously be made a member of certain useful, predefined
classes, allowing the compiler to choose the ``obvious''
implementation of the class methods.  This is done using the
{\te{deriving}} qualification to a {\te{data}} declaration (described
in section {\ref{sec-data}}) or to a {\te{struct}} declaration
(described in section {\ref{sec-struct-type}}).  The only classes for
which {\te{deriving}} can be used for general types are {\te{Bits}}, {\te{Eq}} and
{\te{Bounded}}.  Furthermore, {\te{deriving}} can be used for any class
if the type is a data type that is isomorphic to a type that has an
instance for the derived class.

% ----------------

\subsubsection{Deriving \te{Bits}}

\label{sec-deriving-Bits}

\index{Bits@\te{Bits} (type class)!\te{deriving}}
\index{Bits@\te{Bits} (type class)!representation of data types}
The instances derived for the \te{Bits} class can be described as follows:
\begin{tightlist}

\item
For a \te{struct} type it is simply the the concatenation of the bits for
all the fields.  The first field is in the leftmost (most significant) bits, and so on.

\item
For a \te{data} type, all values of the type occupy the same number of
bits, regardless of which disjunct (constructor) it belongs to.  This
size is determined by the largest disjunct.  The leftmost (most
significant) bits are a code (a tag) for the constructor.  As few bits
as possible are used for this.  The first constructor in the
definition is coded 0, the next constructor is coded 1, and so on.
The size of the rest of the bits is determined by the largest numbers
of bits needed to encode the fields for the constructors.  For each
constructor, the fields are laid out left to right, and the
concatenated bits are stored right justified ({\ie} at the least
significant bits).  For disjuncts that are smaller than the largest
one, the bits between the constructor code and the field bits, if any,
are ``don't care'' bits.

\end{tightlist}

Examples: The type
\BBS
data Bool = False | True
\EBS
uses one bit.  \te{False} is represented by 0 and \te{True} by 1.
\BBS
struct Tv� = \{ f�rsta :: Bit 8; andra :: Bit 16 \}
\EBS
uses 24 bits with \te{f�rsta} in the upper 8 bits and \te{andra} in the lower 16.
\BBS
data Maybe a = Nothing | Just a
\EBS
will use $1+n$ bits, where $n$ bits are needed to represent values of
type \te{a}.  The extra bit will be the most significant bit and it
will be 0 (followed by $n$ unspecified bits) for \te{Nothing} and 1
(followed by the $n$ bits for \te{a}) for \te{Just}.

% ----------------

\subsubsection{Deriving \te{Eq}}

The instances derived for the \te{Eq} class is the natural equality for
the type.  For a struct all fields have to be equal, for a data type the constructors
have to be equal and then all their parts.

% ----------------

\subsubsection{Deriving \te{Bounded}}

An instance for \te{Bounded} can be derived for an enumeration type,
{\ie} a data type where all constructors are niladic.  The \te{minBound} will
be the first constructor and the \te{maxBound} will be the last.

\te{Bounded} can also be derived for a \te{struct} type if all the field types of
the struct are \te{Bounded}.  The \te{minBound} will be the struct with all
fields having their respective \te{minBound}, and correspondingly for \te{maxBound}.

\subsubsection{Deriving for isomorphic types}

A data type with one constructor and one argument is isomorphic to its type
argument.  For such a type any one-parameter class can be used, in a {\te{deriving}},
for which there is an instance for the underlying type.

Example:
\BBS
data Apples = Apple (UInt 32) deriving (Literal, Arith)

five :: Apples
five = 5

eatApple :: Apples -> Apples
eatApple n = n - 1
\EBS

% ----------------------------------------------------------------

\subsection{Value definitions}

\label{sec-val-defs}

\index{value definitions|textbf}
A value definition defines the value of an identifier (which could be
a function).  Value definitions are the meat of a {\BH} program.

Value definitions consist of a type signature followed immediately by
one or more defining clauses:

\gram{topDefn}
     {\nterm{valueDefn}}

\gram{valueDefn}
     { \nterm{varId} \term{::} \nterm{ctxType} \term{;} } \\
\grammore{ \many{\nterm{clause} \term{;}} }

\gram{clause}
     {\nterm{varId} \many{\nterm{apat}} \opt{\term{when} \nterm{guard}} \term{=} \nterm{exp}}
\index{when@\te{when} (keyword)!in clauses of top-level definitions}
\index{guards!with clauses of top-level definitions}

The first line of a value definition is the type signature--- it
simply specifies that the identifier {\nterm{varId}} has the type
{\nterm{ctxType}}.  Subsequent lines define the value, one clause at a
time.  The {\nterm{varId}}'s on the left-hand side of the type
signature and on the left-hand side of each clause must all be the
same, {\ie} they collectively define a single {\nterm{varId}}.

Each clause defines part of the value, using pattern matching and
guards.  If there are patterns ({\nterm{apat}}'s) present, then the
{\nterm{varId}} being defined is a function, and the patterns
represent arguments to the function.  The {\nterm{guard}} is a list of
arbitrary predicates that may use identifiers bound in the patterns
(see Section {\ref{sec-guards}}). The clause should be read as
follows: if the function {\nterm{varId}} is applied to arguments that
match the corresponding {\nterm{apat}}'s (in which case, identifiers
in the {\nterm{apat}}'s are bound to the corresponding components of
the arguments), and if the predicates in the {\nterm{guard}} are true,
then the function returns the value of the expression {\nterm{exp}}.


Example:
\begin{verbatim}
 wordSize :: Integer
 wordSize = 16
\end{verbatim}
This simply defines the identifier {\te{wordSize}} to have type
{\te{Integer}} and value 16.

Example:
\begin{verbatim}
 not :: Bool -> Bool
 not True  = False
 not False = True
\end{verbatim}
This defines the classical Boolean negation function.  The type
signature specifies that {\te{not}} is a function with argument type
{\te{Bool}} and result type {\te{Bool}}.  After that, the first clause
specifies that if the argument matches the value {\te{True}} ({\ie} it
{\emph{is}} the value {\te{True}}), then it returns {\te{False}}.  The
final clause specifies that if the argument is {\te{False}} it returns
{\te{True}}.

Example:
\begin{verbatim}
 f :: Maybe Int -> Int -> Int
 f (Just x) y when x > 10, Just y' <- g y  = x + y'
 f _        _                              = 0
\end{verbatim}
(If necessary, please first remember the definition of the
{\te{Maybe}} type, introduced in section {\ref{sec-data}}).  The first
line specifies that {\te{f}} is a function of two arguments, of type
{\mbox{\te{Maybe Int}}} and {\te{Int}}, respectively, and that its
result has type {\te{Int}}.  The second line specifies that if the
first argument has the form {\te{Just x}} (in which case let us call
its component {\te{x}}), if the second argument is anything (let us
call it {\te{y}}), if {\te{x}}'s value is greater than 10, if the
result of applying {\te{g}} to {\te{y}} has the form {\mbox{\te{Just
y'}}} (in which case let us call the component {\te{y'}}), then the
result is the value of {\mbox{\te{x + y'}}}.  In all other cases, the
result is the value 0.  The bare underscores
{\index{=underscore@\te{\US}!``don't care'' pattern}}
{\index{=underscore@\te{\US}!``don't care'' expression}}
in the second line are
{\emph{wild-card}} patterns that match anything (described in section
{\ref{sec-patterns-var}}).

Clauses are attempted in order, from top to bottom, proceeding to the
next clause only if the pattern matching and guard evaluation fail.
Within each clause, pattern matching and guard evaluation are
attempted from left to right.  If no clause succeeds, then the system
will raise a ``pattern matching error''.
\index{pattern matching!error}

% ----------------------------------------------------------------

\subsection{Calling foreign functions}

\index{foreign functions, calling}
A function can be declared to be foreign which means that its
implementation is not in {\BH}.

\index{foreign@\te{foreign} (keyword)|textbf}
\gram{topDefn}{\term{foreign} \nterm{varId} \term{::} \nterm{type} \opt{ \term{=} \nterm{string} }
	\opt{ \term{,} \term{(} \many{\nterm{string}} \term{)} }
    }

The optional string gives the name of the external ``function'' to
use.  If no string is given the same name as the {\BH} name is used.
The optional strings in parentheses are the port names of the {\veri} module
that implements the function.  Without port names positional arguments
will be used.

Example:
\begin{verbatim}
 foreign countOnes :: Bit n -> Bit 32 = "pop_count"
\end{verbatim}

A call to \te{countOnes} will instantiate the {\veri} \te{pop{\US}count}
module.  It should have the same number of arguments (with the same
type) as the {\BH} function, {\em and} an additional trailing
argument which is the result.  If the function is (size) polymorphic
the instantiated types will be used as {\veri} parameters.

Example: using the declaration above an action, with the type of \te{x} being \te{Bit 5},
\begin{verbatim}
  y := countOnes x
\end{verbatim}
will translate to something like
\begin{verbatim}
  pop_count #(5) ires1(R_x, I_y);
\end{verbatim}

%\begin{NOTE}
%This feature may go away, requiring that {\veri} modules be imported
%as modules.
%\end{NOTE}

% ================================================================

\section{Expressions}

\label{sec-exprs}

As described in {\ref{sec-val-defs}}, expressions appear in the
right-hand sides of value definitions.

\index{exp@{\nterm{exp}} (grammar non-terminal)|textbf}
In the following {\nterm{exp}} stands for an arbitrary expression and
{\nterm{aexp}} for an atomic expression, {\ie} one that is
syntactically delimited.

% ----------------------------------------------------------------

\subsection{Applications}

Function application (a.k.a. a function call) is expressed just by the
juxtaposition of two expressions.  The first expression should
evaluate to a function value, and that function is applied to the
value of the second expression.

\index{application!of functions to arguments|textbf}
\gram{exp}{\nterm{exp} \nterm{aexp}}

Parentheses can be used freely for grouping.  By default, if
parentheses are omitted, function application associates to the left:
\index{application!associativity}
\BBS
f  x  y  z  \hmm  $\equiv$  \hmm  ((f  x)  y)  z
\EBS
{\ie} {\te{f}} is applied to {\te{x}}, producing a function which is
applied to {\te{y}} which produces a function which, in turn, is
applied to {\te{z}}.

% ----------------

\subsubsection*{Infix applications}

\label{sec-infix-applications}

\index{application!infix|textbf}
Infix operators (binary functions written between their arguments)
can be used for convenience.

\gram{exp}{\nterm{$exp_1$} \nterm{binop} \nterm{$exp_2$}} \\
\gram{binop}{\nterm{symbol} \many{\nterm{symbol}}} \\
\gram{symbol}{
                     \term{\BANG}
              {\alt} \term{\HASH}
              {\alt} \term{\$}
              {\alt} \term{\%}
              {\alt} \term{\&}
              {\alt} \term{*}
              {\alt} \term{+}
              {\alt} \term{.}
              {\alt} \term{/}
              {\alt} \term{<}
              {\alt} \term{=}
              {\alt} \term{>}
              {\alt} \term{?}
              {\alt} \term{\@}
              {\alt} \term{\BSL}
              {\alt} \term{\HAT}
              {\alt} \term{\BAR}
              {\alt} \term{-}
              {\alt} \term{\TILDE}
	      {\alt}
	}\\
\grammore{
% ISO 8859-1 symbols
	             \term{�}
%T1	      {\alt} \term{�}
	      {\alt} \term{�}
%T1	      {\alt} \term{�}
%T1	      {\alt} \term{�}
%T1	      {\alt} \term{�}
	      {\alt} \term{�}
	      {\alt} \term{�}
	      {\alt} \term{�}
	      {\alt} \term{�}
	      {\alt} \term{�}
	      {\alt} $\term{�}$
%T1	      {\alt} \term{�}
	      {\alt} \term{�}
	      {\alt} \term{�}
	      {\alt} \term{�}
	      {\alt} $\term{�}$
	      {\alt}
	} \\
\grammore{
	             $\term{�}$
	      {\alt} $\term{�}$
	      {\alt} \term{�}
	      {\alt} $\term{�}$
	      {\alt} \term{�}
	      {\alt} \term{�}
	      {\alt} \term{�}
	      {\alt} $\term{�}$
	      {\alt} \term{�}
	      {\alt} \term{�}
	      {\alt} \term{�}
	      {\alt} \term{�}
	      {\alt} \term{�}
	      {\alt} \term{�}
	      {\alt} $\term{�}$
	      {\alt} $\term{�}$
        }

\index{infix operators!predefined}
\index{infix operators!precedence}
\index{infix operators!associativity}
The following table lists the predefined operators with their
precedence and associativity (see the Standard Prelude in section
{\ref{sec-prelude}} for an explanation of what these operators do):
\begin{center}
  \begin{tabular}[t]{|c|c|c|}
    \hline
    operator     & precedence & associativity \\
    \hline
    \term{\$}    & 0  & Right \\
    \verb":="    & 1  & Right \\
    \verb"||"    & 2  & Right \\
    \verb"&&"    & 3  & Right \\
    \verb"|"     & 4  & Right \\
    \verb"&"     & 5  & Right \\
    \verb"=="    & 6  & n/a   \\
    \verb"/="    & 6  & n/a   \\
    \verb"<="    & 6  & n/a   \\
    \verb">="    & 6  & n/a   \\
    \verb"<"     & 6  & n/a   \\
    \verb">"     & 6  & n/a   \\
    \verb"<<"    & 7  & n/a   \\
    \verb">>"    & 7  & n/a   \\
    \verb"++"    & 8  & Right \\
    \verb":>"    & 8  & Right \\
    \verb"+"     & 10  & Left  \\
    \verb"-"     & 10  & Left  \\
    \verb"*"     & 11  & Left  \\
    \verb"/"     & 11  & Left  \\
    \verb"�"     & 13  & Right \\
    user-defined & 15  & Left \\
    \hline
  \end{tabular}
\end{center}
The last line indicates that any user-defined infix operator has
higher precedence than any predefined operator, and it always
associates to the left.  Function application by juxtaposition always
has higher precedence than all operators, and associates to the left.
Constructs that do not have any closing lexeme (such as
{\te{if-then-else}} or {\te{let-in}}) have lowest precedence so that,
for example,
\begin{verbatim}
    if ... then ... else e1 + e2
\end{verbatim}
parenthesizes as follows:
\begin{verbatim}
    if ... then ... else (e1 + e2)
\end{verbatim}
and not as follows:
\begin{verbatim}
    (if ... then ... else e1) + e2
\end{verbatim}

\index{infix operators!defining new}
The user can define new infix operators by following the above syntax.
For example, here is a new infix operator {\te{|-|}} that ``clips'' a
value to a $[-{\rm{limit}}, +{\rm{limit}}]$ range:
\begin{verbatim}
 |-| :: Int -> Int ->Int

 x  |-|  lim when x < 0-lim = 0-lim
 x  |-|  lim when x >   lim =   lim
 x  |-|  _                  =   x
\end{verbatim}

\index{infix operators!converting to ordinary identifiers}
\index{infix operators!converting from ordinary identifiers}
\index{identifiers!converting to infix operators}
An infix operator can be converted into an ordinary identifier
(eliminating its special syntactic role) by enclosing it in
parentheses.  Conversely, an ordinary identifier representing a binary
function can be used in infix position by enclosing it in back-quotes.
\index{=backtick@\te{`}|see{infix operators, converting from ordinary identifiers}}

Examples:
\begin{verbatim}
 f x y
 x + y
 f x `max` g y
 (+) 1 2
\end{verbatim}

% ----------------------------------------------------------------

\subsection{Variables}

A variable in an expression simply represents its value.

\gram{aexp}{\nterm{varId}}
\index{variables}

Remember that variable names start with a lower case letter.

% ----------------------------------------------------------------

\subsection{Constructors and literal constants}

\label{sec-exprs-constrs}

\index{value constructor names}
\index{constants}
Each value constructor, introduced in some {\te{data}} declaration (section
{\ref{sec-data}}), is a constant.

\gram{aexp}{\nterm{conId}}

Suppose a constructor {\te{C}} is declared as follows:
\begin{verbatim}
 data T a b = ... | C t0 t1 t2 | ...
\end{verbatim}
Then, the constructor identifier {\te{C}} is a constant whose value is
a function of type:
\begin{verbatim}
 C :: t0 -> t1 -> t2 -> T a b
\end{verbatim}
If {\te{C}} had no parameters, then it represents a traditional
(non-functional) constant value.

Remember that constructor names start with an upper case letter.

\index{Literals}
Literal constants are constants supported with special syntax and with
overloading support.  {\BH} has support for integer and string
literals.

% ----------------

\subsubsection{Integer literals}

\label{sec-integer-literals}

\index{Literals!Integer}
\index{Integer literals}
Integer literals are written in the usual way as a sequence of decimal
digits (0-9), or \te{0x} followed by a sequence of hexadecimal digits
(0-9, a-f, A-F), or \te{0b} followed by a sequence of binary digits
(0,1).

There is no direct notation for negative integer literals--- use the
expression {\te{(negate {\rm <literal>})}} instead.

\begin{NOTE}
Constants with base 2 must have a type of the form $t\;n$ where $t$ is
a type like \te{Bit} or \te{UInt} and $n$ is the number of digits in
the literal.
\end{NOTE}

\gram{aexp}{\nterm{int}}

Example:
\begin{verbatim}
 125
 0x48454a
 0b101010
\end{verbatim}

Since {\BH} has several integer-like numeric types (of various bit
widths), a numeric literal $i$ is really shorthand for
{\te{fromInteger($i$)}}, where $i$ is treated as belonging to
{\te{Integer}}, the type of (arbitrary precision) integers.  The
{\te{fromInteger}} method belongs to the class {\te{Literal}}:
\index{fromInteger@\te{fromInteger} (\te{Literal} class method)|textbf}
\begin{verbatim}
 class Literal a where
     fromInteger :: Integer -> a
\end{verbatim}
The normal overloading-resolution mechanism (see section
{\ref{sec-overloading}}) is used by the compiler to figure out what
type the literal should be converted into.  As usual, if necessary you
can insert a type signature to help the compiler resolve this.

% ----------------

\subsubsection{String literals}

\label{sec-string-literals}

\index{Literals!String}
\index{String literals}

String literals are written enclosed in double quotes \te{"}.
Special characters may be inserted in string literals with the
the following backslash notations:
\begin{tabbing}
 \te{{\BSL}n}    \hmmmm \= newline \\
 \te{{\BSL}t}           \> tab \\
 \te{{\BSL}{\BSL}}      \> backslash \\
 \te{{\BSL}"}           \> double quote \\
 \te{{\BSL}x$HH$}       \> any hexadecimal character code $HH$
\end{tabbing}

% ----------------------------------------------------------------

\subsection{\te{case} and {\te{if}}}

\label{sec-case}

Case expressions can be used to scrutinize values of a data type--- to
discover which disjunct it conforms to, and to bind names to the
components of that disjunct.

\gram{exp}{\term{case} \nterm{exp} \term{of} \term{\{} }
\index{case expression}
\index{case@\te{case} (keyword)}
\index{of@\te{of} (keyword)} \\
\grammore{ \many{\nterm{caseArm}} } \\
\grammore{ \term{\}} }

\gram{caseArm}{\nterm{pat} \opt{\term{when} \nterm{guard}} \term{->} \nterm{exp} \term{;}}

The value of the expression (first {\nterm{exp}}) is tested against
the patterns and guards of each case arm in succession, top to bottom.
At each case arm, the value is matched against the pattern; if it
succeeds, the pattern identifiers are bound to the respective
components, and the guard expressions are evaluated, left to right.
If they are all true, then the case arm is successful, and the value
of the right-hand side expression is returned as the value of the
entire {\te{case}} expression.  If none of the case arms succeed, the
result is unspecified.
\index{pattern matching!error}

Example (uses the {\te{Maybe}} type definition of section
{\ref{sec-data}}):
\begin{verbatim}
 case f a of {
     Just x when x  < 0 -> negate x;
     Just x when x >= 0 -> x;
     Nothing            -> 0;
 }
\end{verbatim}
First, the value of {\mbox{\te{f a}}} is obtained.  In the first arm,
the value is checked to see if it has the form {\mbox{\te{Just x}}},
in which case we let {\te{x}} refer to the component.  Then, we check
if this {\te{x}} is less than zero.  If so, then the case arm succeeds
and we return {\mbox{\te{negate x}}} as the value.  Otherwise, we fall
through to the second case arm, and so on.

Booleans can be tested with an \te{if} expression (also known as
conditional expressions):

\gram{exp}{\term{if} \nterm{exp$_1$} \term{then} \nterm{exp$_2$} \term{else} \nterm{exp$_3$}}
\index{conditional expressions|see{if-then-else expressions}}
\index{if-then-else expressions|textbf}
\index{if@\te{if} (keyword)}
\index{then@\te{then} (keyword)}
\index{else@\te{else} (keyword)}

This is just a convenient and familiar shorthand for:
\BBS
case \nterm{exp$_1$} of \{
    True  -> \nterm{exp$_2$};
    False -> \nterm{exp$_3$};
\}
\EBS

% ----------------------------------------------------------------

\subsection{\te{let}}

Local definitions are introduced by the \te{let} expression.  The
definitions look like function definitions on the top level, but the
type signature is optional.

\gram{exp}{\term{let} \many{\nterm{letDefn}} \term{in} \nterm{exp}}
\index{let expressions|textbf}
\index{pattern binding|textbf}
\index{let@\te{let} (keyword)}
\index{in@\te{in} (keyword)}

\gram{letDefn}{ \nterm{localDefn} {\alt} \nterm{patternBind} } {\alt} \nterm{shortDefn}

\gram{localDefn}{ \opt{\nterm{varId} \term{::} \nterm{ctxType} \term{;}} \many{\nterm{clause}} }

\gram{patternBind}{ \nterm{pat} \term{=} \nterm{exp} }

\gram{shortDefn}{ \nterm{varId} \term{::} \nterm{ctxType} \term{=} \nterm{exp} }

Example:
\begin{verbatim}
 let   x2 = x * x
       y2 = y * y
 in
       x2 + x2 - y2
\end{verbatim}

A \te{let} definition can bind a single identifier, but it can also
bind one or more identifier through a pattern binding.  Pattern
bindings are only allowed for patterns that cannot fail.

Example:
\begin{verbatim}
 let (x, y) = foo z
 in  x + y
\end{verbatim}
which is equivalent to saying
\begin{verbatim}
 let x = (foo z).fst
     y = (foo z).snd
 in  x + y
\end{verbatim}

\begin{NOTE}
Currently only \te{struct} patterns are allowed in pattern bindings.
\end{NOTE}

% ----------------------------------------------------------------

\subsection{Structs and Tuples}

\label{sec-struct-val}

Section {\ref{sec-struct-type}} describes how to define a
{\te{struct}} type.  To produce a value of such a type, we write the
type name followed by values for the fields.

\gram{aexp}{\nterm{typeId} \term{\{} \many{\nterm{field}} \term{\}}}
\index{structs!construction|textbf}

\gram{field}{\nterm{fieldId} \term{=} \nterm{exp} \term{;}}

Example:
\begin{verbatim}
 Proc { pc = 0; cc = EQ }
\end{verbatim}

Section {\ref{sec-tuple-type}} describes the {\te{PrimPair}} struct
type.  This ``tuple type'' may be expressed using the special
syntactic shorthand involving parentheses and commas.  Similarly, this
notation can be used for value expressions as well, {\ie} the
expression on the left is a shorthand for for the expression on the
right:
\index{tuples!values|textbf}
\BBS
(a, b)  \hmm  $\equiv$  \hmm  PrimPair \{ fst = a; snd = b \}
\EBS

% ----------------------------------------------------------------

\subsection{Struct field selection}

\label{sec-struct-select}

A field of a struct value can be selected with dot notation.

\gram{exp}{\nterm{exp} \term{.} \nterm{fieldId}}
\index{structs!field selection|textbf}
\index{=dot@\te{.}|see{structs, field selection}}

Example
\begin{verbatim}
 r.pc
\end{verbatim}

\begin{NOTE}
Since the same field-name can occur in multiple types, the compiler
uses type information to resolve which field-name you mean when you do
a field selection.  Occasionally, you may need to add a type signature
to help the compiler resolve this.
\end{NOTE}

There is a shorthand for making a field selection into a function:

\gram{aexp}{\term{(} \term{.} \nterm{fieldId} \term{)}}

The expression \qbs{(.name)} is equivalent to \qbs{\BSL x -> x.name}.

% ----------------------------------------------------------------

\subsection{Struct ``update''}

\label{sec-struct-update}

A struct value can be constructed from another struct value by changing
some of the fields.

\gram{exp}{\nterm{aexp} \term{\{} \many{\nterm{field}} \term{\}}}
\index{structs!update|textbf}

Example
\begin{verbatim}
 s { x = 77; y = 88 }
\end{verbatim}

Here \qbs{s} is an expression of \te{struct} type.  The entire expression
has the same type as \qbs{s} and all fields expect \qbs{x} and \qbs{y}
also have the same values.

% ----------------------------------------------------------------

\subsection{\te{interface} expressions}

\label{sec-interface-expressions}

An interface expression defines a value of interface type.

\gram{exp} { \term{interface} \nterm{typeId} \term{\{} \many{\nterm{ifcDefn} \term{;}} \term{\}} }
\index{interface@\te{interface} (keyword)!in interface expressions}

\gram{ifcDefn}{ \nterm{localDefn} \opt{ \term{when} \nterm{guard} } }
\index{when@\te{when} (keyword)!in interface expression}

Example:
\begin{verbatim}
 interface Stack
     push x = r := Just x    when r == Nothing
     pop    = r := Nothing   when r /= Nothing
     top    = r._read
\end{verbatim}

The {\te{when}} clause in a method specifies the condition,
\index{implicit conditions|textbf}
\index{implicit conditions!on interface methods}
called the {\em implicit condition}, which must hold when this method
is called.  The compiler will make a scheduler that fulfills the
condition.  Using implicit conditions, it is possible to write client
code that is not cluttered with conditionals that test whether the
method is applicable.  For example, a client of a FIFO module can just
call the ``enqueue'' or the ``dequeue'' method without having
explicitly to test whether the FIFO is full or empty, respectively;
those predicates are usually specified in implicit conditions inside
the FIFO module interface definition itself.

The {\term{when}} clause can refer to any variables from the
surrounding scope.  In particular, note that the interface method
arguments are {\emph{not}} available in the {\term{when}} clause.

\begin{NOTE}
There are good implementation (hardware) reasons for not allowing the interface
arguments to be used in the implicit condition.
\end{NOTE}

% ----------------------------------------------------------------

\subsection{``Don't care'' expressions}

\label{sec-exprs-wildcard}

\index{=underscore@\te{\US}!``don't care'' expression|textbf}
When the value of an expression does not matter a ``don't care''
expression can be used.  It is written as an underscore and has any
type.  The compiler will pick some suitable value.

If a ``don't care'' value is part of any computation (such as an
argument to an addition function) the result will be a new ``don't
care'' value.

\gram{aexp}{\term{\US}}

Note that this is a distinct (but related) use of the underscore from
its use as a ``don't care'' pattern (section
{\ref{sec-patterns-var}}).

The programmer is encouraged to use ``don't care'' values where
possible, both because it is useful documentation and because the
compiler can often exploit this to produce better circuits.

% ----------------------------------------------------------------

\subsection{Actions}

\label{sec-actions}

Any expression which is intended to act on state is called an
{\emph{action}} and has type \te{Action}.
\index{actions!\te{Action} (type)|textbf}
Primitive actions are provided as fields of the interfaces to objects
provided by the compiler (such as registers or arrays).  The
programmer can create new actions only by building on these
primitives, or by using {\veri} modules.

Example:

\begin{verbatim}
 interface Reg a =
     _write :: a -> Action
     _read  :: a
\end{verbatim}

Actions are combined by the keyword \te{action} followed by a sequence of
actions.
\index{actions!combining}
\index{actions!\te{action} (keyword)|textbf}

\gram{exp}{\term{action} \term{\{} \many{\nterm{stmt} \term{;}} \term{\}}}

Example:
\begin{verbatim}
 action { x := x+1; y := z }
\end{verbatim}

The Standard Prelude defines the ``empty'' action:
\index{noAction@\te{noAction} (empty action)}
\begin{verbatim}
 noAction :: Action
\end{verbatim}
which is equivalent to the expression: \te{action \LBRACE\RBRACE}.

The \te{Action} type is actually a special case of the more general
type \te{ActionValue}, described in the next section:
\begin{verbatim}
 type Action = ActionValue ()
\end{verbatim}

% ----------------

\subsubsection{\te{ActionValue}}

\label{sec-actionValue}

The \te{ActionValue} is an abstract type:\index{ActionValue@\te{ActionValue} (type)}
\begin{verbatim}
 interface ActionValue a
 instance Monad ActionValue
\end{verbatim}
Values of \te{ActionValue} type should be thought of as performing an
action as well as returning a value.

The \te{ActionValue} type is a monad and the \te{action} syntax allows
variable bindings for the value as well as performing actions with no
return value.

Example:
\begin{verbatim}
 interface IntStack =
     push :: Int -> Action
     pop  :: ActionValue Int



 ...
     s1 :: IntStack
 ...
     s2 :: IntStack
 ...
     action
         x <- s1.pop        -- A
         s2.push (x+1)      -- B
\end{verbatim}

In line A, we perform a {\te{pop}} action on stack {\te{s1}}, and the
returned value is bound to {\te{x}}.  If we were not interested in the
returned value, we could have omitted the ``{\te{x <-}}'' part.  In
line B, we perform a {\te{push}} action on {\te{s2}}, and the returned
value {\te{()}} is discarded (not bound to anything).

% ----------------------------------------------------------------

\subsection{\te{rules}}

\label{sec-rule}

\index{rules!expression|textbf}
The {\te{rules}} expression introduces rewrite rules for the Term
Rewriting System.  It appears inside a module and specifies part of
the behavior of the module.  A {\te{rules}} expression has type
{\te{Rules}}\index{rules@\te{Rules} (type)|textbf} and consists of a
list of rewrite rules.

Each rewrite rule has a left hand side and a right hand side.  The
left hand side is a guard; the rule only applies if the guard is
valid.  The right hand side of a rule is an {\emph{action}}.
Commonly, it is a composite of many actions to be performed when the
rule is applied.  An entire rule may optionally be prefixed with a
{\emph{rule label}} which is useful primarily in debugging, {\ie} when
simulating/executing the hardware description produced by the {\BH}
compiler, the execution engine may be able to inform you about when a
particular rule fires using these rule labels.  A rule label is a
string valued expression.

% TODO: we should have some default rule identifier that identifies the
% source file and line of the rule's WHEN keyword

\gram{exp}{\term{rules} \term{\{} \many{\nterm{rule} \term{;}} \term{\}}}
\index{rules@\te{rules} (keyword)}

\gram{rule}{\opt{\nterm{ruleLabel} \term{:}} \term{when} \nterm{guard} \term{==>} \nterm{exp}}
\index{when@\te{when} (keyword)!in rules expressions}

\gram{ruleLabel}{\nterm{aexp}}

Example:
\begin{verbatim}
 let
   instr :: Word
   instr = mem[pc]
 in
   rules
     when Add r1 r2 r3 <- instr
     ==> action {
           pc := pc+1;
           rf[r1] := rf[r2] + rf[r3]
         }

     when Jz r1 r2 <- instr, rf[r1] == 0
     ==> action { pc := r2 }
\end{verbatim}

% ----------------

\subsubsection{Nested rule guards}

\index{rules!nested rule guards}
Sometimes in a series of rules, each rule may have a conjunction of
conditions, and all the rules share one of those conditions.  In such
a situation, it is useful to be able to factor out the common
condition, and this can be done by nesting {\te{when}} clauses.

\gram{rule}{\opt{\nterm{ruleLabel} \term{:}} \term{when} \nterm{guard} \term{rules} \term{\{} \many{\nterm{rule} \term{;}} \term{\}}}

Example,
\begin{verbatim}
 rules
     when c, c1 ==> action1
     when c, c2 ==> action2
     when c, c3 ==> action3
     when d     ==> action4
\end{verbatim}
can be written more clearly as:
\begin{verbatim}
 rules
     when c
     rules
         when  c1 ==> action1
         when  c2 ==> action2
         when  c3 ==> action3

     when d  ==> action4
\end{verbatim}

% ----------------

\subsubsection{Aggregating and prioritizing rules}

\index{rules!aggregating|textbf}
\index{rules!prioritizing}
Rules are first class objects.  The following operators allow rule
sets to be combined:

\begin{verbatim}
 (<+>) :: Rules -> Rules -> Rules
 (<+)  :: Rules -> Rules -> Rules
 (+>)  :: Rules -> Rules -> Rules
\end{verbatim}
\index{=ltplusgt@\te{<+>} (\te{Rules} aggregation operator)|textbf}
\index{=ltplus@\te{<+} (\te{Rules} aggregation operator)|textbf}
\index{=plusgt@\te{+>} (\te{Rules} aggregation operator)|textbf}

The {\qbs{<+>}} operator makes a symmetric union of two rule sets.  The
{\qbs{<+}} operator makes a directed union, {\ie} the rules on the
right may fire only when none of the rules on the left are enabled.
The {\qbs{+>}} operator makes a directed union in which rules on the
left may fire only when none of the rules on the right are enabled.

% ----------------------------------------------------------------

\subsection{Modules}

\label{sec-modules}

Modules are the heart of {\BH}.  Modules turn into actual hardware,
and correspond roughly to {\veri} modules.  State can exist only
inside a module.  Modules also incorporate the rules which act on
their state.

A module consists of three things: state, rules on that state, and an
interface to the outside world.  This information is given in a
{\te{module}} expression which has type \qbs{Module a},\footnote{
Actually, the type is more general, see \ref{sec-ismodule}.}  where
{\te{a}} is the type of the interface.
\index{modules!module@\te{module} (keyword)|textbf}
\index{modules!Module@\te{Module} (type)|textbf}

There is a strong analogy between {\BH} modules and \emph{objects}
\index{modules!analogy with objects in object-oriented programming languages}
in object-oriented programming languages, particularly objects that
represent processes.  A {\te{module}} expression of type {\qbs{Module
a}} defines an object constructor, {\ie} something that allocates and
initializes an object.  The constructor returns an object reference of
type {\qbs{a}}, {\ie} a handle on which you can call the interface
methods.  Each invocation of the constructor produces a new object,
and returns the handle to that new object, so it is easy to make
multiple copies of an object.  The state elements of the object
correspond to private variables inside the object.  They cannot
directly be manipulated or accessed by any other object; this can only
be done \emph{via} interface methods of the object.  The rules in an
object specify the internal, free-running behavior of the object,
{\ie} the ``process'' that the object represents.

Here is the grammar for {\te{module}} expressions:

\gram{exp}{ \term{module} \term{\{} \many{\nterm{mstmt} \term{;}} \term{\}} }

\gram{mstmt}{ \nterm{stmt} {\alt} \nterm{mrules} {\alt} \nterm{minterface} }

\gram{stmt}{ \nterm{varId} \term{::} \nterm{ctxType} \term{;} \nterm{varId} \term{<-} \nterm{exp}} \\
\gramalt{ \nterm{pat} \term{::} \nterm{ctxType} \term{<-} \nterm{exp}} \index{=ltminus@\te{<-} (in statements)}\\
\gramalt{ \term{let} \many{\nterm{letDefn}}}
\index{let@\te{let} (keyword)}

\gram{mrules}{\term{rules} \term{\{} \many{\nterm{rule} \term{;}} \term{\}}}

\gram{minterface}{\term{interface} \term{\{} \many{ \nterm{fieldDef} \term{;} } \term{\}} }\\
\gramalt{\term{interface} \term{(} \many{\nterm{exp} \term{,}} \term{)}}


The state-creation statements look like this:
\mbox{\nterm{x} \te{::} {\nterm{t}} \te{;}}
\mbox{\nterm{x} \te{<-} {\nterm{e}}}.
An equivalent way of writing this is
\mbox{\nterm{x} \te{::} {\nterm{t}} \te{<-} {\nterm{e}}}.
The first part is a type signature, as usual.  The second part is
called a {\emph{monadic}} binding.
The right hand side expression
{\nterm{e}} allocates some state, which is just another module,
either a module defined elsewhere or a primitive module like a
register, array, or FIFO.  The right-hand side {\nterm{e}} also
returns a value, which is bound to the left-hand side identifier
{\nterm{x}}.  Thus, the right-hand side {\nterm{e}} must have type
{\mbox{\te{Module $t$}}}, and {\nterm{x}} will be bound to a value
of type $t$, {\ie} to the interface of the module.

If you do not want to use the value returned by {\nterm{exp}}, you can
use {\nterm{exp}} as a statement by itself (no need for the
{\qbs{\nterm{varId} <-}} part).

Statements can also be \te{let} statements, which are used for
ordinary value bindings (the left-hand side identifier and the
right-hand side expression can be of any type).

\begin{NOTE}
The advanced user will recognize that module statements are similar
to the body of a \te{do}-statement (see section {\ref{sec-do}}).
Modules are monads.  Thus, an expression which is bound to a
variable of type {\te{a}} by means of the \te{<-} syntax must have
type \te{(Module a)}.  The module constructed by the expression
will have its own internal state and rules, which will be incorporated
into the rules and state of the containing module, although they
will be hidden behind an abstraction barrier.  Only the interface of
the module is accessible and it is that which is bound to the
variable.

While the body of a \te{do}-statement has a final expression which
provides the value for the whole expression, the \te{rules} and
{\te{interface}} section of a module form an implicit final
expression which adds the rules to the monad and returns the
interface.
\end{NOTE}

Example: A register is primitive module whose interfaces
is defined as follows:
\begin{verbatim}
 interface Reg a =
     set :: a -> Action
     get :: a
\end{verbatim}
and with the following module constructor function:
\begin{verbatim}
 -- takes an initial value for the register
 mkReg :: (Bits a sa) => a -> Module (Reg a)
\end{verbatim}

A module built on these primitives would look like:
\begin{verbatim}
 interface ArithIO a =
     input :: a -> a -> Action
     output :: a

 mkGCD :: Module (ArithIO (Bit 32))
 mkGCD = module
              x :: Reg (Bit 32)
              x <- mkReg _

              y :: Reg (Bit 32)
              y <- mkReg _

              done :: Reg Bool
              done <- mkReg True
              interface
                  input a b = action { x._write a; y._write b; done._write False }
                       when done._read
                  output = x._read
                       when done._read
              rules
                  when not done._read, x._read > y._read, y._read /= 0
                    ==> action { x._write y; y._write x }

                  when not done._read, y._read == 0
                    ==> action { done._write True }

                  when not done._read, x._read <= y._read, y._read /= 0
                    ==> action { y._write (y._read - x._read) }
\end{verbatim}

Note how the two methods in the interface can only be applied when
the computation is done, ensuring that the result is not read too early
and that new arguments do not overwrite an ongoing computation.

Because registers are the most common state elements, a special
notation is available to relieve the programmer from having to type
{\te{.\_write}} and {\te{.\_read}} everywhere.  This is described in Section
{\ref{sec-registers}}.

%\subsubsection{Desugaring}
%The module syntax is only shorthand for a \te{do} expression.
%The following translation is used to expand it.
%
%\BBS
%module $s$ rules $r$ interface $i$  \hmm  $\equiv$  \hmm do $s$; addModuleRules $r$ $i$
%\EBS
%The \te{addModuleRules} function (defined in the \te{Prelude}) is the operation in the
%\te{Module} monad that adds rules to the system.

%\subsubsection{Dumping variables}
%It is possible to dump arbitrary expressions in the VCD dump from
%the C simulation by inserting dump pragmas in the module definition.
%
%\gram{statement}{\term{\{-\#} \term{dump} \many{\nterm{exp} \term{,}} \term{\#-\}}}
%
%Example:
%\begin{verbatim}
% module
%     x :: Reg (Int 32) <-mkRegU
%     let y = x+2
%         z = x+y
%     {-# dump y, z, x>10 #-}
% interface
%     ...
%\end{verbatim}


% ================================================================

\section{Patterns}

\label{sec-patterns}

\index{patterns|textbf}
Patterns are used in value definitions (section {\ref{sec-val-defs}}),
in {\te{case}} expressions (section {\ref{sec-case}}), in
$\lambda$-expressions (section {\ref{sec-lambda}}) and in guards
(section {\ref{sec-guards}}).  A pattern is always {\emph{matched}}
against an actual value and, in the process, it plays two roles:
\begin{tightlist}
\item
First, a pattern acts as a Boolean filter--- it succeeds only if the
actual value against which it is matched has the same form as the
pattern, {\ie} (a) the value is built out of the same constructor, and
(b), the corresponding components of the constructor in the pattern and
actual value also match.
\index{pattern matching|textbf}

\item
Second, assuming the pattern does match, then the variables in the
pattern are bound to the corresponding components in the actual value.
\end{tightlist}
Thus, a pattern is used both as a predicate (``does it match?'') and as
a binding mechanism to name components of an actual value.

The variables used in a pattern may not be repeated, {\ie} any given
variable can occur at most once in a pattern.

% ----------------------------------------------------------------

\subsection{Variable and wild-card patterns}

\label{sec-patterns-var}

\gram{pat}{\nterm{varId} {\alt} \term{\US}}

\index{patterns!variable|textbf}
\index{patterns!_@\te{\US} (underscore, don't care)|textbf}
\index{=underscore@\te{\US}!``don't care'' pattern|textbf}
A variable or a wild-card (bare underscore) is a trivial pattern that
matches any actual value.  If it is a variable, it also binds the
variable name to that value.

(Of course, when we say {\emph{any actual value}} here, we mean any
value that could possibly be supplied for matching here.  Static type
checking will ensure that the only actual values supplied here will
have the correct type.)

Note that this use of an underscore is distinct (but related to) its
use as a ``don't care'' expression (section
{\ref{sec-exprs-wildcard}}).

% ----------------------------------------------------------------

\subsection{Constructor and constant patterns}

\label{sec-patterns-constrs}

\index{patterns!constructor|textbf}

\gram{pat}{\nterm{conId} \many{\nterm{apat}}}

In a constructor pattern, there must always be as many {\nterm{apat}}
arguments as the number of arguments in the constructor
{\nterm{conId}}'s original declaration.

Such a pattern matches an actual value that is constructed out of the
same constructor, and where (recursively) each {\nterm{apat}} argument
matches its corresponding component in the actual value.  The variable
bindings produced by the match is the union of the variable bindings
of the individual {\nterm{apat}} matches.

% ----------------------------------------------------------------

\subsection{Struct and Tuple patterns}

\label{sec-patterns-struct}

\index{patterns!struct|textbf}

Struct patterns are used to match struct values.  A struct pattern has
a number of field patterns.  Not all fields in a struct need be
present in the field patterns (if a field patterns is missing, the
corresponding component in the actual value is is ignored).  A field
pattern can be abbreviated, or {\emph{punned}}, if the bound variable
has the same name as the field.

\gram{apat}{\nterm{typeId} \term{\{} \many{\nterm{fieldPat} \term{;}} \term{\}}} \\
\gram{fieldPat}{ \nterm{fieldId} \term{=} \nterm{pat} {\alt} \nterm{fieldId} }

Example (see section {\ref{sec-struct-type}} for corresponding type definition):
\begin{verbatim}
 Proc { pc = pc; rf = regfile }
\end{verbatim}
This matches any value constructed using the {\te{Proc}} constructor,
and binds the identifiers {\te{pc}} and {\te{regfile}} to the {\te{pc}}
and {\te{rf}} field values, respectively.  The {\te{mem}} field is
ignored.

Note that in the phrase {\mbox{\te{pc = pc}}}, the left-hand
occurrence of {\te{pc}} is the fieldname, whereas the right-hand
occurrence is a variable that happens to be spelt the same.  Using the
``punning'' abbreviation described above, we could also write this as:
\begin{verbatim}
 Proc { pc; rf = regfile }
\end{verbatim}

Section {\ref{sec-tuple-type}} describes the {\te{PrimPair}} struct
type, and section {\ref{sec-struct-val}} describes corresponding value
expressions.  In both cases, these ``tuple'' types and objects may be
expressed using the special syntactic shorthand involving parentheses
and commas.  Similarly, this notation can be used in pattern matching
as well, {\ie}
\index{tuples!patterns|textbf}
the pattern on the left is a syntactic shorthand for the pattern on
the right:
\BBS
(a, b)  \hmm  $\equiv$  \hmm  PrimPair \{ fst = a; snd = b \}
\EBS


% ================================================================

\section{Guards}

\label{sec-guards}

\index{guards!with pattern matching|textbf}
Guards are used as extra conditions in a pattern match to limit when a
certain case should be used.  They are used in value definitions
(section {\ref{sec-val-defs}}), in {\te{case}} expressions (section
{\ref{sec-case}}), and in rules (section {\ref{sec-rule}}).

A pattern together with a guard is considered to match only if both
the pattern and the guard match.  A guard consists of a list of zero
or more parts.  A guard matches if all its parts match.  The parts are
tested from left to right.  Identifiers bound in one part may be used
in subsequent parts to its right.

\gram{guard}{\many{\nterm{qual} \term{,}}} \\
\gram{qual}{\nterm{exp} {\alt} {\nterm{pat} \term{<-} \nterm{exp}}}
\index{=ltminus@\te{<-} (in guards)}

A Boolean guard {\nterm{exp}} is an arbitrary expression.  It is
considered to match if the expression evaluates to \te{True}.

A pattern guard {\mbox{\nterm{pat} \term{<-} \nterm{exp}}} has a
pattern and an expression.  It is considered to match if the pattern
matches the value of the expression.  The variables in the pattern get
bound to the corresponding components.

% ================================================================

\section{Important Primitives}

\label{sec-important-prims}

These primitives are available \emph{via} the standard prelude and other
standard libraries.  See also Appendix {\ref{sec-additional-libs}} for
more useful libraries.

% ----------------------------------------------------------------

\subsection{The ``size'' types}

\label{sec-primitives-size-types}

\index{size types|textbf}
As described in section {\ref{sec-type}}, there is a collection of
types representing ``sizes'' that are written as numbers.  Typically,
the only place these types are/can be used are as arguments to other
parameterized types.  For example, the type:
\begin{verbatim}
 Bit 16
\end{verbatim}
consists of the unary type constructor {\te{Bit}} applied to the type
{\qbs{16}}.  The type as a whole represents bit vectors of length 16
bits.

Collections of size types are also instances of certain predefined
classes that can be used to express size constraints:
\index{size types!type classes for constraints}
\index{Add@\te{Add} (type class, of size types)}
\index{Max@\te{Max} (type class, of size types)}
\index{Log@\te{Log} (type class, of size types)}
\begin{verbatim}
 class Add x y z | x y -> z, y z -> x, z x -> y

 class Max x y z | x y -> z

 class Log x y | x -> y, y -> x
\end{verbatim}

The \te{Add} class has instances for all size types $x$, $y$, and $z$
such that $x+y=z$.  The \te{Max} class has instances for all size
types $x$, $y$, and $z$ such that $max(x,y)=z$. The \te{Log} class has
instances for all size types $x$ and $y$ such that $ceil (log_2 x) =
y$.  These functional dependencies enable the type checker to do some
limited forms of arithmetic.

Example:
\begin{verbatim}
 pad0101 :: (Add n 4 m) => Bit n -> Bit m
 pad0101 x = x ++ 0b0101
\end{verbatim}
The second line defines the function {\te{pad0101}} as taking a bit
vector and padding it to the right with the bits ``0101'' using the
bit-concatenation operator {\qbs{++}}.  The type signature on the
first line expresses the idea that the function takes a bit vector of
length $n$ and returns a bit vector of length $m$, where $n+4=m$.

To get the value that corresponds to a size there is a special
``function'', {\te{valueOf}},
{\index{valueOf@\te{valueOf} (``function'' of ``size'' types)}}
that takes a size type and gives the
corresponding {\te{Integer}} value.

\begin{verbatim}
 type Five = 5
 x :: Integer
 x = valueOf Five  -- x will have the value 5
\end{verbatim}
In the first line, the symbol ``5'' represents the size type ``5'',
not the integer value 5.  The type synonym is there just for
readability.  In the last line, {\te{x}} gets the corresponding
integer value of 5.

In a pinch, this mechanism can be used to do arithmetic for you!

\begin{verbatim}
 type WordSize = 32

 logW :: (Log WordSize k) => Integer
 logW = valueOf k
\end{verbatim}
The type synonym is there just for readability.  The type signature
says that {\te{logW}} has an integer value, provided that
{\te{WordSize}} and {\te{k}} are instances of the {\te{Log}} class,
{\ie} provided the values corresponding to {\te{WordSize}} and
{\te{k}} are in the logarithm relation.  Then, the last line binds
{\te{logW}} to the integer value corresponding to {\te{k}}

\begin{verbatim}
 logWPlusOne :: (Log WordSize m, Add m 1 n) => Integer
 logWPlusOne = valueOf n
\end{verbatim}
The type signature establishes that (the integers corresponding to)
{\te{WordSize}} and {\te{m}} are in the log relation; that (the
integers corresponding to) {\te{m}} and {\te{1}} and {\te{n}} are in
the addition relation, and that {\te{logWPlusOne}} has Integer type.
The second line binds {\te{logWPlusOne}} to the integer value
corresponding to {\te{n}}.

{\index{stringOf@\te{stringOf} (``function'' of string types)}}
A ``function'' {\te{stringOf}} is also available to get the string value
of a type-level string.
\begin{verbatim}
 type MyString = "Hello, world!"
 x :: String
 x = stringOf MyString  -- x will have the value "Hello, world!"
\end{verbatim}

% ----------------------------------------------------------------

\subsection{The type \te{Bit}}

\index{Bit@\te{Bit} (type)|textbf}
A very important built-in unary type constructor is {\qbs{Bit}}.  It
represents bit vectors of a certain size.

Example:
\begin{verbatim}
 zero :: Bit 16
 zero = 0

 type BurroughsWord = Bit 51
\end{verbatim}

To extract a sub-vector from a bit-vector there is a special notation
taken from {\veri}.

\gram{exp}{\nterm{exp}\term{[}\nterm{exp}\term{:}\nterm{exp}\term{]}}

The expression \te{e[h:l]} extracts bits from \te{l} (low index) to \te{h}
(high index) inclusively.

\begin{NOTE}
The type system is not powerful enough to express the exact type of
bit extraction, so the extracted bit field can be used as a bit vector
of any width.  To adjust it to the right size, it is either truncated
from the left or extended with zeros to the left, as necessary (most
significant bit side).
\end{NOTE}

To concatenate bit vectors the \te{++} operator can be used.  The type
of this operator expresses its type exactly.

\index{=plusplus@\te{++} (\te{Bit} concatenation operator)|textbf}
\begin{verbatim}
 (++) :: (Add m n mn) => Bit m -> Bit n -> Bit mn
\end{verbatim}

\index{split@\te{split} (\te{Bit} function)|textbf}
There is also a function to split bit fields
\begin{verbatim}
 split :: (Add m n mn) => Bit mn -> (Bit m, Bit n)
\end{verbatim}


% ----------------------------------------------------------------

\subsection{The \te{Bits} class}

\index{Bits@\te{Bits} (type class)|textbf}
The type class \te{Bits} contains the types that are convertible to
bit strings of a certain size.  For a type to be an instance of this
class is a prerequisite for a number of things, such as putting it in
a register, array, or fifo.

\begin{verbatim}
 class Bits a n | a -> n where
     pack   :: a -> Bit n
     unpack :: Bit n -> a
\end{verbatim}

Here, ``\te{a}'' represents the type that can be converted to/from
bits, and ``\te{n}'' is always instantiated by a size type
representing the number of bits needed.

The most trivial instance declaration is that a bit vector can be
converted to a bit vector:
\index{pack@\te{pack} (\te{Bits} class method)|textbf}
\index{unpack@\te{unpack} (\te{Bits} class method)|textbf}
\begin{verbatim}
 instance Bits (Bit k) k where
     pack x = x
     unpack x = x
\end{verbatim}

Another example:
\begin{verbatim}
 data Color = Red | Green | Blue
 instance Bits Color 2 where
     pack Red    = 0b00
     pack Green  = 0b01
     pack Blue   = 0b10
     unpack 0b00 = Red
     unpack 0b01 = Green
     unpack 0b10 = Blue
\end{verbatim}

Instances of the {\te{Bits}} class can be derived by the compiler by
using the {\te{deriving}} directive.  Example:
\begin{verbatim}
 struct Coord = { x :: Int; y :: Int } deriving(Bits)
\end{verbatim}
This defines a new struct type {\te{Coord}} with two {\te{Int}}
fields.  The {\te{deriving}} clause registers {\te{Coord}} as an
instance of the {\te{Bits}} class and automatically produces the
required class methods {\te{pack}} and {\te{unpack}} to convert from
{\te{Coord}}'s to bit vectors and vice versa (the mapping algorithm
is described in more detail in Section~\ref{sec-deriving-Bits}).

There is a type ``function,'' {\te{SizeOf}}, that can be applied to a
type to get its corresponding bit size.
\index{SizeOf@\te{SizeOf} (pseudo-function on types)|textbf}

% ----------------------------------------------------------------

\subsection{UInt, Int}

\index{Int@\te{Int} (type)|textbf}
\index{UInt@\te{UInt} (type)|textbf}
{\mbox{\te{UInt $n$}}} and {\mbox{\te{Int $n$}}} define an unsigned
and a signed integer data type, respectively, of $n$ bits.

These types are instances of the classes {\te{Bits}}, {\te{Literal}},
{\te{Eq}}, {\te{Arith}}, {\te{Ord}}, {\te{Bounded}}, and
{\te{Bitwise}} (see Appendix {\ref{sec-prelude}} for the operations
that come with these classes).

\note{The \te{UInt} and \te{Int} types are not really primitive;
they are defined completely in {\BH}.}

% ----------------------------------------------------------------

\subsection{Registers}

\label{sec-registers}

\index{Reg@\te{Reg} (type)|textbf}
The most elementary form of state available in {\BH} is the register.
Registers can be created with the {\veri} function \te{mkReg} which
is a module with interface \te{Reg}.  The argument to
\te{mkReg} is the initial value of the register.  A function \te{mkRegU}
exists which creates a register whose initial value we don't care about.

\index{set@\te{set} (\te{Reg} interface method)|textbf}
\index{get@\te{get} (\te{Reg} interface method)|textbf}
\index{mkReg@\te{mkReg} (\te{Reg} function)|textbf}
\index{mkRegU@\te{mkRegU} (\te{Reg} function)|textbf}
\begin{verbatim}
 interface Reg a =
     _write :: a -> Action
     _read  :: a

 mkReg :: (Bits a sa) => a -> Module (Reg a)

 mkRegU :: (Bits a sa) => Module (Reg a)
\end{verbatim}

With this interface, it is necessary to use the functions {\te{\_read}}
and {\te{\_write}} to retrieve and set values in a register.  To save the
programmer some keystrokes and to improve readability of programs,
mechanisms have been introduced to allow the functions to be dropped.
First, the {\te{.\_read}} can be dropped from most variable names and the
compiler will add it implicitly if it is needed.  The compiler will
not be able to add {\te{.\_read}} to expressions, only to identifiers.
In some cases, the compiler might accidentally insert {\te{.\_read}}
where the programmer really intended to refer to the register and not
its contents.  If this happens, simply apply the function {\te{asReg}}
to the variable name, thereby turning it into an expression, and the
compiler will not insert the {\te{.\_read}}.

\index{asReg@\te{asReg} (dummy \te{Reg} function)|textbf}
\begin{verbatim}
 asReg :: Reg a -> Reg a
 asReg r = r
\end{verbatim}

The symbol \te{:=} can be used as syntactic shorthand for setting a
register:

\gram{exp}{ \nterm{exp} \term{:=} \nterm{exp} }
\index{=coloneq@\te{:=} (\te{Reg} assignment)|textbf}

Example:
\BBS
pc := pc + 1  \hmm  {\rm\emph{is shorthand for}}  \hmm pc.\_write (pc.\_read + 1)
\EBS

% ----------------------------------------------------------------

\subsection{FIFOs}

Package {\te{FIFO}} defines several useful interfaces and modules for
FIFOs.

\index{FIFO@\te{FIFO} (interface type)|textbf}
\index{enq@\te{enq} (\te{FIFO} interface method)|textbf}
\index{deq@\te{deq} (\te{FIFO} interface method)|textbf}
\index{first@\te{first} (\te{FIFO} interface method)|textbf}
\index{clear@\te{clear} (\te{FIFO} interface method)|textbf}
\index{mkFIFO@\te{mkFIFO} (\te{FIFO} function)|textbf}
\index{mkSizedFIFO@\te{mkSizedFIFO} (\te{FIFO} function)|textbf}


\begin{verbatim}
 interface FIFO a =
     enq      :: a -> Action
     deq      :: Action
     first    :: a
     clear    :: Action

 -- Make a FIFO
 mkFIFO :: (Bits a as) => Module (FIFO a)
 mkSizedFIFO :: (Bits a as) => Integer -> Module (FIFO a)
\end{verbatim}

The constructor {\te{mkFIFO}} leaves the capacity of the FIFO
unspecified (the number of entries in the FIFO before it becomes
full).  The constructor {\te{mkSizedFIFO}} takes the desired capacity
of the FIFO as an argument.

% ----------------------------------------------------------------

\subsection{FIFOFs}

Package {\te{FIFOF}} defines several useful interfaces and modules for
FIFOs.  The \te{FIFOF} interface is like a \te{FIFO}, but it also has methods
to test if the FIFO is full or empty.

\index{FIFOF@\te{FIFOF} (interface type)|textbf}
\index{enq@\te{enq} (\te{FIFOF} interface method)|textbf}
\index{deq@\te{deq} (\te{FIFOF} interface method)|textbf}
\index{first@\te{first} (\te{FIFOF} interface method)|textbf}
\index{clear@\te{clear} (\te{FIFOF} interface method)|textbf}
\index{mkFIFOF@\te{mkFIFOF} (\te{FIFOF} function)|textbf}
\index{mkSizedFIFOF@\te{mkSizedFIFOF} (\te{FIFOF} function)|textbf}


\begin{verbatim}
 interface FIFOF a =
     enq      :: a -> Action
     deq      :: Action
     first    :: a
     clear    :: Action
     notFull  :: Bool
     notEmpty :: Bool

 -- Make a FIFOF
 mkFIFOF :: (Bits a as) => Module (FIFOF a)
 mkSizedFIFOF :: (Bits a as) => Integer -> Module (FIFOF a)
\end{verbatim}

The constructor {\te{mkFIFOF}} leaves the capacity of the FIFO
unspecified (the number of entries in the FIFO before it becomes
full).  The constructor {\te{mkSizedFIFOF}} takes the desired capacity
of the FIFO as an argument.

% ================================================================

\section{Interfacing to {\veri}}
{\BH} programs can include components that are written in {\veri},
and {\BH} also generates {\veri}, so there are two (related)
mechanisms to consider.

% ----------------------------------------------------------------

\subsection{{\veri} modules}

Modules written in {\veri} are an important part of {\BH} since this
is where state elements are defined.  A {\veri} module definition
specifies the naming of all the signals that are needed to implement
the interface methods as well as some standard signals that all
modules have.  {\veri} modules have the following syntax.

\gram{exp}{\term{module} \term{verilog} \nterm{vModName} \opt{\nterm{vParams}} \nterm{clkNames} \nterm{rstNames} \opt{\nterm{vArgs}}} \\
\grammore{\term{\{} \many{\nterm{fieldId} \opt{\nterm{mult}} \term{=} \nterm{portSpec} \many{\nterm{portSpec}} \term{;} } \term{\}}} \\
\grammore{\opt{\nterm{schInfo}}}

\gram{vModName}{\nterm{aexp}} \\
\gram{clkNames}{\many{\nterm{portName} \term{,}}} \\
\gram{rstNames}{\many{\nterm{portName} \term{,}}} \\
\gram{portSpec}{\nterm{portName}\opt{\many{\nterm{portProp} \term{,}}}} \\
\gram{portName}{\nterm{string}} \\
\gram{portProp}{\term{reg} \alt \term{const} \alt \term{unused} \alt \term{inhigh}} \\
\gram{vParams}{\term{(} \many{\nterm{exp} \term{,}} \term{)} \term{,}} \\
\gram{vArgs}{\term{(} \many{\term{(} \nterm{portName} \term{,} \nterm{exp} \term{)}} \term{)}} \\
\gram{mult}{\term{[} \nterm{int} \term{]}} \\

\gram{schInfo}{\term{[} \many{ \nterm{schMeths} \nterm{schOp} \nterm{schMeths} \term{,}} \term{]}} \\
\gram{schMeths}{\nterm{fieldId}} \\
\gramalt{\term{[} \many{\nterm{fieldId} \term{,}} \term{]}} \\
\gram{schOp}{\term{<>} \alt \term{<} \alt \term{<{}<}}

A {\veri} module has many parts because a lot of information needs to
be conveyed to the {\BH} compiler.  Many of the parts are optional,
so most definitions look less formidable than the grammar suggests.

A basic {\veri} module definition gives the name of the {\veri} module
({\nterm{vModName}}), the name of the clock signal ({\nterm{clkName}})
and then a number of definitions of the methods of the interface.

Example:
\begin{verbatim}
 interface Counter =
     up     :: PrimAction
     preset :: Bit 4 -> PrimAction
     value  :: Bit 4

 vCount :: Module Counter
 vCount = module verilog "count4" "clk" {
     up     = "enable";
     preset = "inp" "set";
     value  = "outp";
 }
\end{verbatim}
The name of the {\veri} module is \te{count4} and it is clocked by the
port \te{clk}.  It has three input ports: \te{enable}, \te{inp}, and \te{set},
and one output port: \te{outp}.

{\bf Beware}, the compiler has no way of checking that the definition of
a {\veri} module really corresponds to what the {\veri} code actually does
so it will just believe you.

The names of the {\veri} ports (the quoted names) do not have to be unique
in a {\veri} module description.  If the same port name is used more than
once the compiler will assume that the methods in which the names occur
share a port and it will insert a multiplexer accordingly.

Following a port name there can be port a property,
which is one of the following:
\begindescrlist{xxxxxxx}
\litem{\te{reg}}
specifies that the port is directly connected
to the input or output of a register.  This property is informational only
and propagated by the compiler to the generated top level module.
\litem{\te{const}}
{\em not used at the moment}
\litem{\te{ununsed}}
{\em not used at the moment}
\litem{\te{inhigh}}
specifies that this enable signal (which is the only place where it is allowed) is always
high, i.e., the method executes on every clock cycle.  There will be no {\veri} port corresponding
to this enable signal.
\end{list}

%\begin{NOTE}
%This does not schedule correctly at the moment!
%\end{NOTE}

Example:
\begin{verbatim}
 interface VSyncSRAM adrs dtas =
     exec  :: Bit adrs -> Bit dtas -> Bit 1 -> Bit 1 -> PrimAction
     rdata :: Bit dtas

 mkSPSRAM_V :: Integer -> Module (VSyncSRAM adrs dtas)
 mkSPSRAM_V nwords = do
     module verilog name "CLK" {
         exec  = "ADR" "DI" "WE" "EN" "?"{inhigh};
         rdata = "DO";
     } [ [exec,rdata] <> [exec,rdata] ]
\end{verbatim}
Since there will be no wire for the \term{exec} enable signal the name
does not matter.

% ----------------

\subsubsection{Method definitions}

\label{sec-veri-templ}

Each method definition consists of a number of port names.  There must be one name
for each part of the type of the interface.  A type of a method defined in a {\veri}
interface must be of the form
\BBS
$t_1$ -> $t_2$ $\cdots$ -> $t_n$
\EBS
where each of the $t_i$ must be of type \te{Bit $n$}, except for the final
type, $t_n$, which can also be of type \te{PrimAction}.
If the last type is \te{PrimAction} all the ports will be input ports and the last
port is the enable signal for the action.
If the last type is not \te{PrimAction} the last port will be an output port and the
others will be inputs.

% ----------------

\subsubsection{Parameters and arguments}

\label{sec-param-and-args}

A {\veri} module may require parameters to be instantiated.  Parameters must
be compile time constants and to ensure this they must be of type \te{Integer}
in {\BH}.  The parameters are given right after module name.
\BBS
module verilog "foo" (2,16) { $\cdots$ }
\EBS

Additional arguments (ports) can be given values by the argument part of the definition.
The arguments are given by specifying a {\veri} port name and the value that should
be on this port.  A typical use for this feature is to provide an initial value
for a state element that is to be used when the reset signal is asserted.
A expression used as an argument to a {\veri} module cannot have an implicit
\index{implicit conditions!disallowed in arguments to {\veri} modules}
condition (the compiler checks this).

Example, (part of) the definition of the \te{mkReg} module:
\begin{verbatim}
 interface VReg n =
     set :: Bit n -> PrimAction
     get :: Bit n

 vMkReg :: Bit n -> Module (VReg n)
 vMkReg v =
     module verilog "RegN" (valueOf n) "CLK" "RST" (("init", v)) {
         get = "get"{reg};
         set = "val"{reg} "SET";
     } [ get <> get, get < set, set < set ]
\end{verbatim}
The {\veri} module is named \te{RegN} and is given one parameter, namely
the size of the register to create.  In addition it is passed an additional
value \te{v} on the \te{init} port which is used to set the initial value
when the reset signal (\te{RST}) is asserted.

The definition of \te{mkReg} is completed by wrapping the \te{vMkReg}
module in some packing and unpacking.
\begin{verbatim}
 mkReg :: (Bits a sa) => a -> Module (Reg a)
 mkReg v =
     module
         r :: VReg sa
         r <- vMkReg (pack v)
         interface
             _read    = unpack r.get
             _write x = fromPrimAction (r.set (pack x))
\end{verbatim}

The following {\veri} code is one possible implementation of \te{RegN}:
\begin{verbatim}
 module RegN(CLK, RSTN, init, get, val, SET);
   parameter width = 1;
   input CLK;
   input RSTN;
   input [width - 1 : 0] init;
   input SET;
   input [width - 1 : 0] val;
   output [width - 1 : 0] get;
   reg [width - 1 : 0] get;
   always@(posedge CLK or negedge RSTN) begin
     if (!RSTN)
       get <= init;
     else if (SET)
       get <= val;
   end
 endmodule
\end{verbatim}

\subsubsection{Scheduling information}
The scheduling information is used to describe what operations can
be performed at the same time.  Currently, three relations can be described:
Conflict Free (\te{<>}), Sequentially Composable (\te{<}), and
Restricted Sequentially Composable (\verb'<<') (which means sequentially
composable, but not parallelly composable).
These relations are simply given by enumerating the elements of the
set (of method name pairs) that make up the relation.
A shorthand is provided for generating sets where the left or right
component is the same.

In the absence of scheduling information both relations are considered to be empty,
which is always a safe approximation.

% ----------------

\subsubsection{Multiple methods}

\label{sec-multiple}

For some {\veri} modules several ports with identical operation may
be available.  An example is a multiported memory where there are
several read ports available that can be used simultaneously.
This can, of course, be described by an interface that has several
similar methods and the use of these can then be determined by the
{\BH} code.  But the {\veri} module definitions also offers a
more convenient alternative; a port multiplicity can be specified.
This is done by a \qbs{[n]} following the field name.  This informs
the compiler how many similar ports are available and the compiler
will make sure to use them appropriately.

Example:
\begin{verbatim}
 interface SRAM =
     rd :: Addr -> Data
     wr :: Addr -> Data -> PrimAction

 mkSRAM :: Module SRAM
 mkSRAM = module verilog "SRAM" "CLK" {
     rd[3] = "raddr" "rdata";
     wr    = "waddr" "wdata" "we";
 }
\end{verbatim}
This specifies that there are 3 read ports.  The names of the port
wires are \te{raddr\_1/rdata\_1}, \te{raddr\_2/rdata\_2}, and \te{raddr\_3/rdata\_3}.

\tbd{The naming of the multiple ports may not be the best.}

% ----------------
\com{
\subsubsection{Example: FIFO}

This example shows most of the features described above.  It is taken from
the definition of FIFOs as available in {\BH}.

Note how the \te{mkFIFO} function invokes two different implementations
of FIFOs depending on the size of the stored items (stored items
of size zero need no storage).

\begin{verbatim}
 interface FIFO a =
     enq      :: a -> Action
     deq      :: Action
     first    :: a
     clear    :: Action

 interface FIFO_ n =
     enq_      :: Bit n -> PrimAction
     deq_      :: Action
     first_    :: Bit n
     notFull_  :: Bit 1
     notEmpty_ :: Bit 1
     clear_    :: Action


 -- Note, n>0 must hold.
 vMkFIFO :: Module (FIFO_ n)
 vMkFIFO =
   module verilog "FIFON" (valueOf n) "CLK" {
     enq_      = "idata" "ENQ";
     deq_      = "DEQ";
     first_    = "odata";
     notFull_  = "CANENQ";
     notEmpty_ = "CANDEQ";
     clear_    = "CLR";
   }   [ enq_ <> [deq_, first_, notFull_, notEmpty_],
         deq_ <> [enq_, first_, notFull_, notEmpty_]
       ]

 mkFIFO :: (Bits a as) => Module (FIFO a)
 mkFIFO =
   -- Use a counter if the enqueued item has size 0
   if valueOf as == 0 then
     module
       _n :: Reg (Bit 1)
       _n <- mkReg 0
       interface
         enq _ = _n := _n + 1                  when _n < maxBound
         deq   = _n := _n - 1                  when _n > 0
         first = unpack 0                      when _n > 0
         clear = _n := 0
   else
     -- Use a Verilog FIFO for ordinary items
     module
       _f :: FIFO_ as
       _f <- vMkFIFO
       interface
         enq x    = _f.enq_ (pack x)           when (unpack _f.notFull_)
         deq      = _f.deq_                    when (unpack _f.notEmpty_)
         first    = unpack _f.first_           when (unpack _f.notEmpty_)
         clear    = _f.clear_
\end{verbatim}

Here is some {\veri} code that is one possible implementation of the definition:
\begin{verbatim}
 module FIFON(CLK, idata, ENQ, CANENQ, odata, DEQ, CANDEQ, CLR);
   parameter width = 1;
   input CLK;
   input [width - 1 : 0] idata;
   input ENQ;
   output CANENQ;
   output [width - 1 : 0] odata;
   reg [width - 1 : 0] odata;
   input DEQ;
   output CANDEQ;
   reg CANDEQ;
   input CLR;
   assign CANENQ = !CANDEQ;
   always@(posedge CLK)
   if (CLR)
     CANDEQ <= 0;
   else
     if (ENQ)
       begin
         CANDEQ <= 1;
         odata <= idata;
       end
     else
       if (DEQ)
         CANDEQ <= 0;
 endmodule
\end{verbatim}
}

% ----------------------------------------------------------------

\subsection{Generated {\veri}}

The {\BH} compiler can generate {\veri} code (a module) for a
{\BH} module definition.  The type of the interface for the module
has to obey certain restrictions so that it can be converted to wires.

\tbd{Accurately describe restrictions.}

The interface type (of the designated module) will be ``mangled'' by
the {\BH} compiler to generate an interface that obeys the
restriction that {\veri} modules have, see section
{\ref{sec-veri-templ}}.  The definition of this interface type is
available in the generated signature (\qbs{.bi}) file for
informational purposes.  The ``mangled'' interface will contain one
extra method for each of the original methods (beginning in \te{RDY\_})
which is a handshake signal indicating that the method is ready to be
used.

%If the module for which {\veri} is being generated is named $name$ there
%will be an additional exported definition from the module named \te{v\_$name$}.
%When this definition is used it will use the {\veri} code for the given
%module instead of the {\BH} code.

Example:
\begin{verbatim}
 package Cube(mkCube, mkCube16, Cube) where
 import UInt
 import Mult

 interface Cube n =
     start  :: UInt n -> Action    -- An input causing an action
     result :: UInt n              -- The output

 data State = Idle | Working deriving (Eq, Bits)

 mkCube :: Module (Cube n)
 mkCube =
     module
         state :: Reg State
         state <- mkReg Idle
         x :: Reg (UInt n)
         x <- mkRegU
         r :: Reg (UInt n)
         r <- mkRegU
         m :: Mult n
         m <- mkMult
         let (*) = m.mul
         rules
             when state == Working
              ==> action { r := r * x; state := Idle }
         interface
             start n = action { x := n; r := x*x; state := Working }
                 when state == Idle
             result  = r
                 when state == Idle

 mkCube16 :: Module (Cube 16)
 mkCube16 = mkCube
\end{verbatim}

If code generation for \te{mkCube16} is requested the generated signature file
will contain this:
\begin{verbatim}
 signature Cube where {
 type (Cube.Cube :: # -> *) n;

 Cube.mkCube :: Prelude.Module (Cube.Cube n);

 Cube.mkCube16 :: Prelude.Module (Cube.Cube 16);

 interface (Cube.Cube_16_ :: *) = {
     Cube.start :: Prelude.Bit 16 -> Prelude.Action;
     Cube.RDY_start :: Prelude.Bit 1;
     Cube.result :: Prelude.Bit 16;
     Cube.RDY_result :: Prelude.Bit 1
 };

 Cube.mkCube16_ :: Prelude.Module Cube.Cube_16_
 }
\end{verbatim}

The generated {\veri} module header:
\begin{verbatim}
 module mkCube16_(CLK,
                  RST,
                  RDY_start,   // output, asserted when start can accept
                  result,
                  RDY_result,  // output, asserted when result signal is valid
                  start_1,     // corresponds to first argument of start interface method
                  EN_start);   // input, assert when start method has valid data
   input CLK, RST;
   output RDY_start;
   output [15 : 0] result;
   output RDY_result;
   input [15 : 0] start_1;
   input EN_start;
   ...
\end{verbatim}

The naming conventions for the ports is to take the method name
(of the mangled interface) and suffix it with \te{\_$n$} for the
$n$th argument.  The output of a method will have the method name.
The enable signal (for actions) will have \te{EN\_} prefixed to the
method name.

\begin{NOTE}
The mangled interface is only there for informational purposes; it cannot be
used.  Perhaps there would be a better way to convey this information?
\end{NOTE}

\subsubsection{{\veri} code generation properties}
A number of properties can be specified for a module which affects
{\veri} code generation.

\gram{pragma}{\term{\{-\#} \term{properties} \nterm{varId} \term{=} \term{\{} \many{\nterm{cgprop} \term{,}} \term{\}} \term{\#-\}}}

\gram{cgprop}{\term{verilog}} \\
\gramalt{\term{alwaysReady}}\\
\gramalt{\term{alwaysEnabled}}\\
\gramalt{\term{scanInsert} \term{=} \nterm{int}}\\
\gramalt{\term{bitBlast}}\\
\gramalt{\term{CLK} \term{=} \nterm{varName}}\\
\gramalt{\term{RSTN} \term{=} \nterm{varName}}\\
\gramalt{\term{options} \term{=} \term{\LBRACE} \many{\nterm{string} \term{,}} \term{\RBRACE}}

\gram{varName}{\nterm{varId} \alt \nterm{conId} \alt \nterm{string}}

The \te{properties} pragma is given for a specific module and the listed
properties affect code generation.
\begindescrlist{alwaysEnabledxx}
\litem{\te{verilog}}
generate {\veri} for this module.  Without this, or the command line option,
no code is generated for this module, instead it will be inlined where used.
\litem{\te{alwaysReady}}
all methods in the module interface should be continuously ready, i.e., there
is no need to us the {\BH} ready signalling protocol so those wires are left
out.  The compiler verifies that the methods are indeed always ready.
\litem{\te{alwaysEnabled}}
all methods that are actions (i.e., where the type ends with \te{PrimAction})
are assumed to be continuously enabled, i.e., they execute in every cycle.
There is thus no need for the enable signal for the method and it is omitted.
The compiler generates code as if the enable wire was always high and verifies
that the method will fire in every clock cycle.
\litem{\te{scanInsert}}
put extra ports used for scan insertion into the generated {\veri} code.
The number specifices the number of scan chains to insert.
\litem{\te{bitBlast}}
do ``bit blasting'' of the generated ports, i.e., split ports that consist of
multiple bits into the individual bits, and also make all port names upper case.
\litem{\te{CLK}}
specify the name of the clock, the default is \te{CLK}.
\litem{\te{RSTN}}
specify the name of the reset, the default is \te{RST{\US}N}.
\litem{\te{options}}
specify additional compiler flags that override the current compiler flags
(as given on the command line).
\end{list}

The \te{alwaysReady} and \te{alwaysEnabled} properties are useful when the generated
code will be connected to other {\veri} modules that are not written in {\BH},
and where these modules assume a synchronous signalling protocol.

% This example used to demonstrate the ``name'' pragma, but that no longer
% exists, so this example is overly complicated now.
Example: A module which connects to an external synchronous SRAM
\begin{verbatim}
  {-# properties useSRAM = {
         alwaysReady,
         alwaysEnabled,
  } #-}
  useSRAM :: Module (SyncSRAMC 1 (Bit 20) (Bit 32))
  useSRAM =
     module
         (extram, ram) <- wrapSRAM
         ...
         interface (extram)
\end{verbatim}

Example: Do not perform ATS optimization for the module \te{slow}
\begin{verbatim}
  {-# properties slow = { options = { "-no-opt-ATS" } } #-}
  slow :: Module ...
\end{verbatim}

% ================================================================

\section{Interfacing to C}

The C code generated and used by the {\BH} compiler is structured
similarly to the {\veri} modules generated and used by the compiler.
Each {\veri} module corresponds to a ``class'' and its instances to
``objects''.  Since C is not object oriented the notion of an object
has to be simulated.{\footnote{Generating C++ would have been slightly
easier since it has objects.}}

Each ``class'' definition is a struct.  It first always contains
certain fields, a \qbs{struct obj}, which are explained below.
Following these are function pointers that implement all the methods
in the interface.  Each of these functions takes a pointer to the
``object'' itself as the first argument (the \qbs{self} pointer, which
is the standard way of implementing object oriented languages).  If
the interface method is an action ({\ie} its type ends in \te{Action})
the function will have one argument for each of the methods arguments.
If the method returns a value it will have an additional argument, the
second, where this value will be stored.  All arguments are passed by
reference; the type \qbs{varp} is used for updatable values and
{\qbs{varcp}} for constant value.  Each of these arguments represents
something which in {\BH} has type \qbs{Bit $n$} and in {\veri} is a
bunch of wires.

The initial part of each ``class'' ({\ie} its base class) has the
following definition:
\begin{verbatim}
 typedef struct obj *obj;
 struct obj {
     obj parent;
     const struct varinfo *vinfo;
     updfun update;
     dumpfun dump;
     uInt nrules;
     varp *preds;
     ruleinfo *rules;
     uInt nobjs;
     obj *objs;
 };
\end{verbatim}
None of these fields are needed for a user of an ``object'', but we will
explain them for completeness.
\begin{tightlist}
\item \te{parent} points to the object which this state element is a part of.
\item \te{vinfo} contains name etc. for this state variable.
\item \te{update} is the function that must be called when any action has been performed on the object.         It will recompute all the private fields in the object, and of all sub-objects.
\item \te{dump} will, if called, print the state of the object.
\item \te{nrules} the number of rules in this object.
\item \te{preds} pointers (\te{nrules} of them) point to the predicates for all the rules.
        Use the \te{GETBOOL()} macro to get the value of one of these.
\item \te{rules} points to an array of information for each rule.
        The rule information contains the name of the rule and the function to call to execute the rule.
\item \te{nobjs} the number of sub-objects contained in this object.
\item \te{objs} pointers to all the sub-objects.
\end{tightlist}

% ----------------------------------------------------------------

\subsection{C modules}

Wherever a {\veri} module is used when generating {\veri} a
corresponding C module is needed for generating C.  If the {\veri}
module, with interface type \qbs{{\rm\emph{ifc}}}, is named
\qbs{$mod$} the compiler will assume that there is a corresponding C
header file named {\qbs{$mod$.h}}.  This header file should contain a
type definition (a {\te{typedef}}) for the type \qbs{B{\rm\emph{ifc}}}
and a function called {\qbs{new\_$mod$}} returning a
{\qbs{B{\rm\emph{ifc}}}} object.

Example: The {\veri} register module, see section
{\ref{sec-param-and-args}}, has a corresponding C header file, named
{\qbs{RegN.h}}, with these contents:
\begin{verbatim}
 #if !defined(REG)
 #define REG
 typedef struct BReg *BReg;
 struct BReg {
     struct obj hdr;
     void (*Bget)(BReg, varp);
     void (*Bset)(BReg, varcp);
 };
 #endif

 BReg new_RegN(obj, const struct varinfo *, uInt, varcp);
\end{verbatim}

% ----------------------------------------------------------------

\subsection{Generated C}

The generated C code follows the conventions described in the
preceding sections.  If C code is generated for a module named
{\qbs{\rm\emph{templ}}}, the compiler will generate a header file
{\qbs{\rm\emph{templ}.h}} and a code file {\qbs{\rm\emph{templ}.c}}.

% ================================================================

\section{Guiding the compiler}

\subsection{Pragmas}
\index{pragma@\te{pragma} (keyword)}
To guide the compiler to do the right thing there are a number of
pragmas.  Pragmas can be used where top level definitions are valid.
Pragmas have the following general form:

\gram{topDefn}{\nterm{pragma}}

\gram{pragma}{\term{\{-\#} \nterm{pragmaId} $\cdots$ \term{\#-\}}}

\gram{pragmaId}{\nterm{varId}}

Syntactically, pragmas are comments because they are enclosed in
{\verb|{-|} and {\verb|-}|} brackets.

% ----------------------------------------------------------------

\subsubsection{Pragma \te{verilog}}

\index{verilog@\te{verilog} (pragma)}
When the compiler generates code for a module it normally tries to
integrate all definitions into one big {\veri} module.  If this is not
desirable for some reason you can use the \te{verilog} pragma to
instruct the compiler to generate {\veri} modules for parts of the
design.

The syntax is:

\gram{pragma}{\term{\{-\#} \term{verilog} \nterm{varId} \opt{\many{\nterm{veriProp} \term{,}}} \term{\#-\}}}

\gram{veriProp}{\term{noReady} ~ \alt ~ \term{alwaysEnabled}}

This will tell the compiler to generate {\veri} modules for the named
module when it is doing code generation.

Some properties of the generated code can be specified as well:
\begindescrlist{xxxxxxx}
\litem{\te{noReady}}
specifies that no ready signals should be generated.  The compiler verifies that
all the methods in the interface are permanently ready.
\litem{\te{alwaysEnabled}}
specifies that there should be no enable signal for action methods.
The method will be executed on every clock cycle, and the compiler
verifies that the caller does this.
\end{list}

\note{It is currently not possible to give these properties for individual
method, just for the whole interface.}
% ----------------------------------------------------------------

\subsubsection{Pragma \te{noinline}}

\index{noinline@\te{noinline} (pragma)}
The \te{noinline} pragma can be given for functions, it tells the
compiler not to inline the function, but to generate code for
it directly.  The function has same type restrictions as for interface methods
that are involved in code generation.

The syntax is:

\gram{pragma}{\term{\{-\#} \term{noinline} \many{\nterm{varId}} \term{\#-\}}}

Example:
\begin{verbatim}
{-# noinline cswap #-}
cswap :: Bool -> (Int 32, Int 32) -> (Int 32, Int 32)
cswap True (x, y) = (y, x)
cswap False xy = xy
\end{verbatim}

% ----------------------------------------------------------------

%% \subsubsection{Pragma \te{multiplicity}}

%% \index{multiplicity@\te{multiplicity} (pragma)}
%% The \te{multiplicity} pragma can be used to tell the compiler to
%% generate multiple ports of the same kind for a {\veri} module.  This
%% corresponds to what was described in \ref{sec-multiple} about
%% interfacing to {\veri}.

%% The syntax is:

%% \gram{pragma}{\term{\{-\#} \term{multiplicity} \nterm{varId} \term{\{} \many{\nterm{fieldId} \nterm{mult} \term{;}} \term{\}} \term{\#-\}}}

%% This tells the compiler how many ports to generate for different
%% fields of module \nterm{varId}.  The default is one port.

%% \note{This pragma is not fully implemented yet.}

% ----------------

\subsection{Rule assertions}

\index{rule assertions|textbf}
Rule assertions instruct the compiler to abort compilation unless it
can verify that a rule satisfies a particular condition.  Each
assertion affects the rule that immediately follows it and all rules
nested within.

\gram{rule} \nterm{ruleAssert} \opt{\term{;}} \nterm{rule}

Rule assertions are not triggered until the generation of \veri{} or C
code for the module that includes them.

% ----------------

\subsubsection{Assertion \te{fire when enabled}}

\index{fire when enabled@\te{fire when enabled} (rule assertion)}
\index{rule assertions!\te{fire when enabled}}
This asserts that a rule is scheduled to fire whenever its predicate
and its implicit conditions are true, {\ie} when they are true, there
are no scheduling conflicts that will prevent it from firing.

\gram{ruleAssert}{\term{\{-\#} \term{ASSERT}
	\term{fire} \term{when} \term{enabled} \term{\#-\}}}

% ----------------

\subsubsection{Assertion \te{no implicit conditions}}

\index{no implicit conditions@\te{no implicit conditions} (rule assertion)}
\index{rule assertions!\te{no implicit conditions}}
\index{implicit conditions!asserting that a rule has none}
This asserts that interface methods called within the rule do not have
implicit conditions that contribute to its enabling, {\ie} only the
explicit rule predicate controls whether it is enabled or not.

\gram{ruleAssert}{\term{\{-\#} \term{ASSERT}
	\term{no} \term{implicit} \term{conditions} \term{\#-\}}}


% ================================================================

\bibliography{BH_lang}
\bibliographystyle{alpha}

% ================================================================

\appendix

% ================================================================

\newpage

\section{Advanced topics}

This section contains topics that are not necessary for the beginning
{\BH} programmer.

% ----------------------------------------------------------------

\subsection{Lambda expressions}

\label{sec-lambda}

\index{lambda expressions|textbf}
Value definitions (section {\ref{sec-val-defs}}) enable definition of
functions, but it is bundled with also binding the function value to a
name.  It is possible to define a function value independently of
giving it a name, {\ie} to define a function value ``anonymously''
using so-called ``$\lambda$-expressions'':

\gram{exp}{\term{\BSL} \many{\nterm{varId}} \term{->} \nterm{exp}}
\index{=minusgt@\te{->} (in lambda expressions)}

Example:
\begin{verbatim}
 \x -> x * x
\end{verbatim}
This defines the ``squaring function'', {\ie} a function of one
argument that returns the product of that argument with itself.

\tbd{Allow irrefutable patterns instead of variables for $\lambda$.}

% ----------------------------------------------------------------

\subsection{\te{do}}

\label{sec-do}

\gram{exp}{ \term{do} \term{\{} \many{\nterm{stmt} \term{;}} \nterm{exp} \term{\}} }
\index{do expressions}
\index{do@\te{do} (keyword)}

The {\te{do}} expression in {\BH} provides a convenient syntax for
programming with {\em monads} {\cite{monads,monads2}}.  A translation
of the {\te{do}} expression into simpler expressions is given in the
Haskell report {\cite{haskell12}}.

The value of a {\te{do}} statement is the value of the very last
expression.  This last expression is commonly a call to the method
{\te{return}} which is defined for any monad.  {\te{return}} takes a
value and returns a monad type with that value.

% ----------------

\subsubsection{Creating modules with \te{do}}

To further illustrate how the \te{module} syntax is nearly just a
{\te{do}} expression and the interface nearly a \te{struct}, consider
the following example which creates a module without the special
syntax:

\begin{verbatim}
 struct Two =
     a :: Reg (Bit 5)
     b :: Reg (Bit 10)

 mkTwo :: Module Two
 mkTwo = do
     a <- mkReg 0
     b <- mkReg 0
     return (Two { a = a; b = b })
\end{verbatim}

%In the current unstable state of the compiler, this works.  It is
%likely that, as the compiler settles, the type parameter of modules
%will be required to be an interface---at least for any top-level
%compiled modules.

\subsubsection{Recursive bindings in module}
Normally (i.e., in Haskell) the bindings in a \te{do} expression come into scope in order,
but {\BH} also allows forward references to variables bound in a do expression.

Example: The following is legal
\begin{verbatim}
 do
     x <- foo y
     y <- bar x
     return (x, y)
\end{verbatim}

Such recursive bindings does not make sense in all monads so there is a type
restriction to capture this.  Normally a \te{do} expression has type \te{(Monad m) => m \it{t}},
but a \te{do} expression with as forward reference has type \te{(MonadFix m) => m \it{t}}.
The class \te{MonadFix} is defined as
\begin{verbatim}
 class (Monad m) => MonadFix m where
     mfix :: (a -> m a) -> m a
\end{verbatim}

A \te{do} with forward references is transformed into an ordinary \te{do} as follows:
Let \te{x1} $\cdots$ \te{xk} be the identifiers that are referenced forwardly.
\begin{verbatim}
 do
     s1
     ...
     sn
     e
\end{verbatim}
transforms to (where \te{n}, \te{t}, and \te{r} are fresh identifiers)
\begin{verbatim}
 do
     n <- mfix (\ t -> let (r, x1, ... xk) = t
                       in  do
                               s1
                               ...
                               sn
                               r <- e
                               return (r, x1, ... xk)
               )
     return n.fst
\end{verbatim}

Both the \te{Module} and \te{ActionValue} monads belong to the class \te{MonadFix}.


% ----------------------------------------------------------------

\subsection{\te{IsModule}}

\label{sec-ismodule}

\index{IsModule@\te{IsModule} (typeclass)}
\index{liftModule@\te{liftModule} (\te{IsModule} method)}
The type constructor \te{Module} is a primitive type that the compiler
knows about.  It is possible to build variations of this type
within {\BH}.  To express that a type is related to \te{Module}
we use the class \te{IsModule}.
The special \te{module} syntax is available for all types that belong
to \te{IsModule}.

\begin{verbatim}
 class (Monad m) => IsModule m where
     liftModule :: Module a -> m a
\end{verbatim}

The \te{liftModule} function is the conversion from a standard
module into the augmented module type.

Naturally, the \te{Module} type trivially belongs to the \te{IsModule} class.

\begin{verbatim}
 instance IsModule Module where
     liftModule  m  = m
\end{verbatim}

All the primitive state generators, e.g., \te{mkReg}, have a type that is
general enough that they can be used in any module variation.

\begin{verbatim}
 mkReg :: (IsModule m, Bits a sa) => a -> m (Reg a)
\end{verbatim}

It is very easy to make a value with such a type; just apply
\te{liftModule} to an ordinary \te{Module} value.

% ================================================================

\newpage

\section{Syntax}

\label{sec-syntax}

\subsection{Reserved words}

The following words are reserved in {\BH}:
\begin{quote}
\te{action as} \\
\te{case class} \\
\te{data default data deriving do} \\
\te{else} \\
\te{foreign} \\
\te{hiding} \\
\te{if import in infix infixl infixr instance interface} \\
\te{let} \\
\te{module} \\
\te{newtype} \\
\te{of} \\
\te{package prefix primitive} \\
\te{qualified} \\
\te{return rules} \\
\te{signature struct} \\
\te{then type} \\
\te{valueOf verilog} \\
\te{when where}
\end{quote}

% ================================================================

\newpage

\section{Semantics of primitive operations}

\label{sec-primitives}

The {\BH} compiler internally translates a {\BH} program until
it is entirely defined in terms of primitive operations and
external modules.  This section describes the semantics of the
primitive operations.

All primitive operations are defined for all sizes (including 0)
given the type constraints.
All numbers are interpreted in two's complement representation
where applicable.

\begindescrlist{xxxxxxxxxxx}
\litem{\te{primitive primAdd :: Bit n -> Bit n -> Bit n}}
	\mbox{}\\
	Adds two $n$ bit numbers and returns the lower $n$ bits of the result.\\
	Generates a {\veri} \qbs{+}.
\litem{\te{primitive primSub :: Bit n -> Bit n -> Bit n}}
	\mbox{}\\
	Subtracts two $n$ bit numbers and returns the lower $n$ bits of the result.\\
	Generates a {\veri} \qbs{-}.
\litem{\te{primitive primMul :: Bit n -> Bit n -> Bit n}}
	\mbox{}\\
	Multiplies two $n$ bit numbers and returns the lower $n$ bits of the result.\\
	Generates a {\veri} \qbs{*}.
\litem{\te{primitive primNeg :: Bit n -> Bit n}}
	\mbox{}\\
	Negation of an $n$ bit number.\\
	Generates a {\veri} \qbs{-}.
\litem{\te{primitive primAnd :: Bit n -> Bit n -> Bit n}}
	\mbox{}\\
	Bitwise AND of two $n$ bit numbers.\\
	Generates a {\veri} \qbs{\&}.
\litem{\te{primitive primOr  :: Bit n -> Bit n -> Bit n}}
	\mbox{}\\
	Bitwise OR of two $n$ bit numbers.\\
	Generates a {\veri} \qbs{|}.
\litem{\te{primitive primXor :: Bit n -> Bit n -> Bit n}}
	\mbox{}\\
	Bitwise XOR of two $n$ bit numbers.\\
	Generates a {\veri} ``\HAT''.
\litem{\te{primitive primSL  :: Bit n -> Nat -> Bit n}}
	\mbox{}\\
	Shift the first argument left by the number of bits given by the second argument.
	Vacated positions are filled with 0.  The behaviour is undefined if the shift
	count is equal or larger than the width of the shifted word. \\
	Generates a {\veri} \qbs{<{}<} if the shift count is non-constant, otherwise just bit selection.
\litem{\te{primitive primSRL :: Bit n -> Nat -> Bit n}}
	\mbox{}\\
	Shift the first argument right by the number of bits given by the second argument.
	Vacated positions are filled with 0.  The behaviour is undefined if the shift
	count is equal or larger than the width of the shifted word. \\
	Generates a {\veri} \qbs{>{}>} if the shift count is non-constant, otherwise just bit selection.
\litem{\te{primitive primSRA :: Bit n -> Nat -> Bit n}}
	\mbox{}\\
	Shift the first argument right by the number of bits given by the second argument.
	Vacated positions are filled with the most significant bit.  The behaviour is undefined if the shift
	count is equal or larger than the width of the shifted word. \\
	Generates a {\veri} \qbs{>{}>} and replication of the sign if the shift count is non-constant, otherwise just bit selection.
\litem{\te{primitive primInv :: Bit n -> Bit n}}
	\mbox{}\\
	Bitwise inverting of an $n$ bit number.\\
	Generates a {\veri} ``\TILDE''.
\litem{\te{primitive primEQ  :: Bit n -> Bit n -> Bit 1}}
	\mbox{}\\
	Comparison of two $n$ bit numbers, returns 1 if they are equal otherwise 0. \\
	Generates a {\veri} \qbs{==}.
\litem{\te{primitive primULE :: Bit n -> Bit n -> Bit 1}}
	\mbox{}\\
	Unsigned comparison of two $n$ bit numbers, returns 1 if the first one is less than or equal to the second, otherwise 0. \\
	Generates a {\veri} \qbs{<=}.
\litem{\te{primitive primULT :: Bit n -> Bit n -> Bit 1}}
	\mbox{}\\
	Unsigned comparison of two $n$ bit numbers, returns 1 if the first one is less than the second, otherwise 0. \\
	Generates a {\veri} \qbs{<}.
\litem{\te{primitive primSLE :: Bit n -> Bit n -> Bit 1}}
	\mbox{}\\
	Signed comparison of two $n$ bit numbers, returns 1 if the first one is less than or equal to the second, otherwise 0. \\
	Generates a {\veri} \qbs{<=} with the sign bits inverted.
\litem{\te{primitive primSLT :: Bit n -> Bit n -> Bit 1}}
	\mbox{}\\
	Signed comparison of two $n$ bit numbers, returns 1 if the first one is less than the second, otherwise 0. \\
	Generates a {\veri} \qbs{<} with the sign bits inverted.
\litem{\te{primitive primZeroExt :: (Add n k m) => Bit n -> Bit m}}
	\mbox{}\\
	Extend the argument with 0 on the left to make it the right size. \\
	Generates a {\veri} bit concatenation.
\litem{\te{primitive primSignExt :: (Add n k m) => Bit n -> Bit m}}
	\mbox{}\\
	Extend the argument with the sign bit replicated on the left to make it the right size.
	An argument of size 0 is assumed to have a 0 sign bit. \\
	Generates a {\veri} bit concatenation.
\litem{\te{primitive primTrunc :: (Add k m n) => Bit n -> Bit m}}
	\mbox{}\\
	Truncate bits on the left.
	Generates a {\veri} bit extraction.
\litem{\te{primitive primBNot :: Bit 1-> Bit 1}}
	\mbox{}\\
	Boolean NOT.\\
	Generates a {\veri} \qbs{!}.
\litem{\te{primitive primBAnd :: Bit 1 -> Bit 1 -> Bit 1}}
	\mbox{}\\
	Boolean AND.  During compile time this is a ``short circuit'' operator that avoids evaluating the second operand if possible.\\
	Generates a {\veri} \qbs{\&\&}.
\litem{\te{primitive primBOr :: Bit 1 -> Bit 1 -> Bit 1}}
	\mbox{}\\
	Boolean OR.  During compile time this is a ``short circuit'' operator that avoids evaluating the second operand if possible.\\
	Generates a {\veri} \qbs{||}.
\end{list}

% ================================================================

\newpage

\section{The Standard Prelude and Additional Libraries}

\label{sec-prelude}
\label{sec-additional-libs}

Please see the separate document {\LibRefGuide} for the Standard
Prelude and Additional Libraries.

% ================================================================

\clearpage
\phantomsection
\addcontentsline{toc}{section}{Index}
\printindex

\end{document}
