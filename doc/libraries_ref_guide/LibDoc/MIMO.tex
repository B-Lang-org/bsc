\subsubsection{MIMO}

\index{MIMO@\te{MIMO} (package)}

\label{MIMO}

{\bf Package}


\begin{verbatim}
import MIMO :: *
\end{verbatim}



{\bf Description} 

This package defines a Multiple-In Multiple-Out (MIMO), an enhanced
FIFO that allows the designer to specify the number  of objects enqueued and
dequeued in one cycle.  There are different implementations of the
MIMO available for synthesis: BRAM, Register, and Vector.


{\bf Types and type classes}
\index{MIMOConfiguration@\te{MIMOConfiguration} (type)}
\index{LUInt@\te{LUInt} (type)}

The \te{LUInt} type is a \te{UInt} defined as the log of the \te{n}.

\begin{libverbatim}
typedef UInt#(TLog#(TAdd#(n, 1)))     LUInt#(numeric type n);
\end{libverbatim}

The \te{MimoConfiguration} type defines whether the \te{MIMO} is
guarded or unguarded, and whether it is based on a BRAM.  There is an
instance in the \te{DefaultValue} type class.  The default \te{MIMO}
is guarded and not based on a BRAM.


\begin{libverbatim}
typedef struct {
   Bool         unguarded;
   Bool         bram_based;
} MIMOConfiguration deriving (Eq);
\end{libverbatim}



\begin{libverbatim}
instance DefaultValue#(MIMOConfiguration);
   defaultValue = MIMOConfiguration {
      unguarded:  False,
      bram_based: False
      };
endinstance
\end{libverbatim}



{\bf Interfaces and methods}
\index{MIMO@\te{MIMO} (interface)}

 The \te{MIMO} interface 
 is polymorphic and takes 4 parameters: \te{max\_in}, \te{max\_out},
 \te{size}, and \te{t}.


\begin{center}
\begin{tabular}{|p{.8in}|p{4.5 in}|}
\hline
\multicolumn{2}{|c|}{MIMO Interace Parameters}\\
\hline
Name&Description\\
\hline
\hline
\te{max\_in}&Maximum number of objects enqueued in one cycle.  Must be
numeric.\\
\hline
\te{max\_out}&Maximum number of objects dequeued in one cycle.  Must be
numeric.\\
\hline
\te{size} & Total size of internal storage.  Must be
numeric.\\
\hline
\te{t} & Data type of the stored objects\\
\hline
\end{tabular}
\end{center}

The \te{MIMO} interface provides the following methods: \te{enq},
 \te{first}, \te{deq}, \te{enqReady}, \te{enqReadyN}, \te{deqReady},
 \te{deqReadyN}, \te{count}, and \te{clear}. 

\begin{center}
%\begin{tabular}{|p{.6in}|p{1.1 in}|p{3.6 in}|}
\begin{tabular}{|p{.7in}|p{1.4in}|p{2 in}|p{1.8in}|}
\hline
\multicolumn{4}{|c|}{MIMO methods}\\
\hline
\multicolumn{3}{|c|}{Method}&\multicolumn{1}{|c|}{Argument}\\
\hline
Name & Type & Description& \\
\hline
\hline 
\te{enq}&Action& adds an entry to the
\te{MIMO}&\te{LUInt\#(max\_in)}\\
&&&\te{Vector\#(max\_in, t) data)}  \\
\hline
\te{first}&\te{Vector\#(max\_out, t)}&Returns a Vector containing
\te{max\_out} items of type \te{t}. &\\
\hline
\te{deq}&Action&Removes the first \te{count} entries&\te{LUint\#(max\_out)}\\
&&&\te{count}\\
\hline
\te{enqReady}& Bool& Returns a True value if there is space to enqueue
an entry&\\
\hline
\te{enqReadyN}&Bool&Returns a True value if there is space to enqueue
\te{count} entries&\te{LUint\#(max\_in)}\\
\hline
\te{deqReady}& Bool& Returns a True value if there is an element to dequeue
&\\
\hline
\te{deqReadyN}&Bool&Returns a True value if there is are \te{count} elements to dequeue
&\\
\hline
\te{count}&\te{LUint\#(size)}&Returns the log of the number of elements in the
\te{MIMO}&\\
\hline
\te{clear}&Action&Clears the \te{MIMO} &\\
\hline
\hline
\end{tabular}
\end{center}


\begin{verbatim}
interface MIMO#(numeric type max_in, numeric type max_out, numeric type size, type t);
   method    Action               enq(LUInt#(max_in) count, Vector#(max_in, t) data);
   method    Vector#(max_out, t)  first;
   method    Action               deq(LUInt#(max_out) count);
   method    Bool                 enqReady;
   method    Bool                 enqReadyN(LUInt#(max_in) count);
   method    Bool                 deqReady;
   method    Bool                 deqReadyN(LUInt#(max_out) count);
   method    LUInt#(size)         count;
   method    Action               clear;
endinterface
\end{verbatim}



{\bf Modules}

The package provides  modules to synthesize different implementations
of the MIMO: the basic MIMO (\te{mkMIMO}), BRAM-based (\te{mkMIMOBRAM}), register-based
(\te{mkMIMOReg}), and a Vector of registers (\te{mkMIMOV}).

All implementations must meet the
following provisos: 
\begin{itemize}

\item The object must have bit representation
\item The object must have at least 2 elements of storage.
\item   The
maximum number of objects enqueued (\te{max\_in}) must be less than or equal
to the total bits of storage (\te{size})
\item  The
maximum number of objects dequeued (\te{max\_out}) must be less than or equal
to the total bits of storage (\te{size})
\end{itemize}

\index{mkMIMO@\te{mkMIMO} (module)}
\index[function]{MIMO!mkMIMO}

\begin{center}
\begin{tabular}{|p{.7 in}|p{5.4 in}|}
 \hline
&         \\
\te{mkMIMO}& The basic implementation of MIMO.  Object must be at
least 1 bit in size.\\
\cline{2-2} 
& \begin{libverbatim}
module mkMIMO#(MIMOConfiguration cfg)(MIMO#(max_in, max_out, size, t)); 
\end{libverbatim} 
\\
\hline
\end{tabular}
\end{center}

\index{mkMIMOBRAM@\te{mkMIMOBRAM} (module)}
\index[function]{MIMO!mkMIMOBRAM}

\begin{center}
\begin{tabular}{|p{.7 in}|p{5.4 in}|}
 \hline
&         \\
\te{mkMIMOBRAM}&  Implementation of BRAM-based MIMO.  Object must be
at least 1 byte in size.\\
\cline{2-2} 
& \begin{libverbatim}
module mkMIMOBram#(MIMOConfiguration cfg)(MIMO#(max_in, max_out, size, t));
\end{libverbatim} 
\\
\hline
\end{tabular}
\end{center}

\index{mkMIMOReg@\te{mkMIMOReg} (module)}
\index[function]{MIMO!mkMIMOReg}

\begin{center}
\begin{tabular}{|p{.7 in}|p{5.4 in}|}
 \hline
&         \\
\te{mkMIMOReg}& Implementation of register-based MIMO. \\
\cline{2-2} 
& \begin{libverbatim}
module mkMIMOReg#(MIMOConfiguration cfg)(MIMO#(max_in, max_out, size, t));
\end{libverbatim} 
\\
\hline
\end{tabular}
\end{center}

\index{mkMIMOV@\te{mkMIMOV} (module)}
\index[function]{MIMO!mkMIMOV}

\begin{center}
\begin{tabular}{|p{.7 in}|p{5.4 in}|}
 \hline
&         \\
\te{mkMIMOV}&Implementation of Vector-based MIMO.  The ojbect must
have a default value defined.  \\
\cline{2-2} 
& \begin{libverbatim}
module mkMIMOV(MIMO#(max_in, max_out, size, t));
\end{libverbatim} 
\\
\hline
\end{tabular}
\end{center}
