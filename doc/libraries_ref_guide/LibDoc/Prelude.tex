\subsection{Type classes}
\index{type classes}

A type class   groups  related functions and operators and allows for
  instances across the various datatypes which are members of the typeclass.
 Hence the function
  names within a type class are \emph{overloaded} across the various
  type class members.  

A  {\tt typeclass} declaration creates a type class.  An {\tt instance}
  declaration defines a datatype as belonging to a type class.  A
  datatype may belong to zero or many type classes.

\com{typeclasses}
The Prelude package declares the following type
classes:

\begin{center}
\begin{tabular}{|p{1in}|p{4 in}|}
\hline
\multicolumn{2}{|c|}{Prelude Type Classes}\\
\hline
\te{Bits}&Types that can be converted to bit vectors and back.\\
\hline
\te{Eq}&Types on which equality is defined.\\
\hline
\te{Literal}&Types  which can be created from integer literals.\\
\hline
\te{RealLiteral}&Types  which can be created from real literals.\\
\hline
\te{Arith}&Types on which arithmetic operations are defined.\\
\hline
\te{Ord}&Types on which comparison operations are defined.\\
\hline
\te{Bounded}&Types with a finite range.\\
\hline
\te{Bitwise}&Types on which bitwise operations are defined.\\
\hline
\te{BitReduction}&Types on which bitwise operations on a single
operand to produce a single bit result are defined.\\
\hline
\te{BitExtend}&Types on which  extend operations are defined.\\
\hline
\te{SaturatingArith}&Types with functions to  describe how overflow
and underflow should be handled.\\
\hline
\te{Alias} & Types which can be used interchangeably.\\
\hline
\te{NumAlias} & Types which give a new name to a numeric type.\\
\hline
\te{FShow} & Types which can convert a value to a \te{Fmt}
representation for use with \te{\$display} system tasks.\\
\hline
\te{StringLiteral} &Types which can be created around strings.\\
\hline
\end{tabular}
\end{center}


\com{Bits}
\subsubsection{Bits}

\te{Bits} defines the class of types that can be converted to bit
vectors  and back.  Membership in this class is required for a data
type to be stored in a state, such as  a Register or a FIFO, or to be used
at a synthesized module boundary.   Often instance of this
class can be automatically derived using the \te{deriving} statement.

\index{Bits@\te{Bits} (type class)}
\index{pack@\texttt{pack} (\texttt{Bits} type class overloaded function)}
\index{unpack@\texttt{unpack} (\texttt{Bits} type class overloaded function)}
\index[function]{Prelude!pack}
\index[function]{Prelude!unpack}
\index[typeclass]{Bits}

\begin{libverbatim}
   typeclass Bits #(type a, numeric type n);
       function Bit#(n) pack(a x);
       function a unpack(Bit#(n) x);
   endtypeclass
\end{libverbatim}
Note: the numeric keyword is not required

The functions {\tt pack} and {\tt unpack} are provided to convert
elements to {\tt Bit\#()} and to convert {\tt Bit\#()} elements to another datatype.
\begin{center}
\begin{tabular}{|p{1 in}|p{4in}|}
\hline
\multicolumn{2}{|c|}{\te{Bits} Functions}\\
\hline
\hline
\te{pack}&Converts element {\tt a} of datatype {\tt data\_t} to a
element of datatype {\tt Bit\#()} of {\tt size\_a}. \\
\cline{2-2}
&\\
& \te{function Bit\#(size\_a) pack(data\_t a);}\\
&\\
\hline
\end{tabular}
\end{center}
\begin{center}
\begin{tabular}{|p{1 in}|p{4in}|}
\hline
\te{unpack}&Converts an element {\tt a} of datatype {\tt Bit\#()} and {\tt
size\_a} into an element with of  element type {\tt data\_t}.\\
\cline{2-2}
&\\
& \te{function data\_t unpack(Bit\#(size\_a) a);}\\
&\\
\hline
\end{tabular}
\end{center}

{\bf Examples}

\begin{verbatim}
   Reg#(int) cycle <- mkReg (0);
   ..
   rule r;
     ...
      if (pack(cycle)[0] == 0) a = a + 1;
      else                     a = a + 2;
      if (pack(cycle)[1:0] == 3) a = a + 3;  

    Int#(10) src_step    = unpack(config6[9:0]);
    Bool     src_rdy_en  = unpack(config6[16]);
\end{verbatim}

\com{Eq}
\subsubsection{Eq}

\te{Eq} defines the class of types whose values can be compared for equality.
Instances of the \te{Eq} class are often automatically derived   using
the  \te{deriving} statement.

\index{Eq@\te{Eq} (type class)}
\index{==@{\te{==}} (\te{Eq} class method)}
\index{/=@{\te{/=}} (\te{Eq} class method)}
\index[function]{Prelude!==}
\index[function]{Prelude!"!=}
\index[typeclass]{Eq}

\begin{libverbatim}
   typeclass Eq #(type data_t);
       function Bool \== (data_t x, data_t y);
       function Bool \/= (data_t x, data_t y);
   endtypeclass
\end{libverbatim}

The equality functions {\tt ==} and {\tt !=} are Boolean functions
which return a value of {\tt True} if the equality condition is met.
  When defining an instance of an
\te{Eq} typeclass, the \verb'\==' and \verb'\/=' notations must be
used.  If using or referring to the functions, the standard {\V} operators {\tt
==} and {\tt !=} may be used.
\begin{center}
\begin{tabular}{|p{1 in}|p{4in}|}
\hline
\multicolumn{2}{|c|}{\te{Eq} Functions}\\
\hline
\hline
\te{==}& Returns {\tt True} if {\tt x} is equal to {\tt y}.\\
\cline{2-2}
&\begin{libverbatim}function Bool \== (data_t x, data_t y,);
\end{libverbatim}
\\
\hline
\end{tabular}
\end{center}
\begin{center}
\begin{tabular}{|p{1 in}|p{4in}|}
\hline
\te{/=}& Returns {\tt True} if {\tt x} is not equal to {\tt y}.\\
\cline{2-2}
&\\
&\begin{libverbatim}function Bool \/= (data_t x, data_t y,);
\end{libverbatim}
\\
\hline
\end{tabular}
\end{center}

{\bf Examples}
\begin{verbatim}
  return (pack(i) & 3) == 0;
 
  if (a != maxInt)
\end{verbatim}


\com{Literal}
\subsubsection{Literal}
\label{literal}

\te{Literal} defines the class of types which can be created from
integer literals.
\index{Literal@\te{Literal} (type class)}
\index{fromInteger@\te{fromInteger} (\te{Literal} class method)}
\index[function]{Prelude!fromInteger}
\index[typeclass]{Literal}

\begin{libverbatim}
   typeclass Literal #(type data_t);
       function data_t fromInteger(Integer x);
       function Bool   inLiteralRange(data_t target, Integer x);
   endtypeclass
\end{libverbatim}

The {\tt fromInteger} function converts an {\tt Integer} into an
element of  datatype {\tt data\_t}.  Whenever you write an integer
literal in \BSV (such as ``0'' or ``1''), there is an implied \te{fromInteger}
applied to it, which turns the literal into the type you are using it
as (such as \te{Int}, \te{UInt}, \te{Bit}, etc.).  By defining an
instance of \te{Literal} for your own datatypes, you can create values
from literals just as for these predefined types.

The typeclass also provides a function \texttt{inLiteralRange} that takes 
an argument of the target type and an \texttt{Integer} and returns a 
\texttt{Bool} that indicates whether the \texttt{Integer} argument is in 
the legal range of the target type. For example, assuming \texttt{x} has type 
\texttt{Bit\#(4)}, \texttt{inLiteralRange(x, 15)} would return \texttt{True}, 
but \texttt{inLiteralRange(x,22)} would return \texttt{False}.

\begin{center}
\begin{tabular}{|p{1 in}|p{4in}|}
\hline
\multicolumn{2}{|c|}{\te{Literal} Functions}\\
\hline
\hline
\te{fromInteger}&Converts an element {\tt x} of datatype {\tt Integer} into an
element of  data type \te{data\_t}\\
\cline{2-2}
&\begin{libverbatim} function data_t fromInteger(Integer x);
\end{libverbatim}
\\
\hline
\end{tabular}
\end{center}
\begin{center}
\begin{tabular}{|p{1 in}|p{4in}|}
\hline
\hline
\te{inLiteralRange}&Tests whether an element {\tt x} of datatype {\tt Integer} is in
the legal range of data type \te{data\_t}\\
\cline{2-2}
&\begin{libverbatim} function Bool inLiteralRange(data_t target, Integer x);
\end{libverbatim}
\\ \hline
\end{tabular}
\end{center}

{\bf Examples}
\begin{verbatim}
   function foo (Vector#(n,int) xs) provisos (Log#(n,k));
       Integer maxindex = valueof(n) - 1;
       Int#(k) index;
       index = fromInteger(maxindex);
       ...
   endfunction

   function Bool inLiteralRange(RegAddress a, Integer i);
      return(i >= 0 && i < 83);
   endfunction
\end{verbatim}


\com{RealLiteral}
\subsubsection{RealLiteral}
\label{prelude-realliteral}

\te{RealLiteral} defines the class of types which can be created from
real literals.
\index{RealLiteral@\te{RealLiteral} (type class)}
\index{fromReal@\te{fromReal} (\te{RealLiteral} class method)}
\index[function]{Prelude!fromReal}
\index[typeclass]{RealLiteral}

\begin{libverbatim}
   typeclass RealLiteral #(type data_t);
       function data_t fromReal(Real x);
   endtypeclass
\end{libverbatim}

The {\tt fromReal} function converts a {\tt Real} into an
element of  datatype {\tt data\_t}.  Whenever you write a real
literal in \BSV  (such as ``3.14''), there is an implied \te{fromReal}
applied to it, which turns the real into the specified type.
  By defining an
instance of \te{RealLiteral} for a datatype, you can create values
from reals for any type.

% The typeclass also provides a function \texttt{inLiteralRange} that takes 
% an argument of the target type and an \texttt{Integer} and returns a 
% \texttt{Bool} that indicates whether the \texttt{Integer} argument is in 
% the legal range of the target type. For example, assuming \texttt{x} has type 
% \texttt{Bit\#(4)}, \texttt{inLiteralRange(x, 15)} would return \texttt{True}, 
% but \texttt{inLiteralRange(x,22)} would return \texttt{False}.

\begin{center}
\begin{tabular}{|p{1 in}|p{4in}|}
\hline
\multicolumn{2}{|c|}{\te{RealLiteral} Functions}\\
\hline
\hline
\te{fromReal}&Converts an element {\tt x} of datatype {\tt Real} into an
element of  data type \te{data\_t}\\
\cline{2-2}
&\begin{libverbatim} function data_t fromReal(Real x);
\end{libverbatim}
\\
\hline
\end{tabular}
\end{center}
% \begin{center}
% \begin{tabular}{|p{1 in}|p{4in}|}
% \hline
% \hline
% \te{inLiteralRange}&Tests whether an element {\tt x} of datatype {\tt Integer} is in
% the legal range of data type \te{data\_t}\\
% \cline{2-2}
% &\begin{libverbatim} function Bool inLiteralRange(data_t target, Integer x);
% \end{libverbatim}
% \\ \hline
% \end{tabular}
% \end{center}


{\bf Examples}
\begin{verbatim}
   FixedPoint#(is, fs) f = fromReal(n);  //n is a Real number
\end{verbatim}


\subsubsection{SizedLiteral}
\label{sizedliteral}

\te{SizedLiteral} defines the class of types which can be created from
integer literals with a specified size.
\index{SizedLiteral@\te{SizedLiteral} (type class)}
\index{fromSizedInteger@\te{fromSizedInteger} (\te{SizedLiteral} class method)}
\index[function]{Prelude!fromSizedInteger}
\index[typeclass]{SizedLiteral}

\begin{libverbatim}
   typeclass SizedLiteral #(type data_t, type size_t)
     dependencies (data_t determines size_t);
       function data_t fromSizedInteger(Bit#(size_t);
   endtypeclass
\end{libverbatim}

The {\tt fromSizedInteger} function converts a literal of type 
\te{Bit\#(size\_t)} into an
element of  datatype {\tt data\_t}.  Whenever you write a sized
literal like \te{1'b0}, there is an implied \te{fromSizedInteger}
which turns the literal into the type you are using it as, with the
defined size.  Instances  are  defined for the types \te{Bit}, \te{UInt}, and \te{Int}.

\begin{center}
\begin{tabular}{|p{1.2 in}|p{3.8in}|}
\hline
\multicolumn{2}{|c|}{\te{SizedLiteral} Functions}\\
\hline
\hline
\te{fromSizedInteger}&Converts an element of \te{Bit\#(size\_t)} into an 
element of  data type \te{data\_t}\\
\cline{2-2}
&\begin{libverbatim} function data_t fromSizedInteger(Bit#(size_t));
\end{libverbatim}
\\
\hline
\end{tabular}
\end{center}


\com{Arith}
\subsubsection{Arith}

\te{Arith} defines the class of types on which arithmetic operations are defined.
\index{Arith@\te{Arith} (type class)}
\index{+@{\te{+}} (\te{Arith} class method)}
\index{-@{\te{-}} (\te{Arith} class method)}
\index{negate@\te{negate} (\te{Arith} class method)}
\index{*@{\te{*}} (\te{Arith} class  method)}
\index{\%@{\te{\%}}(\te{Arith} class mod method)}
\index{/@{\te{/}}  (\te{Arith} class div method)}
\index{abs@{\te{abs}} (\te{Arith} class  method)}
\index{signum@{\te{signum}} (\te{Arith} class  method)}
\index{**@{\te{**}} (\te{Arith} class  method)}
\index{exp\_e@{\te{exp\_e}} (\te{Arith} class  method)}
\index{log@{\te{log}} (\te{Arith} class  method)}
\index{logb@{\te{logb}} (\te{Arith} class  method)}
\index{log2@{\te{log2}} (\te{Arith} class  method)}
\index{log10@{\te{log10}} (\te{Arith} class  method)}
\index[function]{Prelude!+}
\index[function]{Prelude!-}
\index[function]{Prelude!negate}
\index[function]{Prelude!*}
\index[function]{Prelude!\%}
\index[function]{Prelude!/}
\index[function]{Prelude!abs}
\index[function]{Prelude!signum}
\index[function]{Prelude!**}
\index[function]{Prelude!exp}
\index[function]{Prelude!log}
\index[function]{Prelude!logb}
\index[function]{Prelude!log2}
\index[function]{Prelude!log10}
\index[typeclass]{Arith}

\begin{libverbatim}
   typeclass Arith #(type data_t)
     provisos (Literal#(data_t));
       function data_t \+ (data_t x, data_t y);
       function data_t \- (data_t x, data_t y);
       function data_t negate (data_t x);
       function data_t \* (data_t x, data_t y);
       function data_t \/ (data_t x, data_t y);
       function data_t \% (data_t x, data_t y);
       function data_t abs (data_t x);
       function data_t signum (data_t x);
       function data_t \** (data_t x, data_t y);
       function data_t exp_e (data_t x);
       function data_t log (data_t x);
       function data_t logb (data_t b, data_t x);
       function data_t log2 (data_t x); 
       function data_t log10 (data_t x);      
   endtypeclass
\end{libverbatim}
The {\tt Arith} functions provide arithmetic operations.  
For the arithmetic symbols, when defining an instance of the
\te{Arith} typeclass, the escaped operator names must be used as
shown in the tables below.  The \te{negate} name may be used instead
of the operator for negation.   If using or referring to these
functions, the standard (non-escaped) {\V}
operators can be used.

\begin{center}
\begin{tabular}{|p{1 in}|p{4in}|}
\hline
\multicolumn{2}{|c|}{\te{Arith} Functions}\\
\hline
\hline
\te{+}&Element {\tt x} is added to element {\tt y}. \\
\cline{2-2}
&\begin{libverbatim}function data_t \+ (data_t x, data_t y);
\end{libverbatim}
\\
\hline
\end{tabular}
\end{center}
\begin{center}
\begin{tabular}{|p{1 in}|p{4in}|}
\hline
\te{-}&Element {\tt y} is subtracted from  element {\tt x}. \\
\cline{2-2}
&\begin{libverbatim}function data_t \- (data_t x, data_t y);
\end{libverbatim}
\\
\hline
\end{tabular}
\end{center}
\begin{center}
\begin{tabular}{|p{1 in}|p{4in}|}
\hline
\te{negate}&Change the sign of the number.  When using the function the {\V}
negate operator, {\tt -}, may be used.\\
{\tt -}& \\
\cline{2-2}
&\begin{libverbatim}function data_t negate (data_t x);
\end{libverbatim}
\\
\hline
\end{tabular}
\end{center}
\begin{center}
\begin{tabular}{|p{1 in}|p{4in}|}
\hline
\te{*}&Element {\tt x} is multiplied by  {\tt y}. \\
\cline{2-2}
&\begin{libverbatim}function data_t \* (data_t x, data_t y);
\end{libverbatim}
\\
\hline
\end{tabular}
\end{center}

\begin{center}
\begin{tabular}{|p{1 in}|p{4in}|}
\hline
\te{/}&Element {\tt x} is divided  by  {\tt y}. The definition depends
on the type - many types truncate the remainder . Note: may not be
synthesizable with downstream tools.  \\
\cline{2-2}
&\begin{libverbatim}function data_t \/ (data_t x, data_t y);
\end{libverbatim}
\\
\hline
\end{tabular}
\end{center}

\begin{center}
\begin{tabular}{|p{1 in}|p{4in}|}
\hline
\te{\%}&Returns the remainder of $x/y$.  Obeys the identity
$((x/y)*y) + (x\%y) = x$.\\
\cline{2-2}
&\begin{libverbatim}function data_t \% (data_t x, data_t y);
\end{libverbatim}
\\
\hline
\end{tabular}
\end{center}

Note: Division by 0 is undefined. Both $x/0$ and $x\%0$ will generate 
errors at compile-time and run-time for most instances.  


\begin{center}
\begin{tabular}{|p{1 in}|p{4in}|}
\hline
\te{abs}&Returns the absolute value of \te{x}.\\
\cline{2-2}
&\begin{libverbatim}function data_t abs (data_t x);
\end{libverbatim}
\\
\hline
\end{tabular}
\end{center}

\begin{center}
\begin{tabular}{|p{1 in}|p{4in}|}
\hline
\te{signum}&Returns a unit value with the same sign as \te{x}, such that
\te{abs(x)*signum(x) = x}.  \te{signum(12)} returns \te{1} and
\te{signum(-12)} returns \te{-1}.   \\
\cline{2-2}
&\begin{libverbatim}function data_t signum (data_t x);
\end{libverbatim}
\\
\hline
\end{tabular}
\end{center}


\begin{center}
\begin{tabular}{|p{1 in}|p{4in}|}
\hline
\te{**}&The element \te{x} is raised to the \te{y} power (\te{x**y} = $x^y$).\\
\cline{2-2}
&\begin{libverbatim}function data_t \** (data_t x, data_t y);
\end{libverbatim}
\\
\hline
\end{tabular}
\end{center}

\begin{center}
\begin{tabular}{|p{1 in}|p{4in}|}
\hline
\te{log2}&Returns the base \te{2} logarithm of \te{x} ($\log{_2} x$).\\
\cline{2-2}
&\begin{libverbatim}
function data_t log2(data_t x) ;
\end{libverbatim}
\\
\hline
\end{tabular}
\end{center}


\begin{center}
\begin{tabular}{|p{1 in}|p{4in}|}
\hline
\te{exp\_e}&\te{e} is raised to the power of \te{x} ($e^x$).   \\
\cline{2-2}
&\begin{libverbatim}function data_t exp_e (data_t x);
\end{libverbatim}
\\
\hline
\end{tabular}
\end{center}


\begin{center}
\begin{tabular}{|p{1 in}|p{4in}|}
\hline
\te{log}&Returns the base \te{e} logarithm of \te{x} ($\log{_e} x$).\\
\cline{2-2}
&\begin{libverbatim}function data_t log (data_t x);
\end{libverbatim}
\\
\hline
\end{tabular}
\end{center}

\begin{center}
\begin{tabular}{|p{1 in}|p{4in}|}
\hline
\te{logb}&Returns the base \te{b} logarithm of \te{x} ($\log{_b} x$).\\
\cline{2-2}
&\begin{libverbatim}function data_t logb (data_t b, data_t x);
\end{libverbatim}
\\
\hline
\end{tabular}
\end{center}



\begin{center}
\begin{tabular}{|p{1 in}|p{4in}|}
\hline
\te{log10}&Returns the base 10 logarithm of x ($\log{_{10}} x$).\\
\cline{2-2}
&\begin{libverbatim}
function data_t log10(data_t x) ;
\end{libverbatim}
\\
\hline
\end{tabular}
\end{center}


{\bf Examples}

\begin{verbatim}
    real u = log(1);
    real x = 128.0;
    real y = log2(x);
    real z = 100.0;
    real v = log10(z);
    real w = logb(3,9.0);
    real a = -x;
    real b = abs(x);
\end{verbatim}


\com{Ord}
\subsubsection{Ord}
\label{sec-ord}

\te{Ord} defines the class of types for which an \emph{order} is
defined,  allowing comparison operations.  A complete  definition of an
instance of  \te{Ord} requires defining  either \te{<=}
or \te{compare}. 


\index{Ord@\te{Ord} (type class)}
\index{<@{\te{<}} (\te{Ord} class method)}
\index{<=@{\te{<=}} (\te{Ord} class method)}
\index{>@{\te{>}} (\te{Ord} class method)}
\index{>=@{\te{>=}} (\te{Ord} class method)}
\index[function]{Prelude!<}
\index[function]{Prelude!>}
\index[function]{Prelude!<=}
\index[function]{Prelude!>=}
\index[typeclass]{Ord}
\index[function]{Prelude!compare}
\index[function]{Prelude!min}
\index[function]{Prelude!max}
\index{compare@\te{compare} (\te{Ord} class method)}
\index{min@\te{min} (\te{Ord} class method)}
\index{max@\te{max} (\te{Ord} class method)}


\begin{libverbatim}
   typeclass Ord #(type data_t);
       function Bool \<  (data_t x, data_t y);
       function Bool \<= (data_t x, data_t y);
       function Bool \>  (data_t x, data_t y);
       function Bool \>= (data_t x, data_t y);
       function Ordering compare(data_t x, data_t y); 
       function data_t min(data_t x, data_t y);
       function data_t max(data_t x, data_t y);
   endtypeclass
\end{libverbatim}


The functions \te{<}, \te{<=}, \te{>}, and \te{>=} are Boolean functions which return a
value of {\tt True} if the comparison condition is met.  

\begin{center}
\begin{tabular}{|p{1 in}|p{4in}|}
\hline
\multicolumn{2}{|c|}{\te{Ord} Functions}\\
\hline
\hline
\te{<}&Returns {\tt True} if {\tt x} is less than {\tt y}.\\
\cline{2-2}
&\begin{libverbatim}function Bool \<  (data_t x, data_t y);
\end{libverbatim}
\\
\hline
\end{tabular}
\end{center}
\begin{center}
\begin{tabular}{|p{1 in}|p{4in}|}
\hline
\te{<=}&Returns {\tt True} if {\tt x} is less than or equal to {\tt y}.\\
\cline{2-2}
&\begin{libverbatim}function Bool \<=  (data_t x, data_t y);
\end{libverbatim}
\\
\hline
\end{tabular}
\end{center}
\begin{center}
\begin{tabular}{|p{1 in}|p{4in}|}
\hline
\te{>}&Returns {\tt True} if {\tt x} is greater than {\tt y}.\\
\cline{2-2}
&\begin{libverbatim}function Bool \>  (data_t x, data_t y);
\end{libverbatim}
\\
\hline
\end{tabular}
\end{center}
\begin{center}
\begin{tabular}{|p{1 in}|p{4in}|}
\hline
\te{>=}&Returns {\tt True} if {\tt x} is greater than or equal to {\tt y}.\\
\cline{2-2}
&\begin{libverbatim}function Bool \>=  (data_t x, data_t y);
\end{libverbatim}
\\
\hline
\end{tabular}
\end{center}

The function \te{compare} returns a value of the \te{Ordering}
(Section \ref{sec-ordering}) data type (\te{LT}, \te{GT}, or \te{EQ}). 

\begin{center}
\begin{tabular}{|p{1 in}|p{4in}|}
\hline
\te{compare}&Returns the \te{Ordering}  value describing the
relationship of  \te{x} to \te{y}.\\
\cline{2-2}
&\begin{libverbatim}
function Ordering compare (data_t x, data_t y);
\end{libverbatim}
\\
\hline
\end{tabular}
\end{center}


The functions \te{min} and \te{max} return a value of datatype
\te{data\_t} which is either the minimum or maximum of the two values,
depending on the function. 
\begin{center}
\begin{tabular}{|p{1 in}|p{4in}|}
\hline
\te{min}&Returns the minimum of the  values \te{x} and \te{y}.\\
\cline{2-2}
&\begin{libverbatim}
function data_t min (data_t x, data_t y);
\end{libverbatim}
\\
\hline
\end{tabular}
\end{center}

\begin{center}
\begin{tabular}{|p{1 in}|p{4in}|}
\hline
\te{max}&Returns the maximum of the values \te{x} and \te{y}.\\
\cline{2-2}
&\begin{libverbatim}
function data_t max (data_t x, data_t y);
\end{libverbatim}
\\
\hline
\end{tabular}
\end{center}

{\bf Examples}

\begin{verbatim}
   rule r1 (x <= y);
   
   rule r2 (x > y);

   function Ordering onKey(Record r1, Record r2);
      return compare(r1.key,r2.key);
   endfunction
   ...
      List#(Record) sorted_rs = sortBy(onKey,rs);
      List#(List#(Record)) grouped_rs = groupBy(equiv,sorted_rs);

   let read_count = min(reads_remaining, 16);
\end{verbatim}

\com{Bounded}
\subsubsection{Bounded}

\te{Bounded} defines the class of types with a finite range and
provides functions to define the range.
\index{Bounded@\te{Bounded} (type class)}
\index{minBound@\te{minBound} (\te{Bounded} class method)}
\index{maxBound@\te{maxBound} (\te{Bounded} class method)}
\index[function]{Prelude!minBound}
\index[function]{Prelude!maxBound}
\index[typeclass]{Bounded}

\begin{libverbatim}
   typeclass Bounded #(type data_t);
       data_t minBound;
       data_t maxBound;
   endtypeclass
\end{libverbatim}
The {\tt Bounded} functions {\tt minBound} and {\tt maxBound} define
the minimum and maximum values for the type {\tt data\_t}.  
Instances of the \te{Bounded} class are often automatically derived
using the \te{deriving} statement.


\begin{center}
\begin{tabular}{|p{1 in}|p{4in}|}
\hline
\multicolumn{2}{|c|}{\te{Bounded} Functions}\\
\hline
\hline
\te{minBound}&The minimum value the type {\tt data\_t} can have.\\
\cline{2-2}
&\begin{libverbatim}data_t minBound;
\end{libverbatim}
\\
\hline
\end{tabular}
\end{center}
\begin{center}
\begin{tabular}{|p{1 in}|p{4in}|}
\hline
\te{maxBound}&The maximum value the type {\tt data\_t} can have.\\
\cline{2-2}
&\begin{libverbatim}data_t maxBound;
\end{libverbatim}
\\
\hline
\end{tabular}
\end{center}

{\bf Examples}

\begin{verbatim}
   module mkGenericRandomizer (Randomize#(a))
      provisos (Bits#(a, sa), Bounded#(a));

   typedef struct {
       Bit#(2) red;
       Bit#(1) blue;
   } RgbColor deriving (Eq, Bits, Bounded);
\end{verbatim}


\com{Bitwise}
\subsubsection{Bitwise}

\te{Bitwise} defines the class of types on which bitwise operations
    are defined.

\index{Bitwise@\te{Bitwise} (type class)}
\index{\&@\te{\&} (\te{Bitwise} class method)}
\index{$\mid$ (\te{Bitwise} class method)}
%\index{^@\te{\^} (\te{Bitwise} class method)}
%\index{~^@{\te{~\^}} (\te{Bitwise} class method)}
%\index{~@\te{~} (\te{Bitwise} class method)}
%\index{^~@{\te{\^~}} (\te{Bitwise} class method)}
\index{{\Caret}@\te{\Caret} (\te{Bitwise} class method)}
\index{{\Tilde}{\Caret} (\te{Bitwise} class method)}
\index{{\Caret}{\Tilde} (\te{Bitwise} class method)}
\index{{\Tilde} (\te{Bitwise} class method)}
\index{invert@\te{invert} (\te{Bitwise} class method)}

\index[function]{Prelude!\&}
\index[function]{Prelude!$\mid$}
\index[function]{Prelude!{\Caret}}
\index[function]{Prelude!{\Tilde}}
\index[function]{Prelude!{\Caret}{\Tilde}}
\index[function]{Prelude!{\Tilde}{\Caret}}
\index[function]{Prelude!invert}
\index[typeclass]{Bitwise}

\begin{libverbatim}
   typeclass Bitwise #(type data_t);
       function data_t \& (data_t x1, data_t x2);
       function data_t \| (data_t x1, data_t x2);
       function data_t \^ (data_t x1, data_t x2);
       function data_t \~^ (data_t x1, data_t x2);
       function data_t \^~ (data_t x1, data_t x2);
       function data_t invert (data_t x1);
       function data_t \<< (data_t x1, x2);
       function data_t \>> (data_t x1, x2);
       function Bit#(1) msb (data_t x);
       function Bit#(1) lsb (data_t x);
   endtypeclass
\end{libverbatim}

The {\tt Bitwise} functions compare two operands bit by bit to
    calculate a result.  That is, the  bit in the first operand is compared
    to its equivalent bit in the second operand to calculate a single
    bit for
    the result.
\begin{center}
\begin{tabular}{|p{1 in}|p{4in}|}
\hline
\multicolumn{2}{|c|}{\te{Bitwise} Functions}\\
\hline
\hline
\verb'&'&Performs an {\em and} operation on  each bit in {\tt x1} and
    {\tt x2} to calculate the result.  \\
\cline{2-2}
&\begin{libverbatim}function data_t \& (data_t x1, data_t x2);
\end{libverbatim}
\\
\hline
\end{tabular}
\end{center}
\begin{center}
\begin{tabular}{|p{1 in}|p{4in}|}
\hline
\verb'|' &Performs an {\em or} operation on  each bit in {\tt x1} and
    {\tt x2} to calculate the result.  \\
\cline{2-2}
&\begin{libverbatim} function data_t \| (data_t x1, data_t x2);
\end{libverbatim}
\\
\hline
\end{tabular}
\end{center}
\begin{center}
\begin{tabular}{|p{1 in}|p{4in}|}
\hline
\verb'^' &Performs an {\em exclusive or} operation on  each bit in {\tt x1} and
    {\tt x2} to calculate the result.  \\
\cline{2-2}
&\begin{libverbatim}function data_t \^ (data_t x1, data_t x2);
\end{libverbatim}
\\
\hline
\end{tabular}
\end{center}
\begin{center}
\begin{tabular}{|p{1 in}|p{4in}|}
\hline
\verb'~^' &Performs an {\em exclusive nor} operation on  each bit in {\tt x1} and
    {\tt x2} to calculate the result. \\
\verb'^~'&\\
\cline{2-2}
&\begin{libverbatim}
function data_t \~^ (data_t x1, data_t x2);
function data_t \^~ (data_t x1, data_t x2);
\end{libverbatim}
\\
\hline
\end{tabular}
\end{center}
\begin{center}
\begin{tabular}{|p{1 in}|p{4in}|}
\hline
\verb|~| &Performs a {\em unary negation} operation on each bit in
{\tt x1}.  When using this function, the corresponding {\V} operator, \verb|~|, may be used.\\
{\tt invert}& \\
\cline{2-2}
&\begin{libverbatim}function data_t invert (data_t x1);
\end{libverbatim}
\\
\hline
\end{tabular}
\end{center}

\index{$<<$ (\te{Bitwise} class method)}
\index{$>>$@\te{$>>$}  (\te{Bitwise} class method)}
\index[function]{Prelude!$<<$}
\index[function]{Prelude!$>>$}

The \verb|<<| and \verb|>>| operators perform left and right shift
operations.  Whether the shift is an arithmetic shift (\te{Int}) or a
logical shift (\te{Bit}, \te{UInt}) is dependent on how the type is defined. 


\begin{center}
\begin{tabular}{|p{1 in}|p{4in}|}
\hline
\verb|<<|  &Performs a {\em left shift} operation of {\tt x1} by the number
of bit positions given by {\tt x2}.  \te{x2} must be of an
acceptable index type 
(\te{Integer}, \te{Bit\#(n)}, \te{Int\#(n)} or \te{UInt\#(n)}).\\
\cline{2-2}
&\begin{libverbatim}function data_t \<< (data_t x1, x2);
\end{libverbatim}
\\
\hline
\end{tabular}
\end{center}

\begin{center}
\begin{tabular}{|p{1 in}|p{4in}|}
\hline
\verb|>>|  &Performs a {\em right shift} operation of {\tt x1} by the number
of bit positions given by {\tt x2}.  \te{x2} must be of an
acceptable index type 
(\te{Integer}, \te{Bit\#(n)}, \te{Int\#(n)} or \te{UInt\#(n)}).\\
\cline{2-2}
&\begin{libverbatim}function data_t \>> (data_t x1, x2);
\end{libverbatim}
\\
\hline
\end{tabular}
\end{center}

The functions \te{msb} and \te{lsb} operate on a single argument.

\index{msb@\te{msb} (\te{Bitwise} class method)}
\index[function]{Prelude!msb}
\begin{center}
\begin{tabular}{|p{1 in}|p{4in}|}
\hline
\te{msb} &Returns the value of the  most significant bit of \te{x}. Returns 0 if
width of data\_t is 0.\\
\cline{2-2}
&\begin{libverbatim}function Bit#(1) msb (data_t x);
\end{libverbatim}
\\
\hline
\end{tabular}
\end{center}

\index{lsb@\te{lsb} (\te{Bitwise} class method)}
\index[function]{Prelude!lsb}
\begin{center}
\begin{tabular}{|p{1 in}|p{4in}|}
\hline
\te{lsb} &Returns the value of the least significant bit of \te{x}. Returns 0 if
width of data\_t is 0.\\
\cline{2-2}
&\begin{libverbatim}function Bit#(1) lsb (data_t x);
\end{libverbatim}
\\
\hline
\end{tabular}
\end{center}


{\bf Examples}

\begin{verbatim}
   function Value computeOp(AOp aop, Value v1, Value v2) ;
       case (aop) matches
           Aand : return v1 & v2;
           Anor : return invert(v1 | v2);
           Aor : return v1 | v2;
           Axor : return v1 ^ v2;
           Asll : return v1 << vToNat(v2);
           Asrl : return v1 >> vToNat(v2);
       endcase
   endfunction: computeOp

   Bit#(3) msb = read_counter [5:3];
   Bit#(3) lsb = read_counter [2:0];
   read_counter <= (msb == 3'b111) ? {msb+1,lsb+1} : {msb+1,lsb};
\end{verbatim}

\com{BitReduction}

\subsubsection{BitReduction}

\te{BitReduction} defines the class of types on which the {\V} bit
reduction operations are defined.  
    

\index{BitReduction@\te{BitReduction} (type class)}
\index{reduceAnd@\te{reduceAnd} (\te{BitReduction} class method)}
\index{reduceOr@\te{reduceOr} (\te{BitReduction} class method)}
\index{reduceXor@\te{reduceXor} (\te{BitReduction} class method)}
\index{reduceNand@\te{reduceNand} (\te{BitReduction} class method)}
\index{reduceNor@\te{reduceNor} (\te{BitReduction} class method)}
\index{reduceXnor@\te{reduceXnor} (\te{BitReduction} class method)}
\index{\&@\te{\&} (\te{BitReduction} class method)}
\index{$\mid$ (\te{BitReduction} class method)}
%\index{^@\te{\^} (\te{BitReduction} class method)}
\index{{\Caret}@\te{\Caret} (\te{BitReduction} class method)}
%\index{^&@\te{\^\&} (\te{BitReduction} class method)}
\index{{\Caret}\&@\te{\Caret\&} (\te{BitReduction} class method)}
%\index{~^@\te{~\^} (\te{BitReduction} class method)}
\index{{\Tilde\Caret}@\te{\Tilde\Caret} (\te{BitReduction} class method)}
%\index{~|@\te{~|} (\te{BitReduction} class method)}
\index{{\Tilde}$\mid$@\te{\Tilde$\mid$} (\te{BitReduction} class method)}
%\index{^~@\te{\^~} (\te{BitReduction} class method)}
\index{{\Caret\Tilde}@\te{\Caret\Tilde} (\te{BitReduction} class method)}
\index[function]{Prelude!\&}
\index[function]{Prelude!$\mid$}
\index[function]{Prelude!{\Caret}}
\index[function]{Prelude!{\Tilde}}
\index[function]{Prelude!{\Caret}{\Tilde}}
\index[function]{Prelude!{\Tilde}{\Caret}}
\index[function]{Prelude!{\Tilde}{$\mid$}}
\index[function]{Prelude!reduceAnd}
\index[function]{Prelude!reduceOr}
\index[function]{Prelude!reduceXor}
\index[function]{Prelude!reduceNand}
\index[function]{Prelude!reduceNor}
\index[function]{Prelude!reduceXNor}
\index[typeclass]{BitReduction}


\begin{libverbatim}
   typeclass BitReduction #(type x, numeric type n)
       function x#(1) reduceAnd (x#(n) d);
       function x#(1) reduceOr (x#(n) d);
       function x#(1) reduceXor (x#(n) d);
       function x#(1) reduceNand (x#(n) d);
       function x#(1) reduceNor (x#(n) d);
       function x#(1) reduceXnor (x#(n) d);
   endtypeclass
\end{libverbatim}
Note: the numeric keyword is not required

The {\tt BitReduction} functions take a sized type and reduce it to
one element.  The most common example is to operate on  a \te{Bit\#()}
to produce a single bit result.  The first step of the operation applies the operator
    between the first bit of the operand and the second bit of the
    operand to produce a result.  The function then applies the operator between the
    result and the next bit of the operand, until the final bit is processed.

Typically the bit reduction operators will be accessed through their
{\V} operators.  When defining a new instance of the \te{BitReduction} type class the
{\BSV} names must be used. The table below lists both values.  For
example,  the
{\BSV} bit reduction \emph{and} operator is \te{reduceAnd} and the
corresponding {\V} operator is \te{\&}.
\begin{center}
\begin{tabular}{|p{1 in}|p{4in}|}
\hline
\multicolumn{2}{|c|}{\te{BitReduction} Functions}\\
\hline
\hline
\te{reduceAnd}&Performs an {\em and} bit reduction operation between the elements
    of  {\tt d}  to calculate the result.  \\
\verb'&'&\\
\cline{2-2}
&\begin{libverbatim}function x#(1) reduceAnd (x#(n) d);
\end{libverbatim}
\\
\hline
\end{tabular}
\end{center}
\begin{center}
\begin{tabular}{|p{1 in}|p{4in}|}
\hline
\te{reduceOr}&Performs an {\em or} bit reduction operation between the elements
    of  {\tt d}  to calculate the result. \\
\verb'|'&\\
\cline{2-2}
&\begin{libverbatim}function x#(1) reduceOr (x#(n) d);
\end{libverbatim}
\\
\hline
\end{tabular}
\end{center}
\begin{center}
\begin{tabular}{|p{1 in}|p{4in}|}
\hline
\te{reduceXor}&Performs an {\em xor} bit reduction operation between the elements
    of  {\tt d}  to calculate the result.  \\
\verb'^'&\\
\cline{2-2}
&\begin{libverbatim}function x#(1) reduceXor (x#(n) d);
\end{libverbatim}
\\
\hline
\end{tabular}
\end{center}
\begin{center}
\begin{tabular}{|p{1 in}|p{4in}|}
\hline
\te{reduceNand}&Performs an {\em nand} bit reduction operation between the elements
    of  {\tt d}  to calculate the result.  \\
\verb'^&'&\\
\cline{2-2}
&\begin{libverbatim}function x#(1) reduceNand (x#(n) d);
\end{libverbatim}
\\
\hline
\end{tabular}
\end{center}
\begin{center}
\begin{tabular}{|p{1 in}|p{4in}|}
\hline
\te{reduceNor}&Performs an {\em nor} bit reduction operation between the elements
    of  {\tt d}  to calculate the result. \\
\verb'~|'&\\
\cline{2-2}
&\begin{libverbatim}function x#(1) reduceNor (x#(n) d);
\end{libverbatim}
\\
\hline
\end{tabular}
\end{center}
\begin{center}
\begin{tabular}{|p{1 in}|p{4in}|}
\hline
\te{reduceXnor}&Performs an {\em xnor} bit reduction operation between the elements
    of  {\tt d}  to calculate the result. \\
\verb'~^'&\\
\verb'^~'&\\
\cline{2-2}
&\begin{libverbatim}function x#(1) reduceXnor (x#(n) d);
\end{libverbatim}
\\
\hline
\end{tabular}
\end{center}

\com{BitExtend}
\subsubsection{BitExtend}
\index{BitExtend@\te{BitExtend} (type class)}
\index{extend@\te{extend} (\te{BitExtend} class method)}
\index{zeroExtend@\te{zeroExtend} (\te{BitExtend} class method)}
\index{signExtend@\te{signExtend} (\te{BitExtend} class method)}
\index{truncate@\te{truncate} (\te{BitExtend} class method)}
\index[function]{Prelude!extend}
\index[function]{Prelude!zeroExtend}
\index[function]{Prelude!signExtend}
\index[function]{Prelude!truncate}
\index[typeclass]{BitExtend}

\te{BitExtend} defines types on which bit extension operations are
    defined.  
\begin{verbatim}
   typeclass BitExtend #(numeric type m, numeric type n, type x);  // n > m
       function x#(n) extend (x#(m) d);
       function x#(n) zeroExtend (x#(m) d);  
       function x#(n) signExtend (x#(m) d);
       function x#(m) truncate (x#(n) d);
   endtypeclass
\end{verbatim}

The \te{BitExtend} operations take as input  of one size and changes
    it to an input of another size, as described in the tables below.
    It is recommended that \te{extend} be used in place of
    \te{zeroExtend} or \te{signExtend}, as it will automatically
    perform the correct operation based on the data type of the argument.

\begin{center}
\begin{tabular}{|p{1 in}|p{4in}|}
\hline
\multicolumn{2}{|c|}{\te{BitExtend} Functions}\\
\hline
\hline
\te{extend}&Performs either a zeroExtend or a signExtend as
appropriate, depending on the data type of the argument (zeroExtend
for an unsigned argument, signExtend for a signed argument).\\
\cline{2-2}
&\begin{libverbatim}
function x#(n) extend (x#(m) d)
   provisos (Add#(k, m, n));
\end{libverbatim}
\\
\hline
\end{tabular}
\end{center}
\begin{center}
\begin{tabular}{|p{1 in}|p{4in}|}
\hline
\te{zeroExtend}& Use of \te{extend} instead is recommended. Adds extra zero bits to the MSB of argument {\tt d} of size {\tt
    m} to make the datatype size {\tt n}.\\
\cline{2-2}
&\begin{libverbatim}
function x#(n) zeroExtend (x#(m) d)
   provisos (Add#(k, m, n));
\end{libverbatim}
\\
\hline
\end{tabular}
\end{center}
\begin{center}
\begin{tabular}{|p{1 in}|p{4in}|}
\hline
\te{signExtend}& Use of \te{extend} instead is recommended. Adds extra sign bits to
the MSB of argument  {\tt d} of size {\tt m} 
to make the datatype size {\tt n} by  replicating the sign bit.\\
\cline{2-2}
&\begin{libverbatim}
function x#(n) signExtend (x#(m) d)
   provisos (Add#(k, m, n));
\end{libverbatim}
\\
\hline
\end{tabular}
\end{center}
\begin{center}
\begin{tabular}{|p{1 in}|p{4in}|}
\hline
\te{truncate}&Removes bits from the MSB of argument {\tt d} of size
{\tt n} to make the datatype size {\tt m}.\\
\cline{2-2}
&\begin{libverbatim}
function x#(m) truncate (x#(n) d)
   provisos (Add#(k, n, m));
\end{libverbatim}
\\
\hline
\end{tabular}
\end{center}

{\bf Examples}
\begin{verbatim}
   UInt#(TAdd#(1,TLog#(n))) zz = extend(xx) + extend(yy);
   Bit#(n) v1 = zeroExtend(v);
   Int#(4) i_index = signExtend(i) + 4;
   Bit#(32) upp = truncate(din);
   r <= zeroExtend(c + truncate(r))
\end{verbatim}

\subsubsection{SaturatingArith}
\label{sec-saturatingarith}

\index{SaturatingArith@\te{SaturatingArith} (type class)}
\index{satPlus@\te{satPlus} (\te{SaturatingArith} class method)}
\index[function]{Prelude!satPlus}
\index{satMinus@\te{satMinus} (\te{SaturatingArith} class method)}
\index[function]{Prelude!satMinus}
\index[typeclass]{SaturatingArith}


The \te{SaturatingArith} typeclass  contains modified addition and
subtraction functions which saturate to the values defined by  maxBound or
minBound when the operation  would otherwise overflow or wrap-around.


There are 4 types of saturation modes which determine how an overflow
or underflow should be handled, as defined by the
\te{SaturationMode} type.  

\begin{center}
\begin{tabular}{|p{1in}|p{4.5 in}|}
\hline
\multicolumn{2}{|c|}{Saturation Modes}\\
\hline
Enum Value & Description\\
\hline\hline
\te{Sat\_Wrap}&Ignore overflow and underflow, just wrap around\\
\hline
\te{Sat\_Bound}&On overflow or underflow result becomes maxBound or minBound\\
\hline
\te{Sat\_Zero}&On overflow or underflow result becomes 0\\
\hline
\te{Sat\_Symmetric}&On overflow or underflow result becomes maxBound or
(minBound+1)\\
\hline
\end{tabular}
\end{center}

\begin{verbatim}
typedef enum { Sat_Wrap         
              ,Sat_Bound      
              ,Sat_Zero     
              ,Sat_Symmetric 
   } SaturationMode deriving (Bits, Eq);
\end{verbatim}


\begin{verbatim}
   typeclass SaturatingArith#( type t);
      function t satPlus (SaturationMode mode, t x, t y);
      function t satMinus (SaturationMode mode, t x, t y);
      function t boundedPlus  (t x, t y) = satPlus (Sat_Bound, x, y);
      function t boundedMinus (t x, t y) = satMinus(Sat_Bound, x, y);
   endtypeclass
\end{verbatim}

Instances of the \te{SaturatingArith} class  are defined for \te{Int},
\te{UInt}, \te{Complex}, and \te{FixedPoint}.




\begin{center}
\begin{tabular}{|p{.8 in}|p{4.8in}|}
\hline
\te{satPlus}&Modified plus function which saturates  when
the operation would otherwise overflow or wrap-around.  The saturation
value (\te{maxBound}, wrap, or 0) is determined by the value of
\te{mode}, the \te{SaturationMode}.\\
\cline{2-2}
&\begin{libverbatim}
function t satPlus (SaturationMode mode, t x, t y);
\end{libverbatim}
\\
\hline
\end{tabular}
\end{center}

\begin{center}
\begin{tabular}{|p{.8 in}|p{4.8in}|}
\hline
\te{satMinus}&Modified minus function which saturates when
the operation would otherwise overflow or wrap-around.    The saturation
value (\te{minBound}, wrap, \te{minBound +1}, or 0) is determined by
the value of 
\te{mode}, the \te{SaturationMode}.\\
\cline{2-2}
&\begin{libverbatim}
function t satMinus (SaturationMode mode, t x, t y);
\end{libverbatim}
\\
\hline
\end{tabular}
\end{center}

\begin{center}
\begin{tabular}{|p{.8 in}|p{4.8in}|}
\hline
\te{boundedPlus}&Modified plus function which saturates to \te{maxBound}  when
the operation would otherwise overflow or wrap-around.  The function
is the same as \te{satPlus} where the \te{SaturationMode} is \te{Sat\_Bound}.\\
\cline{2-2}
&\begin{libverbatim}
function t boundedPlus (t x, t y) = satPlus (Sat_Bound, x, y);
\end{libverbatim}
\\
\hline
\end{tabular}
\end{center}

\begin{center}
\begin{tabular}{|p{.8 in}|p{4.8in}|}
\hline
\te{boundedMinus}&Modified minus function which saturates to
\te{minBound} when
the operation would otherwise overflow or wrap-around.  The function
is the same as \te{satMinus} where the \te{SaturationMode} is \te{Sat\_Bound}. \\
\cline{2-2}
&\begin{libverbatim}
function t boundedMinus (t x, t y) = satMinus(Sat_Bound, x, y);
\end{libverbatim}
\\
\hline
\end{tabular}
\end{center}

{\bf Examples}

\begin{verbatim}
   Reg#(SaturationMode) smode <- mkReg(Sat_Wrap);

   rule okdata (isOk);
      tstCount <= boundedPlus (tstCount, 1);
   endrule


\end{verbatim}

\subsubsection{Alias and NumAlias}
\label{sec-alias}
\index{Alias}
\index{NumAlias}
\index[typeclass]{NumAlias}
\index[typeclass]{Alias}

\te{Alias} specifies that two types can be used interchangeably,
providing a way to introduce local names for types within a module.
They are used in Provisos.

\begin{verbatim}
typeclass Alias#(type a, type b)
   dependencies (a determines b,
                 b determines a);
endtypeclass
\end{verbatim}

\te{NumAlias} is used to give a new name to a numeric type.

\begin{verbatim}
typeclass NumAlias#(numeric type a, numeric type b)
   dependencies (a determines b,
                 b determines a);
endtypeclass
\end{verbatim}

{\bf Examples}
\begin{verbatim}
   Alias#(fp, FixedPoint#(i,f));
   NumAlias#(TLog#(a,b), logab);
\end{verbatim}
%===================================================

\subsubsection{FShow}
\index{FShow@\te{FShow} (type class)}
\index[typeclass]{FShow}
\index[function]{Prelude!fshow}

The \te{FShow} typeclass defines the types to which the function
\te{fshow} can be applied.  The function converts a  value to an associated \te{Fmt}
representation for use with  the \te{\$display} family of system
tasks.  Instances of the \te{FShow} class can often be automatically
derived using the \te{deriving} statement.

\begin{verbatim}
typeclass FShow#(type t);
   function Fmt fshow(t value);
endtypeclass
\end{verbatim}

\begin{center}
\begin{tabular}{|p{.8 in}|p{4.8in}|}
\hline
\multicolumn{2}{|c|}{FShow function}\\
\hline
\hline
\te{fshow}&Returns a \te{Fmt} representation when applied to a value\\
\cline{2-2}
&\begin{libverbatim}
   function Fmt fshow(t value);
\end{libverbatim}
\\
\hline
\end{tabular}
\end{center}



 Instances of \te{FShow} for \te{Prelude} data types are defined in the \te{Prelude}
package.    Instances  for non-Prelude types are documented in the
 type package.     If an instance of \te{FShow} is not already
defined for a type you can create your own instance.  You can also 
 redefine existing instances as required for your design.


\begin{center}
\begin{tabular}{|p{1.1 in}|p{1.4in}|p{1.8in}|p{1.2in}|}
\hline
\multicolumn{4}{|c|}{\te{FShow} Instances}\\
\hline
Type&Fmt Object&Description&Example\\
\hline
\hline
\te{String}, \te{Char}&value  &  value of the string&Hello\\
\hline
\te{Bool}& \te{True} & Bool values&True\\
&\te{False}&&False\\
\hline
\te{Int\#(n)}& \te{n} & n in decimal format&\multicolumn{1}{|r|}{\te{-17}}\\
\hline
\te{UInt\#(n)}&\te{n} & n in decimal format&\multicolumn{1}{|r|}{\te{42}}\\
\hline
\te{Bit\#(n)}&\te{'hn}& n in hex, prepended with \te{'h}&\te{'h43F2}\\
\hline
\te{Maybe\#(a)}&\te{tagged Valid value}&FShow applied to value&\te{tagged Valid 42} \\
&\te{tagged Invalid}&&\te{tagged Invalid}\\
\hline
\te{Tuple2\#(a,b)}&\te{< a, b>}&FShow applied to each value& < 0, 1> \\
\te{Tuple3\#(a,b,c)}&\te{< a, b, c>}&& < 0, 1, 2> \\
\te{Tuple4\#(a,b,c,d)}&\te{< a, b, c, d>}&& < 0, 1, 2, 3> \\
...&&&\\
\multicolumn{2}{|l|}{\te{Tuple8\#(a,b,c,d,e,f,g,h)}}&&\\
\hline
\end{tabular}
\end{center}



% \begin{center}
% \begin{tabular}{|p{1.1 in}|p{4.5in}|}
% \hline
% \multicolumn{2}{|c|}{\te{FShow} Instances}\\
% \hline
% \hline
% \te{String}&Returns a \te{Fmt} object showing the value of the string.\\
% \cline{2-2}
% &\begin{libverbatim} 
% instance FShow#(String);
% \end{libverbatim}
% \\
% \hline
% \end{tabular}
% \end{center}
% \begin{center}
% \begin{tabular}{|p{1.1 in}|p{4.5in}|}
% \hline
% \te{Bool}&Returns a \te{Fmt} object showing \te{True} or \te{False}.\\
% \cline{2-2}
% &\begin{libverbatim}  
% instance FShow#(Bool);
% \end{libverbatim}
% \\
% \hline
% \end{tabular}
% \end{center}
% \begin{center}
% \begin{tabular}{|p{1.1 in}|p{4.5in}|}
% \hline
% \te{Maybe\#(a)}&Returns a \te{Fmt} object showing \te{tagged Valid <valid>} or
%  \te{tagged Invalid}. The \te{Maybe} type is a derived tagged union.\\
% % \cline{2-2}
% % &\begin{libverbatim} 
% % instance FShow#(Maybe#(a))
% %    provisos(FShow#(a));
% % \end{libverbatim}
% % \\
% \hline
% \end{tabular}
% \end{center}
% \begin{center}
% \begin{tabular}{|p{1.1 in}|p{4.5in}|}
% \hline
% \te{Int\#(n)}&Returns a \te{Fmt} object showing \te{n} in a decimal format.\\
% \cline{2-2}
% &\begin{libverbatim} 
% instance FShow#(Int#(n));
% \end{libverbatim}
% \\
% \hline
% \end{tabular}
% \end{center}
% \begin{center}
% \begin{tabular}{|p{1.1 in}|p{4.5in}|}
% \hline
% \te{UInt\#(n)}&Returns a \te{Fmt} object showing \te{n} in a decimal format.\\
% \cline{2-2}
% &\begin{libverbatim} 
% instance FShow#(UInt#(n));
% \end{libverbatim}
% \\
% \hline
% \end{tabular}
% \end{center}
% \begin{center}
% \begin{tabular}{|p{1.1 in}|p{4.5in}|}
% \hline
% \te{Bit\#(n)}&Returns a \te{Fmt} object showing \te{'hn} in a
% hexadecimal format. The \te{'h} is prepended to the value of \te{n}. \\
% \cline{2-2}
% &\begin{libverbatim} 
% instance FShow#(Bit#(n));
% \end{libverbatim}
% \\
% \hline
% \end{tabular}
% \end{center}

% \begin{center}
% \begin{tabular}{|p{1.1 in}|p{4.5in}|}
% \hline
% \te{FixedPoint\#(i,f)}&Returns a \te{Fmt} object showing \te{FP int.frac} where \te{int} is the
% integer part and \te{frac} is the fractional part of the fixed point number.\\
% \cline{2-2}
% &\begin{libverbatim} 
% instance FShow#(FixedPoint#(i,f));
% \end{libverbatim}
% \\
% \hline
% \end{tabular}
% \end{center}


% \begin{center}
% \begin{tabular}{|p{1.1 in}|p{4.5in}|}
% \hline
% \te{Complex\#(a)}&Returns a \te{Fmt} object showing <\te{C x.rel,
% x.img}> where \te{x.rel} is the real and \te{x.img} is the imaginary
% part of  \te{x}.\\
% \cline{2-2}
% &\begin{libverbatim} 
% instance FShow#(Complex#(a))
%    provisos (FShow#(a));
% \end{libverbatim}
% \\
% \hline
% \end{tabular}
% \end{center}



{\bf Example }
\begin{verbatim}
typedef enum {READ, WRITE, UNKNOWN}  OpCommand   deriving(Bounded,Bits, Eq, FShow);

typedef struct {OpCommand command;
		Bit#(8)   addr;
		Bit#(8)   data;
		Bit#(8)   length;
		Bool      lock;
		} Header deriving (Eq, Bits, Bounded);

typedef union tagged {Header  Descriptor;
		      Bit#(8) Data;
		      } Request deriving(Eq, Bits, Bounded);

// Define FShow instances where definition is different
// than the derived values

instance FShow#(Header);
   function Fmt fshow (Header value);
      return ($format("<HEAD ")
	      +
	      fshow(value.command)
	      +
	      $format(" (%0d)", value.length)
	      +
	      $format(" A:%h",  value.addr)
	      +
	      $format(" D:%h>", value.data));
   endfunction
endinstance

instance FShow#(Request);
   function Fmt fshow (Request request);
      case (request) matches
         tagged Descriptor .a:
            return fshow(a);
         tagged Data .a:
            return $format("<DATA %h>", a);
      endcase
   endfunction
endinstance
\end{verbatim}

%===================================================

\subsubsection{StringLiteral}
\index{StringLiteral@\te{StringLiteral} (type class)}
\index[typeclass]{StringLiteral}
\index[function]{Prelude!fromString}

 \te{StringLiteral} defines the class of types which can be created
 from strings. 

\begin{libverbatim}
   typeclass StringLiteral #(type data_t);
       function data_t fromString(String x);
   endtypeclass
\end{libverbatim}

\begin{center}
\begin{tabular}{|p{1 in}|p{4in}|}
\hline
\multicolumn{2}{|c|}{\te{StringLiteral} Functions}\\
\hline
\hline
\te{fromString}&Converts an element {\tt x} of datatype {\tt String} into an
element of  data type \te{data\_t}\\
\cline{2-2}
&\begin{libverbatim} function data_t fromString(String x);
\end{libverbatim}
\\
\hline
\end{tabular}
\end{center}



%===================================================

% \subsubsection{Monad}
% \index{Monad@\te{Monad} (type class)}
% \index{bind@\te{bind} (\te{Monad} class method)}
% \index{return@\te{return} (\te{Monad} class method)}

% The \te{Monad} typeclass is an abstraction which allows different
% composition strategies and is useful for combining computations into more
% complex computations.  The definition and use of monads in BSV is the
% standard definition as used in other programming languages.  It is
% mostly only needed inside libraries and not something that users will
% be working with directly, so beyond  modules and actions, this is an
% advanced topic that isn't commonly used.

% The \te{Monad} instance for \te{Module} and \te{Action} are how
% statements in those blocks are composed, which is why you can use
% \te{replicateM} and \te{mapM} in those blocks.




% \begin{verbatim}
% typeclass Monad #(type m);
%   function m#(b) \bind(m#(a) fn1, function m#(b) fn2(a x));
%   function m#(a) \return(a x);
% endtypeclass
% \end{verbatim}


% \begin{center}
% \begin{tabular}{|p{1 in}|p{4in}|}
% \hline
% \multicolumn{2}{|c|}{\te{Monad} Functions}\\
% \hline
% \hline
% \te{bind}&  Combines two computations, passing the return
%   value from the first as the argument to the second. \\
% \cline{2-2}
% &\begin{libverbatim}
% function m#(b) bind(m#(a) fn1, function m#(b) fn2(a x));
% \end{libverbatim}
% \\
% \hline
% \end{tabular}
% \end{center}
% \begin{center}
% \begin{tabular}{|p{1 in}|p{4in}|}
% \hline
% \te{return}& Creates a computation that returns the given value. \\
% \cline{2-2}
% &\begin{libverbatim}
% function m#(a) return(a x);
% \end{libverbatim}
% \\
% \hline
% \end{tabular}
% \end{center}

% \te{MonadFix} 



% The class of monads that admit recursion.
% \index{MonadFix@\te{MonadFix} (type class)}
% \index{mfix@\te{mfix} (\te{MonadFix} class method)}
% \begin{libverbatim}
% typeclass MonadFix #(type m)
% \end{libverbatim}
% The \te{MonadFix} type class belongs to the \te{Monad} type class.
% \begin{libverbatim}
% instance Monad#(MonadFix (m))
% \end{libverbatim}

% \begin{center}
% \begin{tabular}{|p{1 in}|p{4in}|}
% \hline
% \multicolumn{2}{|c|}{\te{MonadFix} Functions}\\
% \hline
% \hline
% \te{mfix}&  \\
% \cline{2-2}
% &\begin{libverbatim}
% function m#(a) mfix(function m#(a) x1(a x1));
% \end{libverbatim}
% \\
% \hline
% \end{tabular}
% \end{center}
%================================================================

\subsection{Data Types}
\label{prelude-datatypes}


Every variable and every expression in {\BSV}  has a \emph{type}.
Prelude defines the data types which are always available.
An \te{instance} declaration defines a  data type as belonging to  a 
type class. 
Each data type may belong to one or more type classes; all functions, modules,
and  operators declared for the type class are then 
defined for the data type.   A data type does not have to belong to any
 type classes.

Data type identifiers must always begin with a capital letter.  There
are three exceptions; \te{bit}, \te{int}, and \te{real},  which are
predefined for backwards compatibility.

  % The following table summarizes which type classes each
% data type in Prelude belongs to.

% \begin{center}
% \begin{tabular}{|c|c|c|c|c|c|c|c|c|c|}
% \hline
% \multicolumn{10}{|c|}{}\\
% \multicolumn{10}{|c|}{Type Classes  by Data Types}\\
% \hline
% \hline
% &\te{Bits}&\te{Eq}&\te{Literal}&\te{Arith}&\te{Ord}&\te{Bounded}&\te{Bitwise}&\te{Bit}&\te{Bit}\\
% &&&&&&&&Reduction&Extend\\
% \hline
% \te{Bit}&$\surd$&$\surd$&$\surd$&$\surd$&$\surd$&$\surd$&$\surd$&$\surd$&$\surd$\\
% \hline
% \te{UInt}&$\surd$&$\surd$&$\surd$&$\surd$&$\surd$&$\surd$&$\surd$&$\surd$&$\surd$\\
% \hline
% \te{Int}&$\surd$&$\surd$&$\surd$&$\surd$&$\surd$&$\surd$&$\surd$&$\surd$&$\surd$\\
% \hline
% \te{Integer}&&$\surd$&$\surd$&$\surd$&$\surd$&&&&\\
% \hline
% \te{Bool}&$\surd$&$\surd$&&&&&&&\\
% \hline
% \te{String}&&$\surd$&$\surd$&$\surd$&&&&&\\
% \hline
% \te{Fmt}&&&$\surd$&$\surd$&&&&&\\
% \hline
% \te{Maybe}&$\surd$&$\surd$&&&&&&&\\
% \hline
% \te{Action}&&&&&&&&&\\
% \hline
% \te{Rules}&&&&&&&&&\\
% \hline
% \end{tabular}
% \end{center}





% %%%%%%%%%%%%%%%%%%%%%%%%%%%%%%%%

\com{Bit}
\subsubsection{Bit}
\index{Bit@\te{Bit} (type)}
\label{sec-bit}

To define a value of type Bit:
\begin{libverbatim}
   Bit#(type n);
\end{libverbatim}



\begin{center}
\begin{tabular}{|c|c|c|c|c|c|c|c|c|c|}
\hline
\multicolumn{10}{|c|}{Type Classes for \te{Bit}}\\
\hline
\hline
&\te{Bits}&\te{Eq}&\te{Literal}&\te{Arith}&\te{Ord}&\te{Bounded}&\te{Bitwise}&\te{Bit}&\te{Bit}\\
&&&&&&&&\te{Reduction}&\te{Extend}\\
\hline
\te{Bit}&$\surd$&$\surd$&$\surd$&$\surd$&$\surd$&$\surd$&$\surd$&$\surd$&$\surd$\\
\hline
\hline
\end{tabular}
\end{center}


\index{bit@\te{bit} (type)}
% \index{Nat@\te{Nat} (type)}

\begin{center}
\begin{tabular}{|p{1 in}|p{4in}|}
\hline
\multicolumn{2}{|c|}{\te{Bit} type aliases}\\
\hline
\hline
\te{bit}&The data type \te{bit} is defined as a single bit.
This  is a special case of \te{Bit}.\\
\cline{2-2}
&\begin{libverbatim}
typedef Bit#(1) bit;
\end{libverbatim}
\\
\hline
\end{tabular}
\end{center}
\begin{center}
\begin{tabular}{|p{1 in}|p{4in}|}
\hline
\end{tabular}
\end{center}

{\bf Examples}
\begin{verbatim}
   Bit#(32) a;  // like 'reg [31:] a'
   Bit#(1)  b;  // like 'reg a'
   bit      c;  // same as Bit#(1) c
\end{verbatim}

\index{bitconcat@\te{bitconcat} (\te{Bit} concatenation operator)}
\index{split@\te{split} (\te{Bit} function)}
\index{\{\}@\te{\{\}} (\te{Bit} concatenation operator)}
\index[function]{Prelude!split}

The {\tt Bit} data type provides functions  to concatenate and split bit-vectors.

\begin{center}
\begin{tabular}{|p{1 in}|p{4in}|}
\hline
\multicolumn{2}{|c|}{\te{Bit} Functions}\\
\hline
\hline
\verb'{x,y}'&Concatenate two bit vectors, {\tt x} of size {\tt n} and {\tt
y} of size {\tt m}  returning a
bit vector of size {\tt k}.  The {\V} operator \verb'{ }' is used.\\
\cline{2-2}
&\begin{libverbatim}
function Bit#(k) bitconcat(Bit#(n) x, Bit#(m) y)
  provisos (Add#(n, m, k));
\end{libverbatim}
\\
\hline
\end{tabular}
\end{center}
\begin{center}
\begin{tabular}{|p{1 in}|p{4in}|}
\hline
\te{split}&Split a bit vector into two bit vectors (higher-order
bits (n), lower-order bits (m)).\\
\cline{2-2}
&\begin{libverbatim}
function Tuple2 #(Bit#(n), Bit#(m)) split(Bit#(k) x)
  provisos (Add#(n, m, k));
\end{libverbatim}
\\
\hline
\end{tabular}
\end{center}
{\bf Examples}
\begin{verbatim}
   module mkBitConcatSelect ();
        Bit#(3) a = 3'b010;         //a = 010
        Bit#(7) b = 7'h5e;          //b = 1011110

        Bit#(10) abconcat = {a,b};  //  = 0101011110
        Bit#(4)  bselect  = b[6:3]; //  = 1011
   endmodule

   function Tuple2#(Bit#(m), Bit#(n)) split (Bit#(mn) xy)
       provisos (Add#(m,n,mn));
\end{verbatim}


% \index{zeroExtend@\te{zeroExtend} (\te{Bit} function)}
% \begin{libverbatim}
% function Bit#(m) zeroExtend(Bit#(n) x)
%   provisos (Add#(k, n, m));
% \end{libverbatim}


% \index{signExtend@\te{signExtend} (\te{Bit} function)}
% \begin{libverbatim}
% function Bit#(m) signExtend(Bit#(n) x)
%   provisos (Add#(k, n, m));
% \end{libverbatim}

% Truncate by discarding higher-order bits.
% \index{truncate@\te{truncate} (\te{Bit} function)}
% \begin{libverbatim}
% function Bit#(m) truncate(Bit#(n) x)
%   provisos (Add#(k, m, n));
% \end{libverbatim}

% Comparisons and shifts, interpreting as signed values.
% \index{signedLT@\te{signedLT} (\te{Bit} function)}
% \begin{libverbatim}
% function Bool signedLT(Bit#(n) x, Bit#(n) y);
% \end{libverbatim}


% \index{signedLE@\te{signedLE} (\te{Bit} function)}
% \begin{libverbatim}
% function Bool signedLE(Bit#(n) x, Bit#(n) y);
% \end{libverbatim}

% \index{signedGT@\te{signedGT} (\te{Bit} function)}
% \begin{libverbatim}
% function Bool signedGT(Bit#(n) x, Bit#(n) y);
% \end{libverbatim}

% \index{signedGE@\te{signedGE} (\te{Bit} function)}
% \begin{libverbatim}
% function Bool signedGE(Bit#(n) x, Bit#(n) y);
% \end{libverbatim}

% \index{signedShiftRight@\te{signedShiftRight} (\te{Bit} function)}
% \begin{libverbatim}
% function Bit#(n) signedShiftRight(Bit#(n) x,Nat  c);
% \end{libverbatim}


% %%%%%%%%%%%%%%%%%%%%%%%%%%%%%%%%

% \subsubsection{Empty}
% %@
% *****Check reference guide main text for this info *******

% An interface with no methods.
% \index{Empty@\te{Empty} (interface type)}
% \begin{libverbatim}
% interface Empty;
% endinterface: Empty
% \end{libverbatim}
% interface Empty = { }


% %%%%%%%%%%%%%%%%%%%%%%%%%%%%%%%%% 
\com{UInt}
\subsubsection{UInt}
\label{sec-uint}

The \te{UInt} type is an unsigned fixed width
representation  of an integer value.   

\index{UInt@\texttt{UInt} (type)}

\begin{center}
\begin{tabular}{|c|c|c|c|c|c|c|c|c|c|}
\hline
\multicolumn{10}{|c|}{Type Classes for \te{UInt}}\\
\hline
\hline
&\te{Bits}&\te{Eq}&\te{Literal}&\te{Arith}&\te{Ord}&\te{Bounded}&\te{Bitwise}&\te{Bit}&\te{Bit}\\
&&&&&&&&\te{Reduction}&\te{Extend}\\
\hline
\te{UInt}&$\surd$&$\surd$&$\surd$&$\surd$&$\surd$&$\surd$&$\surd$&$\surd$&$\surd$\\
\hline
\end{tabular}
\end{center}

{\bf Examples}
\begin{verbatim}
   UInt#(8)  a = 'h80;
   UInt#(12) b = zeroExtend(a); // b => 'h080
   UInt#(8)  c = truncate(b);   // c => 'h80
\end{verbatim}

% \te{UInt} belongs to the following typeclasses:
% \begin{libverbatim}
% instance Bits #(UInt#(k), k);
% instance Eq #(UInt#(n));
% instance Literal #(UInt#(n));
% instance Ord #(UInt#(n));
% instance Bounded #(UInt#(n));
% instance Bitwise #(UInt#(n));
% instance Bit\te{Reduction} #(UInt#(n));
% instance BitExtend #(UInt#(n));
% instance Arith #(UInt#(n));
% \end{libverbatim}


% %%%%%%%%%%%%%%%%%%%%%%%%%%%%%%%%% 


\com{Int}
\subsubsection{Int}
\label{sec-int}

The \te{Int} type is a signed fixed width representation of
an integer value.   

\index{Int@\texttt{Int} (type)}
\index{int@\texttt{int} (type)}

\begin{center}
\begin{tabular}{|c|c|c|c|c|c|c|c|c|c|}
\hline
\multicolumn{10}{|c|}{Type Classes for \te{Int}}\\
\hline
\hline
&\te{Bits}&\te{Eq}&\te{Literal}&\te{Arith}&\te{Ord}&\te{Bounded}&\te{Bitwise}&\te{Bit}&\te{Bit}\\
&&&&&&&&\te{Reduction}&\te{Extend}\\
\hline
\te{Int}&$\surd$&$\surd$&$\surd$&$\surd$&$\surd$&$\surd$&$\surd$&$\surd$&$\surd$\\
\hline
\end{tabular}
\end{center}

{\bf Examples}
\begin{verbatim}
   Int#(8)  a = 'h80;
   Int#(12) b = signExtend(a);   // b => 'hF80
   Int#(8)  c = truncate(d);     // c => 'h80
\end{verbatim}

% \te{Int} belongs to the following typeclasses:
% \begin{libverbatim}
% instance Bits #(Int#(k), k);
% instance Eq #(Int#(n));
% instance Literal #(Int#(n));
% instance Ord #(Int#(n));
% instance Bounded #(Int#(n));
% instance Bitwise #(Int#(n));
% instance BitReduction #(Int#(n));
% instance BitExtend #(Int#(n));
% instance Arith #(Int#(n));
% \end{libverbatim}



\begin{center}
\begin{tabular}{|p{1 in}|p{4in}|}
\hline
\multicolumn{2}{|c|}{\te{Int} type aliases}\\
\hline
\hline
\te{int}&The data type \te{int} is defined as a 32-bit signed integer.
This  is a special case of \te{Int}.\\
\cline{2-2}
&\begin{libverbatim}
typedef Int#(32) int;
\end{libverbatim}
\\
\hline
\end{tabular}
\end{center}


%%%%%%%%%%%%%%%%%%%%%%%%%%%%%%%%

\com{Integer}
\subsubsection{Integer}
\label{sec-integer}

The \te{Integer} type is a data type used for integer values and
functions.  Because
\te{Integer} is not part of the \te{Bits} typeclass, the
\te{Integer} type is used for static elaboration only; all values must be
resolved at compile time.

\index{Integer@\texttt{Integer} (type)}
\index{div@{\te{div}} (\te{Integer} function)}
\index{mod@{\te{mod}} (\te{Integer} function)}
%\index{exp@{\te{exp}} (\te{Integer} function)}
%\index{log2@{\te{log2}} (\te{Integer} function)}
\index{quot@{\te{quot}} (\te{Integer} function)}
\index{rem@{\te{rem}} (\te{Integer} function)}
\index[function]{Prelude!div}
\index[function]{Prelude!mod}
%\index[function]{Prelude!exp}
%\index[function]{Prelude!log2}
\index[function]{Prelude!quot}
\index[function]{Prelude!rem}



\begin{center}
\begin{tabular}{|c|c|c|c|c|c|c|c|c|c|}
\hline
\multicolumn{10}{|c|}{Type Classes for \te{Integer}}\\
\hline
\hline
&\te{Bits}&\te{Eq}&\te{Literal}&\te{Arith}&\te{Ord}&\te{Bounded}&\te{Bitwise}&\te{Bit}&\te{Bit}\\
&&&&&&&&\te{Reduction}&\te{Extend}\\
\hline
\te{Integer}&&$\surd$&$\surd$&$\surd$&$\surd$&&&&\\
\hline
\end{tabular}
\end{center}

% \begin{libverbatim}
% instance Literal #(Integer);
% instance Eq #(Integer);
% instance Ord #(Integer);
% instance Arith #(Integer);
% \end{libverbatim}



\begin{center}
\begin{tabular}{|p{1 in}|p{4in}|}
\hline
\multicolumn{2}{|c|}{\te{Integer} Functions}\\
\hline
\hline
\te{div}&Element \te{x} is divided by element \te{y} and the result is
rounded toward negative infinity. Division by 0 is undefined.\\
\cline{2-2}
&\begin{libverbatim}
function Integer div(Integer x, Integer y);
\end{libverbatim}
\\
\hline
\end{tabular}
\end{center}
\begin{center}
\begin{tabular}{|p{1 in}|p{4in}|}
\hline
\te{mod}& Element \te{x} is divided by element \te{y} using the
\te{div} function  and the
remainder is returned as an \te{Integer} value. \te{div} and \te{mod}
satisfy the identity $(div(x,y) * y) + mod(x,y) == x$. Division by 0 is undefined.\\
\cline{2-2}
&\begin{libverbatim}
function Integer mod(Integer x, Integer y);
\end{libverbatim}
\\
\hline
\end{tabular}
\end{center}

\begin{center}
\begin{tabular}{|p{1 in}|p{4in}|}
\hline
\te{quot}& Element \te{x} is divided by element \te{y} and the result
 is truncated (rounded towards 0). Division by 0 is undefined.\\
\cline{2-2}
&\begin{libverbatim}
function Integer quot(Integer x, Integer y);
\end{libverbatim}
\\
\hline
\end{tabular}
\end{center}

\begin{center}
\begin{tabular}{|p{1 in}|p{4in}|}
\hline
\te{rem}& Element \te{x} is divided by element \te{y} using the
\te{quot} function and the
remainder is returned as an \te{Integer} value.  \te{quot} and
\te{rem} satisfy the identity $(quot(x,y) * y) + rem(x,y) == x$.  Division by 0 is undefined.\\
\cline{2-2}
&\begin{libverbatim}
function Integer rem(Integer x, Integer y);
\end{libverbatim}
\\
\hline
\end{tabular}
\end{center}

\index{fromInteger@\te{fromInteger} (\te{Literal} class method)}
\index[function]{Prelude!fromInteger}

The \te{fromInteger} function, defined in Section \ref{literal}, can be
used to convert an {\tt Integer} into any type in the \te{Literal} typeclass.

{\bf Examples}

\begin{verbatim}
   Int#(32) arr2[16];
   for (Integer i=0; i<16; i=i+1)
      arr2[i] = fromInteger(i);

   Integer foo = 10;
   foo = foo + 1;
   foo = foo * 5;
   Bit#(16) var1 = fromInteger( foo );
\end{verbatim}



% \begin{center}
% \begin{tabular}{|p{1 in}|p{4in}|}
% \hline
% \te{exp}& The element \emph{base} is raised to the \emph{pwr} power
% and the exponential value is returned as type \te{Integer}, \emph{exp}.\\
% \cline{2-2}
% &\begin{libverbatim}
% function Integer exp(Integer base, Integer pwr);
% \end{libverbatim}
% \\
% \hline
% \end{tabular}
% \end{center}
% \begin{center}
% \begin{tabular}{|p{1 in}|p{4in}|}
% \hline
% \te{log2}&Takes  the base 2 logarithm of the Integer \te{x} and
% returns the closest higher Integer value, if the result itself is not
% an Integer.\\
% \cline{2-2}
% &\begin{libverbatim}
% function Integer log2(Integer x) ;
% \end{libverbatim}
% \\
% \hline
% \end{tabular}
% \end{center}



% %%%%%%%%%%%%%%%%%%%%%%%%%%%%%%%%

\com{Bool}
\subsubsection{Bool}
\label{sec-bool}

\index{Bool@\te{Bool} (type)}
\index{True@\te{True} (\te{Bool} constant)}
\index{False@\te{False} (\te{Bool} constant)}
\index{not@\te{not} (\te{Bool} function)}
\index{!@\te{!} (\te{Bool} function)}
\index{||@\te{||} (\te{Bool} operator)}
\index{&&@\te{\&\&} (\te{Bool} operator)}
\index[function]{Prelude!not}


The {\tt Bool} type is defined to have two values, {\tt True} and {\tt
False}.
\begin{libverbatim}
typedef enum {False, True} Bool;
\end{libverbatim}

\begin{center}
\begin{tabular}{|c|c|c|c|c|c|c|c|c|c|}
\hline
\multicolumn{10}{|c|}{Type Classes for \te{Bool}}\\
\hline
\hline
&\te{Bits}&\te{Eq}&\te{Literal}&\te{Arith}&\te{Ord}&\te{Bounded}&\te{Bitwise}&\te{Bit}&\te{Bit}\\
&&&&&&&&\te{Reduction}&\te{Extend}\\
\hline
\te{Bool}&$\surd$&$\surd$&&&&&&&\\
\hline
\end{tabular}
\end{center}


% \te{Bool} is part of the following type classes:
% \begin{libverbatim}
% instance Bits #(Bool x);
% instance Eq #(Bool x);
% instance Bounded #(Bool x);
% \end{libverbatim}

%typedef enum {False, True} Bool deriving (Eq, Bits, Bounded);

The {\tt Bool} functions return either a value of {\tt True} or {\tt False}.
\begin{center}
\begin{tabular}{|p{1 in}|p{4in}|}
\hline
\multicolumn{2}{|c|}{\te{Bool} Functions}\\
\hline
\hline
\te{not}&Returns True if x is false, returns False if x is true.\\
\te{!}&\\
\cline{2-2}
&\begin{libverbatim}
function Bool not (Bool x);
\end{libverbatim}
\\
\hline
\end{tabular}
\end{center}
\begin{center}
\begin{tabular}{|p{1 in}|p{4in}|}
\hline
\te{\&\&}&Returns True if x \emph{and} y are true, else it returns False.\\
\cline{2-2}
&\begin{libverbatim}
function Bool \&& (Bool x, Bool y);
\end{libverbatim}
\\
\hline
\end{tabular}
\end{center}
\begin{center}
\begin{tabular}{|p{1 in}|p{4in}|}
\hline
\te{||} & Returns True if x \emph{or} y is true, else it returns False.\\
\cline{2-2}
&\begin{libverbatim}
function Bool \|| (Bool x, Bool y);
\end{libverbatim}
\\
\hline
\end{tabular}
\end{center}

{\bf Examples}
\begin{verbatim}
   Bool  a             // A variable named a with a type of Bool
   Reg#(Bool) done     // A register named done with a type of Bool
   Vector#(5, Bool)    // A vector of 5 Boolean values

   Bool  a = 0;     // ERROR!  You cannot do this
   Bool  a = True;  // correct
   if (a)           // correct
     ....
   if (a == 0)      // ERROR!
     ....

   Bool b1 = True;
   Bool b2 = True;
   Bool b3 = b1 && b2;
\end{verbatim}


%========================================
\com{Real}
\subsubsection{Real}
\index{Real@\te{Real} (type)}
\index{real@\te{real} (type)}
\label{sec-real}

The \te{Real} type is a data type used for real values and
functions. 

 Real numbers are of the form:
 
\gram{Real}{
\nterm{decNum}{\opt{\term{.}\nterm{decDigitsUnderscore}}} \nterm{exp}
\opt{\nterm{sign}} \nterm{decDigitsUnderscore}}\\
\gramalt{\nterm{decNum}\term{.}\nterm{decDigitsUnderscore}}

\gram{sign}{\term{+} {\alt} \term{-}}

\gram{exp}{\term{e} {\alt} \term{E}}

\gram{decNum} {\nterm{decDigits} \opt{\nterm{decDigitsUnderscore}}}

\gram{decDigits}{\many{\texttt{0}...\texttt{9}}} \\
\gram{decDigitsUnderscore} {\many{\texttt{0}...\texttt{9}, \te{\_}}}\\




If there is a decimal point,  there must be  digits following 
the decimal point.  An exponent can start with  either an \te{E} or
an \te{e}, followed by an optional 
sign (\te{+} or \te{-}), followed by digits.  There cannot be an
exponent or a sign without any digits.  Any of the numeric components
may include an underscore, but an underscore cannot be the first digit
of the real number.

Unlike integer numbers, real numbers are of limited precision.  They are
represented as IEEE floating point numbers of 64 bit length, as defined by
the IEEE standard.
 
Because the type \te{Real} is not part of the \te{Bits} typeclass, the
\te{Real} type is used for static elaboration only; all values must be
resolved at compile time.

There are many functions defined for \te{Real} types, provided in the \te{Real}
package (Section \ref{package-Real}).  To use these functions, the
\te{Real} package must be imported.

\begin{center}
\begin{tabular}{|c|c|c|c|c|c|c|c|c|c|c|}
\hline
\multicolumn{11}{|c|}{Type Classes for \te{Real}}\\
\hline
\hline
&\te{Bits}&\te{Eq}&\te{Literal}&\te{Real}&\te{Arith}&\te{Ord}&\te{Bounded}&\te{Bitwise}&\te{Bit}&\te{Bit}\\
&&&&\te{Literal}&&&&&\te{Reduction}&\te{Extend}\\
\hline
\te{Real}&&$\surd$&$\surd$&$\surd$&$\surd$&$\surd$&&&&\\
\hline
\end{tabular}
\end{center}

\begin{center}
\begin{tabular}{|p{1 in}|p{4in}|}
\hline
\multicolumn{2}{|c|}{\te{Real} type aliases}\\
\hline
\hline
\te{real}&The SystemVerilog name \te{real} is an alias for \te{Real}\\
\cline{2-2}
&\begin{libverbatim}
typedef Real real;
\end{libverbatim}
\\
\hline
\end{tabular}
\end{center}


There are two system tasks defined for the \te{Real} data type, used
to convert between \te{Real} and IEEE standard 64-bit vector
representation (\te{Bit\#(64))}.
\index{\$realtobits@\te{\$realtobits} (\te{Real} system function)}
\index{\$bitstoreal@\te{\$bitstoreal} (\te{Real} system function)}
\index[function]{Prelude!\$bitstoreal}
\index[function]{Prelude!\$realtobits}
\


\begin{center}
\begin{tabular}{|p{1 in}|p{4in}|}
\hline
\multicolumn{2}{|c|}{\te{Real} system tasks}\\
\hline
\hline
\te{\$realtobits}&Converts from a \te{Real} to the IEEE 64-bit vector representation. \\
\cline{2-2}
&\begin{libverbatim}
function Bit#(64) $realtobits (Real x) ;
\end{libverbatim}
\\
\hline
\end{tabular}
\end{center}
\begin{center}
\begin{tabular}{|p{1 in}|p{4in}|}
\hline
\te{\$bitstoreal}& Converts from a 64-bit vector representation to a \te{Real}. \\
\cline{2-2}
&\begin{libverbatim}
function Real $bitstoreal (Bit#(64) x) ;
\end{libverbatim}
\\
\hline
\end{tabular}
\end{center}


{\bf Examples}

\begin{verbatim}
   Bit#(1) sign1 = 0;
   Bit#(52) mantissa1 = 0;
   Bit#(11) exp1 = 1024;
   Real r1 = $bitstoreal({sign1, exp1, mantissa1});  //r1 = 2.0

   Real x = pi;
   let m = realToString(x));   // m = 3.141592653589793
\end{verbatim}

% %%%%%%%%%%%%%%%%%%%%%%%%%%%%%%%%

% \subsubsection{Either}
% \index{Either@\te{Either} (type)}
% %@
% Used for values that are either of type {\te a} or of type {\te b}.
% \begin{libverbatim}
% typedef tagged union {
%     a Left;
%     b Right;
% } Either #(type a, type b) deriving (Eq, Bits);
% \end{libverbatim}
% %%%%%%%%%%%%%%%%%%%%%%%%%%%%%%%%

\com{String}
\subsubsection{String}

\label{prelude-string}
\index{String@\te{String} (type)}
\index{strConcat@\te{strConcat} (\te{String} concatenation operator)}
\index{+@\te{+} (\te{String} concatenation operator)}
\index[function]{Prelude!strConcat}
\index[function]{Prelude!+}
\index[function]{Prelude!stringSplit}
\index[function]{Prelude!stringLength}
\index[function]{Prelude!quote}
\index[function]{Prelude!doubleQuote}
\index[function]{Prelude!stringHead}
\index[function]{Prelude!stringTail}
\index[function]{Prelude!stringCons}
\index[function]{Prelude!stringToCharList}
\index[function]{Prelude!sharListToString}


Strings are mostly used in system tasks (such as \te{\$display}).
The \te{String} type belongs to the \te{Eq} type class;
 strings
can be tested  for equality and inequality using the \te{==} and
\te{!=} operators.  The \te{String} type is also part of the
\te{Arith} class, but only the addition (\te{+}) operator is defined.
All other \te{Arith} operators will produce an error message. 

\begin{center}
\begin{tabular}{|c|c|c|c|c|c|c|c|c|c|}
\hline
\multicolumn{10}{|c|}{Type Classes for \te{String}}\\
\hline
\hline
&\te{Bits}&\te{Eq}&\te{Literal}&\te{Arith}&\te{Ord}&\te{Bounded}&\te{Bitwise}&\te{String}&\te{FShow}\\
&&&&&&&&\te{Literal}&\\
\hline
\te{String}&&$\surd$&&$\surd$&&&&$\surd$&$\surd$\\
\hline
\end{tabular}
\end{center}


% \begin{libverbatim}
% instance Eq #(String);
% \end{libverbatim}


\begin{center}
\begin{tabular}{|p{1.2 in}|p{4.2in}|}
\hline
\multicolumn{2}{|c|}{\te{String} Functions}\\
\hline
\hline
\te{strConcat}&Concatenates two strings (same as the + operator) \\
\te{+}&\\
\cline{2-2}
&\begin{libverbatim}
function String strConcat(String s1, String s2);
\end{libverbatim}
\\
\hline
\end{tabular}
\end{center}

\begin{center}
\begin{tabular}{|p{1.2 in}|p{4.2in}|}
\hline
\te{stringLength}& Returns the number of characters in a string \\
\cline{2-2}
&\begin{libverbatim}
function Integer stringLength (String s);
\end{libverbatim}
\\
\hline
\end{tabular}
\end{center}


\begin{center}
\begin{tabular}{|p{1.0 in}|p{4.4in}|}
\hline
\te{stringSplit}& If the string is empty, returns \te{Invalid}; otherwise
it returns \te{Valid} with the \te{Tuple} containing the first
character as the head and the rest of the string as the tail. \\
\cline{2-2}
&\begin{libverbatim}
function Maybe#(Tuple#2(Char, String)) stringSplit(String s);
\end{libverbatim}
\\
\hline
\end{tabular}
\end{center}

\begin{center}
\begin{tabular}{|p{1.2 in}|p{4.2in}|}
\hline
\te{stringHead}& Extracts the first character of a string; reports an error if
the string is empty. \\
\cline{2-2}
&\begin{libverbatim}
function Char stringHead(String s);
\end{libverbatim}
\\
\hline
\end{tabular}
\end{center}

\begin{center}
\begin{tabular}{|p{1.2 in}|p{4.2in}|}
\hline
\te{stringTail}& Extracts all the characters of a string after the
first; reports an error if
the string is  empty. \\
\cline{2-2}
&\begin{libverbatim}
function String stringTail(String s);
\end{libverbatim}
\\
\hline
\end{tabular}
\end{center}

\begin{center}
\begin{tabular}{|p{1.2 in}|p{4.2in}|}
\hline
\te{stringCons}& Adds a character to the front of a string.  This
function is the complement of \te{stringSplit}.\\
\cline{2-2}
&\begin{libverbatim}
function String stringCons(Char c, String s);
\end{libverbatim}
\\
\hline
\end{tabular}
\end{center}

\begin{center}
\begin{tabular}{|p{1.2 in}|p{4.2in}|}
\hline
\te{stringToCharList}& Converts a \te{String} to a \te{List} of characters \\
\cline{2-2}
&\begin{libverbatim}
function List#(Char) stringToCharList (String s);
\end{libverbatim}
\\
\hline
\end{tabular}
\end{center}

\begin{center}
\begin{tabular}{|p{1.2 in}|p{4.2in}|}
\hline
\te{charListToString}& Converts a \te{List} of characters to a \te{String}\\
\cline{2-2}
&\begin{libverbatim}
function String charListToString (List#(Char) cs);
\end{libverbatim}
\\
\hline
\end{tabular}
\end{center}

\begin{center}
\begin{tabular}{|p{1.2 in}|p{4.2in}|}
\hline
\te{quote}&Add single quotes around a string: \te{`str'} \\
\cline{2-2}
&\begin{libverbatim}
function String quote (String s);
\end{libverbatim}
\\
\hline
\end{tabular}
\end{center}

\begin{center}
\begin{tabular}{|p{1.2 in}|p{4.2in}|}
\hline
\te{doubleQuote}&Add double quotes around a string: \te{"str"} \\
\cline{2-2}
&\begin{libverbatim}
function String doubleQuote (String s);
\end{libverbatim}
\\
\hline
\end{tabular}
\end{center}




{\bf Examples}

\begin{verbatim}
   String s1 = "This is a test";
   $display("first string = ", s1);
      
   // we can use + to concatenate
   String s2 = s1 + " of concatenation";
   $display("Second string = ", s2);
\end{verbatim}


% %%%%%%%%%%%%%%%%%%%%%%%%%%%%%%%%

\subsubsection{Char}
\label{prelude-char}
\index{Char@\te{Char} (type)}
\index[function]{Prelude!charToString}
\index[function]{Prelude!charToInteger}
\index[function]{Prelude!integerToChar}
\index[function]{Prelude!isSpace}
\index[function]{Prelude!isLower}
\index[function]{Prelude!isUpper}
\index[function]{Prelude!isAlpha}
\index[function]{Prelude!isAlphaNum}
\index[function]{Prelude!isDigit}
\index[function]{Prelude!isOctDigit}
\index[function]{Prelude!isHexDigit}
\index[function]{Prelude!toUpper}
\index[function]{Prelude!toLower}
\index[function]{Prelude!digitToInteger}
\index[function]{Prelude!digitToBits}
\index[function]{Prelude!integerToDigit}
\index[function]{Prelude!bitsToDigit}
\index[function]{Prelude!hexDigitToInteger}
\index[function]{Prelude!hexDigitToBits}
\index[function]{Prelude!integerToHexDigit}
\index[function]{Prelude!bitsToHexDigit}

The \te{Char} data type is used mostly in system tasks (such as
\te{\$display}).  The \te{Char} type provides the ability to traverse
the characters of a string.  The \te{Char} type belongs to the \te{Eq} type
class; chars can be tested for equality and inequality using the
\te{==} and \te{!=} operators.  

The \te{Char} type belongs to the \te{Ord} type class.

\begin{center}
\begin{tabular}{|c|c|c|c|c|c|c|c|c|c|}
\hline
\multicolumn{10}{|c|}{Type Classes for \te{Char}}\\
\hline
\hline
&\te{Bits}&\te{Eq}&\te{Literal}&\te{Arith}&\te{Ord}&\te{Bounded}&\te{Bitwise}&\te{String}&\te{FShow}\\
&&&&&&&&\te{Literal}&\\
\hline
\te{Char}&&$\surd$&&&$\surd$&&&$\surd$&$\surd$\\
\hline
\end{tabular}
\end{center}


\begin{center}
\begin{tabular}{|p{1.2 in}|p{4in}|}
\hline
\multicolumn{2}{|c|}{\te{Char} Functions}\\
\hline
\hline
\te{charToString}& Convert a single character to a string\\
\cline{2-2}
&\begin{libverbatim}
function String charToString (Char c);
\end{libverbatim}
\\
\hline
\end{tabular}
\end{center}

\begin{center}
\begin{tabular}{|p{1.2 in}|p{4in}|}
\hline
\te{charToInteger}&Convert a character to its ASCII numeric value \\
\cline{2-2}
&\begin{libverbatim}
function Integer charToInteger (Char c);
\end{libverbatim}
\\
\hline
\end{tabular}
\end{center}

\begin{center}
\begin{tabular}{|p{1.2 in}|p{4in}|}
\hline
\te{integerToChar}&Convert an ASCII value to its character equivalent,
returns an error if the number is out of range\\
\cline{2-2}
&\begin{libverbatim}
function Char integerToChar (Integer n);
\end{libverbatim}
\\
\hline
\end{tabular}
\end{center}

\begin{center}
\begin{tabular}{|p{1.2 in}|p{4in}|}
\hline
\te{isSpace}&Determine if a character is whitespace (space, tab
\verb+\t+, vertical tab \verb+\v+, newline \verb+\n+, carriage return
\verb+\r+, linefeed \verb+\f+) \\
\cline{2-2}
&\begin{libverbatim}
function Bool isSpace (Char c);
\end{libverbatim}
\\
\hline
\end{tabular}
\end{center}

\begin{center}
\begin{tabular}{|p{1.2 in}|p{4in}|}
\hline
\te{isLower}&Determine if a character is a lowercase ASCII character
(\te{a} - \te{z}) \\
\cline{2-2}
&\begin{libverbatim}
function Bool isLower (Char c);
\end{libverbatim}
\\
\hline
\end{tabular}
\end{center}

\begin{center}
\begin{tabular}{|p{1.2 in}|p{4in}|}
\hline
\te{isUpper}&Determine if a character is an uppercase ASCII character
(\te{A} - \te{Z})\\
\cline{2-2}
&\begin{libverbatim}
function Bool isUpper (Char c);
\end{libverbatim}
\\
\hline
\end{tabular}
\end{center}

\begin{center}
\begin{tabular}{|p{1.2 in}|p{4in}|}
\hline
\te{isAlpha}& Determine if a character is an ASCII letter, either
upper or lowercase\\
\cline{2-2}
&\begin{libverbatim}
function Bool isAlpha (Char c);
\end{libverbatim}
\\
\hline
\end{tabular}
\end{center}

\begin{center}
\begin{tabular}{|p{1.2 in}|p{4in}|}
\hline
\te{isDigit}& Determine if a character is an ASCII decimal digit (\te{0}
- \te{9})\\
\cline{2-2}
&\begin{libverbatim}
function Bool isDigit (Char c);
\end{libverbatim}
\\
\hline
\end{tabular}
\end{center}

\begin{center}
\begin{tabular}{|p{1.2 in}|p{4in}|}
\hline
\te{isAlphaNum}&Determine if a character is an ASCII letter or
decminal digit \\
\cline{2-2}
&\begin{libverbatim}
function Bool isAlphaNum (Char c);
\end{libverbatim}
\\
\hline
\end{tabular}
\end{center}



\begin{center}
\begin{tabular}{|p{1.2 in}|p{4in}|}
\hline
\te{isOctDigit}&Determine if a character is an ASCII octal digit (\te{0}
- \te{7})\\
\cline{2-2}
&\begin{libverbatim}
function Bool isOctDigit (Char c);
\end{libverbatim}
\\
\hline
\end{tabular}
\end{center}

\begin{center}
\begin{tabular}{|p{1.2 in}|p{4in}|}
\hline
\te{isHexDigit}&Determine if a character is an ASCII hexadecimal digit (\te{0}
- \te{9}, \te{a} - \te{f}, or \te{A} - \te{F})\\
\cline{2-2}
&\begin{libverbatim}
function Bool isHexDigit (Char c);
\end{libverbatim}
\\
\hline
\end{tabular}
\end{center}

\begin{center}
\begin{tabular}{|p{1.2 in}|p{4in}|}
\hline
\te{toUpper}& Convert an ASCII lowercase letter to uppercase;
other characters are unchanged\\
\cline{2-2}
&\begin{libverbatim}
function Char toUpper (Char c);
\end{libverbatim}
\\
\hline
\end{tabular}
\end{center}

\begin{center}
\begin{tabular}{|p{1.2 in}|p{4in}|}
\hline
\te{toLower}& Convert an ASCII uppercase letter to lowercase;
other characters are unchanged\\
\cline{2-2}
&\begin{libverbatim}
function Char toLower (Char c);
\end{libverbatim}
\\
\hline
\end{tabular}
\end{center}

\begin{center}
\begin{tabular}{|p{1.2 in}|p{4in}|}
\hline
\te{digitToInteger}& Convert an ASCII decimal digit to its numeric
value (\te{0} to 0, unlike \te{charToInteger} which would return the ASCII
code 48); returns an error  if the character is not a digit.\\
\cline{2-2}
&\begin{libverbatim}
function Integer digitToInteger (Char c);
\end{libverbatim}
\\
\hline
\end{tabular}
\end{center}

\begin{center}
\begin{tabular}{|p{1.2 in}|p{4in}|}
\hline
\te{digitToBits}&  Convert an ASCII decimal digit to its numeric
value; returns an error  if the character is not a digit. Same as
\te{digitToInteger}, but returns the value as a bit vector; 
the vector can be any size, but the user will get an error if the
 size is too small to represent the value \\
\cline{2-2}
&\begin{libverbatim}
function Bit#(n) digitToBits (Char c);
\end{libverbatim}
\\
\hline
\end{tabular}
\end{center}

\begin{center}
\begin{tabular}{|p{1.2 in}|p{4in}|}
\hline
\te{integerToDigit}&Convert a decimal digit value (0 to 9) to the ASCII character for that
digit; returns an error if the value is out of range. This function is
the complement of \te{digitToInteger} \\
\cline{2-2}
&\begin{libverbatim}
function Char integerToDigit (Integer d);
\end{libverbatim}
\\
\hline
\end{tabular}
\end{center}

\begin{center}
\begin{tabular}{|p{1.2 in}|p{4in}|}
\hline
\te{bitsToDigit}& Convert a Bit type digit value to the ASCII
character for that digit; returns an error if the value is out of
range.  This is the same as \te{integerToDigit} but for values that
are Bit types\\
\cline{2-2}
&\begin{libverbatim}
function Char bitsToDigit (Bit#(n) d);
\end{libverbatim}
\\
\hline
\end{tabular}
\end{center}

\begin{center}
\begin{tabular}{|p{1.2 in}|p{4in}|}
\hline
\te{hexDigitToInteger}& Convert an ASCII decimal digit to its numeric,
including hex characters \te{a} - \te{f} and \te{A} - \te{F}\\
\cline{2-2}
&\begin{libverbatim}
function Integer hexDigitToInteger (Char c);
\end{libverbatim}
\\
\hline
\end{tabular}
\end{center}

\begin{center}
\begin{tabular}{|p{1.2 in}|p{4in}|}
\hline
\te{hexDigitToBits}&  Convert an ASCII decimal digit to its numeric,
including hex characters \te{a} - \te{f} and \te{A} - \te{F} returning
the value as a bit vector.  The vector can be any size, but an error
will be returned if the size is too small to represent the value.\\
\cline{2-2}
&\begin{libverbatim}
function Bit#(n) hexDigitToBits (Char c);
\end{libverbatim}
\\
\hline
\end{tabular}
\end{center}

\begin{center}
\begin{tabular}{|p{1.2 in}|p{4in}|}
\hline
\te{integerToHexDigit}& Convert a hexadecimal digit value (0 to 15) to the ASCII character for that
digit; returns an  error if the value is out of range. This function is the complement of
\te{hexDigitToInteger}.  The function returns lowercase for the
letters \te{a} to \te{f};  apply the function \te{toUpper} to get uppercase.\\
\cline{2-2}
&\begin{libverbatim}
function Char integerToHexDigit (Integer d);
\end{libverbatim}
\\
\hline
\end{tabular}
\end{center}

\begin{center}
\begin{tabular}{|p{1.2 in}|p{4in}|}
\hline
\te{bitsToHexDigit}& Convert a Bit type hexadecimal digit value to the
ASCII character for that digit, returns an error if the value is out
of range. The function returns lowercase for the
letters \te{a} to \te{f};  apply the function \te{toUpper} to get uppercase.\\

\cline{2-2}
&\begin{libverbatim}
function Char bitsToHexDigit (Bit#(n) d);
\end{libverbatim}
\\
\hline
\end{tabular}
\end{center}




% %%%%%%%%%%%%%%%%%%%%%%%%%%%%%%%%

\com{Fmt}
\subsubsection{Fmt}

\index{Fmt@\te{Fmt} (type)}
\label{prelude-fmt}

The \te{Fmt} primitive type provides a representation of arguments to the
\te{\$display} family of system tasks that
can be  manipulated in BSV
code. \te{Fmt} representations of data objects can be written
hierarchically and applied to polymorphic types. 

Objects of type \te{Fmt} can be
supplied directly as arguments to system tasks in the \te{\$display}
family.  An object of type \te{Fmt} is returned by the \te{\$format}
system task.  

 The \te{Fmt} type is
part of the \te{Arith} class, but only the addition (\te{+}) operator
is defined.   All other \te{Arith} operators will produce an error message. 




\begin{center}
\begin{tabular}{|c|c|c|c|c|c|c|c|c|c|}
\hline
\multicolumn{10}{|c|}{Type Classes for \te{Fmt}}\\
\hline
\hline
&\te{Bits}&\te{Eq}&\te{Literal}&\te{Arith}&\te{Ord}&\te{Bounded}&\te{Bitwise}&\te{Bit}&\te{Bit}\\
&&&&&&&&\te{Reduction}&\te{Extend}\\
\hline
\te{Fmt}&&&$\surd$&$\surd$&&&&&\\
\hline
\end{tabular}
\end{center}


{\bf Examples}
\begin{verbatim}
   Reg#(Bit#(8)) count <- mkReg(0);
   Fmt f = $format("(%0d)", count + 1);
   $display(" XYZ ", f, " ", $format("(%0d) ", count));
      
   \\value displayed:  XYZ (6) (5)
\end{verbatim}



% %%%%%%%%%%%%%%%%%%%%%%%%%%%%%%%%
\subsubsection{Void}
\label{sec-void}
\index{Void@\te{Void} (type)}

The \te{Void} type is a type which has one literal \te{?} used for
constructing concrete values of the type \te{void}
.
The \te{Void} type is part of the \te{Bits} and \te{Literal} typeclasses.

\begin{center}
\begin{tabular}{|c|c|c|c|c|c|c|c|c|c|}
\hline
\multicolumn{10}{|c|}{Type Classes for \te{Void}}\\
\hline
\hline
&\te{Bits}&\te{Eq}&\te{Literal}&\te{Arith}&\te{Ord}&\te{Bounded}&\te{Bitwise}&\te{Bit}&\te{Bit}\\
&&&&&&&&\te{Reduction}&\te{Extend}\\
\hline
\te{Void}&$\surd$&&$\surd$&&&&&&\\
\hline
\end{tabular}
\end{center}

{\bf Examples}

\begin{verbatim}
   typedef union tagged {
       void    Invalid;
       data_t  Valid;
   } Maybe #(type data_t) deriving (Eq, Bits);

   typedef union tagged {
        void     InvalidFile ;
        Bit#(31) MCD;
        Bit#(31) FD;
   } File;
\end{verbatim}



% %%%%%%%%%%%%%%%%%%%%%%%%%%%%%%%%
\com{Maybe}
\subsubsection{Maybe}
\label{sec-maybe}
\index{Maybe@\te{Maybe} (type)}
\index{Invalid@\te{Invalid} (type constructor)}
\index{Valid@\te{Valid} (type constructor)}
\index{fromMaybe@\te{fromMaybe} (\te{Maybe} function)}
\index{isValid@\te{isValid} (\te{Maybe} function)}
\index[function]{Prelude!fromMaybe}
\index[function]{Prelude!isValid}

The \te{Maybe} type is used for tagging values as either
\emph{Valid} or \emph{Invalid}.   If the value is \emph{Valid}, the value
contains a datatype \te{data\_t}.

\begin{libverbatim}
   typedef union tagged {
       void    Invalid;
       data_t  Valid;
   } Maybe #(type data_t) deriving (Eq, Bits);
\end{libverbatim}

\begin{center}
\begin{tabular}{|c|c|c|c|c|c|c|c|c|c|}
\hline
\multicolumn{10}{|c|}{Type Classes for \te{Maybe}}\\
\hline
\hline
&\te{Bits}&\te{Eq}&\te{Literal}&\te{Arith}&\te{Ord}&\te{Bounded}&\te{Bitwise}&\te{Bit}&\te{Bit}\\
&&&&&&&&\te{Reduction}&\te{Extend}\\
\hline
\te{Maybe}&$\surd$&$\surd$&&&&&&&\\
\hline
\end{tabular}
\end{center}


The \te{Maybe} data type provides functions to check if the value is
\emph{Valid} and to extract the valid value.

\begin{center}
\begin{tabular}{|p{1 in}|p{4in}|}
\hline
\multicolumn{2}{|c|}{\te{Maybe} Functions}\\
\hline
\hline
\te{fromMaybe}&Extracts the \te{Valid} value out of a \te{Maybe}
argument.  If the tag is \te{Invalid} the default value, \te{defaultval},
is returned.\\
\cline{2-2}
&\begin{libverbatim}
function data_t fromMaybe( data_t defaultval,  
                           Maybe#(data_t) val ) ;
\end{libverbatim}
\\
\hline
\end{tabular}
\end{center}
\begin{center}
\begin{tabular}{|p{1 in}|p{4in}|}
\hline
\te{isValid}& Returns a value of \te{True} if the \te{Maybe} argument is \te{Valid}.\\
\cline{2-2}
&\begin{libverbatim}
function Bool isValid( Maybe#(data_t) val ) ;
\end{libverbatim}
\\
\hline
\end{tabular}
\end{center}

{\bf Examples}
\begin{verbatim}
   RWire#(int) rw_incr <- mkRWire();  //  increment method is being invoked
   RWire#(int) rw_decr <- mkRWire();  //  decrement method is being invoked

   rule doit;
      Maybe#(int) mbi = rw_incr.wget();
      Maybe#(int) mbd = rw_decr.wget();
      int   di        = fromMaybe (?, mbi);
      int   dd        = fromMaybe (?, mbd);
      if      ((! isValid (mbi)) && (! isValid (mbd)))
         noAction;
      else if (   isValid (mbi)  && (! isValid (mbd)))
         value2 <= value2 + di;
      else if ((! isValid (mbi)) &&    isValid (mbd))
         value2 <= value2 - dd;
      else // (   isValid (mbi)  &&    isValid (mbd))
         value2 <= value2 + di - dd;
   endrule
\end{verbatim}

% %%%%%%%%%%%%%%%%%%%%%%%%%%%%%%%%
\com{Tuples}
\subsubsection{Tuples}
\index{tuples!type definition}
\label{sec-tuples}


Tuples are predefined structures which group a small number of values
together.  
The following pseudo code explains the structure of the 
tuples.  You cannot define your own tuples, but must use the seven predefined
tuples,  \te{Tuple2} through 
\te{Tuple8}.  As shown, \te{Tuple2} groups two items together, \te{Tuple3}
groups three items together, up through \te{Tuple8} which groups eight
items together.     

\begin{libverbatim}
 typedef struct{ 
     a tpl_1;
     b tpl_2;
   } Tuple2 #(type a, type b) deriving (Bits, Eq, Bounded);

 typedef struct{ 
     a tpl_1;
     b tpl_2;
     c tpl_3; 
  } Tuple3 #(type a, type b, type c) deriving (Bits, Eq, Bounded);

 typedef struct{ 
     a tpl_1;
     b tpl_2;
     c tpl_3; 
     d tpl_4;
  } Tuple4 #(type a, type b, type c, type d) deriving (Bits, Eq, Bounded);

 typedef struct{ 
     a tpl_1;
     b tpl_2;
     c tpl_3; 
     d tpl_4;
     e tpl_5;
  } Tuple5 #(type a, type b, type c, type d, type e) 
    deriving (Bits, Eq, Bounded);

 typedef struct{ 
     a tpl_1;
     b tpl_2;
     c tpl_3; 
     d tpl_4;
     e tpl_5;
     f tpl_6;
  } Tuple6 #(type a, type b, type c, type d, type e, type f) 
    deriving (Bits, Eq, Bounded);

 typedef struct{ 
     a tpl_1;
     b tpl_2;
     c tpl_3; 
     d tpl_4;
     e tpl_5;
     f tpl_6;
     g tpl_7;
  } Tuple7 #(type a, type b, type c, type d, type e, type f, type g) 
    deriving (Bits, Eq, Bounded);

 typedef struct{ 
     a tpl_1;
     b tpl_2;
     c tpl_3; 
     d tpl_4;
     e tpl_5;
     f tpl_6;
     g tpl_7;
     h tpl_8;
  } Tuple8 #(type a, type b, type c, type d, type e, type f, type g, type h) 
    deriving (Bits, Eq, Bounded);
\end{libverbatim}

\begin{center}
\begin{tabular}{|c|c|c|c|c|c|c|c|c|c|}
\hline
\multicolumn{10}{|c|}{Type Classes for \te{Tuples}}\\
\hline
\hline
&\te{Bits}&\te{Eq}&\te{Literal}&\te{Arith}&\te{Ord}&\te{Bounded}&\te{Bitwise}&\te{Bit}&\te{Bit}\\
&&&&&&&&\te{Reduction}&\te{Extend}\\
\hline
\te{TupleN}&$\surd$&$\surd$&&&$\surd$&$\surd$&&&\\
\hline
\end{tabular}
\end{center}

Tuples cannot be manipulated like normal structures; you cannot create
values of and select fields from tuples as you would a normal structure.
Values of these types can be created only by applying a predefined family of
constructor functions.
\index{tuples!expressions}

\begin{center}
\begin{tabular}{|p{2.6 in}|p{2.7 in}|}
\hline
\multicolumn{2}{|c|}{\te{Tuple} Constructor Functions}\\
\hline
\hline
\te{tuple2 (e1, e2)}& Creates a variable of type Tuple2 with component values e1
and e2.\\
\hline
\te{tuple3 (e1, e2, e3)}& Creates a variable of type Tuple3 with
values e1, e2, and e3.\\
\hline
\te{tuple4 (e1, e2, e3, e4)}& Creates a variable of type Tuple4 with component 
values e1, e2, e3, and e4.\\
\hline
\te{tuple5 (e1, e2, e3, e4, e5)}& Creates a variable of type Tuple5
with component values e1, e2, e3, e4, and e5.\\
\hline
\te{tuple6 (e1, e2, e3, e4, e5, e6)}& Creates a variable of type
Tuple6 with component values e1, e2, e3, e4, e5, and e6.\\
\hline
\te{tuple7 (e1, e2, e3, e4, e5, e6, e7)}& Creates a variable of type
Tuple7 with component values e1, e2, e3, e4, e5, e6, and e7.\\
\hline
\te{tuple8 (e1, e2, e3, e4, e5, e6, e7, e8)}& Creates a variable of type
Tuple8 with component values e1, e2, e3, e4, e5, e6, e7, and e8.\\
\hline
\end{tabular}
\end{center}

Fields of these types can be extracted only by applying a predefined family of
selector functions.
\index{tuples!selecting components}
\begin{center}
\begin{tabular}{|p{1 in}|p{4.3in}|}
\hline
\multicolumn{2}{|c|}{\te{Tuple} Extract Functions}\\
\hline
\hline
\te{tpl\_1 (x)}&Extracts the first field of x from a Tuple2 to Tuple8.\\
\hline
\te{tpl\_2 (x)}&Extracts the second field of x from a Tuple2 to Tuple8.\\
\hline
\te{tpl\_3 (x)}&Extracts the third field of x from a Tuple3 to Tuple8.\\
\hline
\te{tpl\_4 (x)}&Extracts the fourth field of x from a Tuple4 to Tuple8.\\
\hline
\te{tpl\_5 (x)}&Extracts the fifth field of x from a Tuple5 to Tuple8.\\
\hline
\te{tpl\_6 (x)}&Extracts the sixth field of x from a Tuple6, Tuple7 or Tuple8.\\
\hline
\te{tpl\_7 (x)}&Extracts the seventh field of x from a  Tuple7 or Tuple8.\\
\hline
\te{tpl\_8 (x)}&Extracts the seventh field of x from a Tuple8.\\
\hline
\end{tabular}
\end{center}

{\bf Examples}

\begin{verbatim}
   Tuple2#( Bool, int ) foo = tuple2( True, 25 );
   Bool field1 = tpl_1( foo ); // this is value 1 in the list
   int  field2 = tpl_2( foo ); // this is value 2 in the list
   foo = tuple2( !field1, field2 );
\end{verbatim}

% %%%%%%%%%%%%%%%%%%%%%%%%%%%%%%%%
\com{Array}
\subsubsection{Array}
\index{array!named type}
\index{Array@\te{Array} (type)}
\label{sec-array}

Array variables are generally declared anonymously, using the bracket
syntax.  However,
the type of such variables can be expressed with the type constructor
\te{Array}, when an explicit type is needed.

\begin{center}
\begin{tabular}{|c|c|c|c|c|c|c|c|c|c|}
\hline
\multicolumn{10}{|c|}{Type Classes for \te{Array}}\\
\hline
\hline
&\te{Bits}&\te{Eq}&\te{Literal}&\te{Arith}&\te{Ord}&\te{Bounded}&\te{Bitwise}&\te{Bit}&\te{Bit}\\
&&&&&&&&\te{Reduction}&\te{Extend}\\
\hline
\te{Array}&&$\surd$&&&&&&&\\
\hline
\hline
\end{tabular}
\end{center}

For example, the following declarations using bracket syntax:
\begin{verbatim}
   Bool arr[3];

   function Bool fn(Bool bits[]);
\end{verbatim}
are equivalent to the following declarations using the explicit type constructor:
\begin{verbatim}
   Array#(Bool) arr;

   function Bool fn(Array#(Bool) bits);
\end{verbatim}
Note that, unlike \te{Vector}, the size of an array is not part of its type.
In the first declaration, a size is given for the array \te{arr}.  However,
since \te{arr} is not assigned to a value, the size is unused here.  If the
array were assigned, the size would be used like a type declaration, to
check that the assigned value has the declared size.  Since it is not part
of the type, this check would occur during elaboration, and not during
type checking.

% %%%%%%%%%%%%%%%%%%%%%%%%%%%%%%%%
\com{Ordering}
\subsubsection{Ordering}
\label{sec-ordering}
\index{Ordering@\te{Ordering} (type)}


The \te{Ordering} type is used as the return type for the result of
generic comparators, including the \te{compare} function defined in
the \te{Ord} (Section \ref{sec-ord}) type class.  The valid  values
of \te{Ordering} are: \te{LT}, \te{GT}, and \te{EQ}. 


\begin{libverbatim}
   typedef enum {
       LT,
       EQ,
       GT
   } Ordering deriving (Eq, Bits, Bounded);
\end{libverbatim}

\begin{center}
\begin{tabular}{|c|c|c|c|c|c|c|c|c|c|}
\hline
\multicolumn{10}{|c|}{Type Classes for \te{Ordering}}\\
\hline
\hline
&\te{Bits}&\te{Eq}&\te{Literal}&\te{Arith}&\te{Ord}&\te{Bounded}&\te{Bitwise}&\te{Bit}&\te{Bit}\\
&&&&&&&&\te{Reduction}&\te{Extend}\\
\hline
\te{Ordering}&$\surd$&$\surd$&&&&$\surd$&&&\\
\hline
\end{tabular}
\end{center}

{\bf Examples}

\begin{verbatim}
function Ordering onKey(Record r1, Record r2);
   return compare(r1.key,r2.key);
endfunction
\end{verbatim}

% %%%%%%%%%%%%%%%%%%%%%%%%%%%%%%%%
\subsubsection{File}
\index{File (type)}

\te{File} is a defined type in BSV which is defined as:
\begin{libverbatim}
    typedef union tagged {
        void     InvalidFile ;
        Bit#(31) MCD;
        Bit#(31) FD;
    } File;
\end{libverbatim}

\begin{center}
\begin{tabular}{|c|c|c|c|c|c|c|c|c|c|}
\hline
\multicolumn{10}{|c|}{Type Classes for \te{File}}\\
\hline
\hline
&\te{Bits}&\te{Eq}&\te{Literal}&\te{Arith}&\te{Ord}&\te{Bounded}&\te{Bitwise}&\te{Bit}&\te{Bit}\\
&&&&&&&&\te{Reduction}&\te{Extend}\\
\hline
\te{File}&$\surd$&$\surd$&&&&&$\surd$&&\\
\hline
\hline
\end{tabular}
\end{center}

\hmm Note: \te{Bitwise} operations are  valid only for subtype \te{MCD}.

The \te{File} type is used by the system tasks for file I/O.

% %%%%%%%%%%%%%%%%%%%%%%%%%%%%%%%%
\com{Clock}
\subsubsection{Clock}

\index{Clock@\te{Clock} (type)}
\label{sec-clock}

\te{Clock}  is an abstract  type of two components: a single \te{Bit}
oscillator and a \te{Bool} gate.

\begin{libverbatim}
     typedef ... Clock ; 
\end{libverbatim}

\te{Clock} is in the \te{Eq} type class, meaning two values can be
compared for equality.

\begin{center}
\begin{tabular}{|c|c|c|c|c|c|c|c|c|c|}
\hline
\multicolumn{10}{|c|}{Type Classes for \te{Clock}}\\
\hline
\hline
&\te{Bits}&\te{Eq}&\te{Literal}&\te{Arith}&\te{Ord}&\te{Bounded}&\te{Bitwise}&\te{Bit}&\te{Bit}\\
&&&&&&&&\te{Reduction}&\te{Extend}\\
\hline
\te{Clock}&&$\surd$&&&&&&&\\
\hline
\end{tabular}
\end{center}

{\bf Examples}

\begin{verbatim}
   Clock clk <- exposeCurrentClock;

   module mkTopLevel( Clock readClk, Reset readRst, Top ifc );
\end{verbatim}

% %%%%%%%%%%%%%%%%%%%%%%%%%%%%%%%%
\com{Reset}
\subsubsection{Reset}
\index{Reset@\te{Reset} (type)}
\label{sec-reset}

\te{Reset} is an abstract type.
\index{Reset@\te{Reset} (type)}

\begin{libverbatim}
     typedef ... Reset ; 
\end{libverbatim}

\te{Reset} is in the \te{Eq} type class, meaning two fields can be
compared for equality.


\begin{center}
\begin{tabular}{|c|c|c|c|c|c|c|c|c|c|}
\hline
\multicolumn{10}{|c|}{Type Classes for \te{Reset}}\\
\hline
\hline
&\te{Bits}&\te{Eq}&\te{Literal}&\te{Arith}&\te{Ord}&\te{Bounded}&\te{Bitwise}&\te{Bit}&\te{Bit}\\
&&&&&&&&\te{Reduction}&\te{Extend}\\
\hline
\te{Reset}&&$\surd$&&&&&&&\\
\hline
\end{tabular}
\end{center}


{\bf Examples}

\begin{verbatim}
   Reset rst <- exposeCurrentReset;
   
   module mkMod (Reset r2, (* reset_by="no_reset" *) Bool b,
                   ModIfc ifc);  

   interface ResetGenIfc;
     interface Reset gen_rst;
   endinterface
\end{verbatim}

% %%%%%%%%%%%%%%%%%%%%%%%%%%%%%%%%
\com{Inout}
\subsubsection{Inout}
\index{Inout@\te{Inout} (type)}
\label{sec-inout}

An \te{Inout} type is a first class type that is used to pass Verilog
inouts through a BSV module.  It takes an argument which is the type
of the underlying signal:

\begin{verbatim}
   Inout#(type t)
\end{verbatim}

For example, the type of an \te{Inout} signal which communicates boolean
values would be:
\begin{verbatim}
   Inout#(Bool)
\end{verbatim}

\begin{center}
\begin{tabular}{|c|c|c|c|c|c|c|c|c|c|}
\hline
\multicolumn{10}{|c|}{Type Classes for \te{Inout}}\\
\hline
\hline
&\te{Bits}&\te{Eq}&\te{Literal}&\te{Arith}&\te{Ord}&\te{Bounded}&\te{Bitwise}&\te{Bit}&\te{Bit}\\
&&&&&&&&\te{Reduction}&\te{Extend}\\
\hline
\te{Inout}&&&&&&&&&\\
\hline
\end{tabular}
\end{center}


An \te{Inout} type is a valid subinterface type (like \te{Clock} and
\te{Reset}).  A value of an \te{Inout} type is \te{clocked\_by} and
\te{reset\_by} a particular clock and reset. 



\te{Inout}s are connectable via the \te{Connectable} typeclass. The
  use of \te{mkConnection} instantiates a Verilog module
  \te{InoutConnect}.  The connected inouts must be on the same clock
and  the same reset.  The clock and reset of the
inouts may be different than the clock and reset of the parent
module of the \te{mkConnection}. 
\begin{libverbatim}
  instance Connectable#(Inout#(a, x1), Inout#(a, x2))
     provisos (Bit#(a,sa));
\end{libverbatim}

 A module with an \te{Inout} subinterface cannot leave that interface
 undefined since there is no way to create or examine inout values in 
BSV.  For example, you cannot even write:
\begin{verbatim}
  Inout#(int) i = ? ;    // not valid in BSV
\end{verbatim}

The
\te{Inout} type exists only so that RTL inout signals can be connected
 in BSV; the ultimate users of the signal will be outside the
BSV code.   An imported Verilog module might have an inout
port that your BSV design doesn't use, but which needs to be exposed
at the top level.  In this case, the submodule will introduce an
inout signal that the BSV cannot read or write, but merely
provides in its interfaces until it is exposed at the top level.  Or,
 a design  may contain two imported Verilog modules that
 have  inout ports that expect to be connected.  You can
 import these  two modules, declaring that they each have a
port of type \te{Inout\#(t)} and connect them together.  The compiler
will check that both ports are of the same type \te{t} and that they are
in the same clock domain with  the same reset.  Beyond that, BSV does
not concern itself with the values of the inout signals.

{\bf Examples}

Instantiating a submodule with an inout and exposing
it at the next level:
\begin{verbatim}
   interface SubIfc;
      ...
      interface Inout#(Bool) b;
   endinterface

   interface TopIfc;
      ...
      interface Inout#(Bool) bus;
   endinterface

   module mkTop (TopIfc);
      SubIfc sub <- mkSub;
      ...
      interface bus = sub.b;
   endmodule
\end{verbatim}

Connecting two submodules, using \te{SubIfc} defined above:
\begin{verbatim}
   module mkTop(...);
      ...
      SubIfc sub1 <- mkSub;
      SubIfc sub2 <- mkSub;
      mkConnection (sub1.b, sub2.b);
      ...
   endmodule
\end{verbatim}

% %%%%%%%%%%%%%%%%%%%%%%%%%%%%%%%%
\com{Action}
\com{ActionValue}
\subsubsection{Action/ActionValue}

\index{Action@\te{Action} (type)}
\index{ActionValue@\te{ActionValue} (type)}
\label{sec-action}


Any expression that is intended to act on the state of the circuit (at circuit
execution time) is called an {\emph{action}} and has type
\te{Action} or \te{ActionValue\#(a)}.
  The type parameter \te{a} represents the type of the
returned value. 

\begin{center}
\begin{tabular}{|c|c|c|c|c|c|c|c|c|c|}
\hline
\multicolumn{10}{|c|}{Type Classes for \te{Action/ActionValue}}\\
\hline
\hline
&\te{Bits}&\te{Eq}&\te{Literal}&\te{Arith}&\te{Ord}&\te{Bounded}&\te{Bitwise}&\te{Bit}&\te{Bit}\\
&&&&&&&&\te{Reduction}&\te{Extend}\\
\hline
\te{Action}&&&&&&&&&\\
\hline
\end{tabular}
\end{center}


The types \te{Action} and \te{ActionValue} are special
keywords, and therefore cannot be redefined. %  Section
% \ref{sec-action} describes \te{Action} and \te{ActionValue} expressions
% in more detail.
  
\BBS
    typedef {\rm{\emph{$\cdots$ abstract $\cdots$}}} struct ActionValue#(type a);
\EBS


\index{noAction@\te{noAction} (empty action)}
\begin{center}
\begin{tabular}{|p{1 in}|p{4in}|}
\hline
\multicolumn{2}{|c|}{\te{ActionValue} type aliases}\\
\hline
\hline
\te{Action}&The \te{Action} type is a special case of the more general type
\te{ActionValue} where nothing is returned.  That is, the returns
type is  \te{(void)}.\\
\cline{2-2}
&\begin{libverbatim}
typedef ActionValue#(void) Action;
\end{libverbatim}
\\
\hline
\end{tabular}
\end{center}
\begin{center}
\begin{tabular}{|p{1 in}|p{4in}|}
\hline
\multicolumn{2}{|c|}{\te{Action} Functions}\\
\hline
\hline
\te{noAction}&An empty \te{Action}, this is an \te{Action} that does nothing.\\
\cline{2-2}
&\begin{libverbatim}
function Action noAction();
\end{libverbatim}
\\
\hline
\end{tabular}
\end{center}

{\bf Examples}

\begin{verbatim}
   method Action grab(Bit#(8) value);
      last_value <= value;
   endmethod
 
   interface IntStack;
      method Action             push (int x);
      method ActionValue#(int)  pop();
   endinterface: IntStack

   seq
     noAction;
   endseq
\end{verbatim}

% %%%%%%%%%%%%%%%%%%%%%%%%%%%%%%%%
\com{Rules}
\subsubsection{Rules}
\label{sec-rules}

A rule  expression has type \te{Rules} and consists of a collection of
individual rule constructs.  Rules are first class objects, hence variables of
type \te{Rules} may be created and manipulated. \te{Rules} values must
eventually be added to a module in order to appear in synthesized hardware.

\index{rules@\te{Rules} (type)}
\index{addRules@\te{addRules} (\te{Rules} function)}
\index[function]{Prelude!addRules}
\index{rJoin@\te{rJoin} (\te{Rules} operator)}
\index{rJoinConflictFree@\te{rJoinConflictFree} (\te{Rules} operator)}
\index{rJoinDescendingUrgency@\te{rJoinDescendingUrgency} (\te{Rules} operator)}
\index{rJoinExecutionOrder@\te{rJoinExecutionOrder} (\te{Rules} operator)}
\index{rJoinMutuallyExclusive@\te{rJoinMutuallyExclusive} (\te{Rules} operator)}
\index{rJoinPreempts@\te{rJoinPreempts} (\te{Rules} operator)}
\index{emptyRules@\te{emptyRules} (\te{Rules} variable)}
\index[function]{Prelude!addRules}
\index[function]{Prelude!rJoin}
\index[function]{Prelude!rJoinConflictFree}
\index[function]{Prelude!rJoinDescendingUrgency}
\index[function]{Prelude!rJoinExecutionOrder}
\index[function]{Prelude!rJoinMutuallyExclusive}
\index[function]{Prelude!rJoinPreempts}

\begin{center}
\begin{tabular}{|c|c|c|c|c|c|c|c|c|c|}
\hline
\multicolumn{10}{|c|}{Type Classes for \te{Rules}}\\
\hline
\hline
&\te{Bits}&\te{Eq}&\te{Literal}&\te{Arith}&\te{Ord}&\te{Bounded}&\te{Bitwise}&\te{Bit}&\te{Bit}\\
&&&&&&&&\te{Reduction}&\te{Extend}\\
\hline
\te{Rules}&&&&&&&&&\\
\hline
\end{tabular}
\end{center}

The \te{Rules} data type provides functions to create, manipulate, and
combine values of the type \te{Rules}.


\begin{center}
\begin{tabular}{|p{1 in}|p{4in}|}
\hline
\multicolumn{2}{|c|}{\te{Rules} Functions}\\
\hline
\hline
\te{emptyRules}& An empty rules variable.\\
\cline{2-2}
&\begin{libverbatim}
function Rules emptyRules();
\end{libverbatim}
\\
\hline
\end{tabular}
\end{center}
\begin{center}
\begin{tabular}{|p{1 in}|p{4in}|}
\hline
\te{addRules}&Takes rules \te{r} and adds them into a module.  
This function may only be called from within
a module. The return type \te{Empty} indicates that the instantiation does not return
anything.\\
\cline{2-2}
&\begin{libverbatim}
module addRules#(Rules r) (Empty);
\end{libverbatim}
\\
\hline
\end{tabular}
\end{center}
\begin{center}
\begin{tabular}{|p{1 in}|p{4in}|}
\hline
\te{rJoin}&Symmetric union of two sets of rules. A symmetric union
means that neither set is implied to have any relation to the other:
not more urgent, not execute before, etc.\\
\cline{2-2}
&\begin{libverbatim}
function Rules rJoin(Rules x, Rules y);
\end{libverbatim}
\\
\hline
\end{tabular}
\end{center}
\begin{center}
\begin{tabular}{|p{1 in}|p{4in}|}
\hline
\te{rJoinPreempts}&Union of two sets of rules, with rules on the left getting
scheduling precedence and blocking the rules on the right.% (see Section \ref{preempts}).  
That is, if a rule in set {\tt x} fires, then all rules in set {\tt y} are 
prevented from firing.  This is the same as specifying \te{descending\_urgency} 
plus a forced conflict.\\
\cline{2-2}
&\begin{libverbatim}
function Rules rJoinPreempts(Rules x, Rules y);
\end{libverbatim}
\\
\hline
\end{tabular}
\end{center}
\begin{center}
\begin{tabular}{|p{.6 in}|p{4.4 in}|}
\hline
\multicolumn{2}{|l|}{\te{rJoinDescendingUrgency}}\\
\hline
&Union of two sets of rule, with rules in the left having higher
urgency.% (see Section \ref{urgency-annotation}). 
That is, if some 
rules compete for resources, then scheduling will select rules in 
set {\tt x} set before set {\tt y}.  If the rules do not conflict, 
no conflict is added; the rules can fire together.\\
\cline{2-2}
&\begin{libverbatim}
function Rules rJoinDescendingUrgency(Rules x, Rules y);
\end{libverbatim}
\\
\hline
\end{tabular}
\end{center}
\begin{center}
\begin{tabular}{|p{.6 in}|p{4.4 in}|}
\hline
\multicolumn{2}{|l|}{\te{rJoinMutuallyExclusive}}\\
\hline
&Union of two sets of rule, with rules in the all rules 
in the left set annotated as mutually exclusive with all 
rules in the right set.% (see Section \ref{mutually-exclusive}). 
No relationship between the rules in the left set or between 
the rules in the right set is assumed. This annotation 
is used in scheduling and checked during simulation. \\
\cline{2-2}
&\begin{libverbatim}
function Rules rJoinMutuallyExclusive(Rules x, Rules y);
\end{libverbatim}
\\
\hline
\end{tabular}
\end{center}
\begin{center}
\begin{tabular}{|p{.6 in}|p{4.4 in}|}
\hline
\multicolumn{2}{|l|}{\te{rJoinExecutionOrder}}\\
\hline
&Union of two sets of rule, with the rules in the left set 
executing before the rules in the right set.% (see Section 
%\ref{execution-order}). 
No relationship between the rules 
in the left set or between the rules in the right set 
is assumed. If any pair of rules cannot execute in the 
specified order in the same clock cycle, that pair
of rules will conflict. \\
\cline{2-2}
&\begin{libverbatim}
function Rules rJoinExecutionOrder(Rules x, Rules y);
\end{libverbatim}
\\
\hline
\end{tabular}
\end{center}
\begin{center}
\begin{tabular}{|p{.6 in}|p{4.4 in}|}
\hline
\multicolumn{2}{|l|}{\te{rJoinConflictFree}}\\
\hline
&Union of two sets of rule, with the rules in the left set 
annotated as conflict-free with the rules in the right set.
% (see Section \ref{conflict-free-annotation}).
This assumption is used
during scheduling and checked during simulation. No relationship 
between the rules in the left set or between the rules in 
the right set is assumed.  \\
\cline{2-2}
&\begin{libverbatim}
function Rules rJoinConflictFree(Rules x, Rules y);
\end{libverbatim}
\\
\hline
\end{tabular}
\end{center}

{\bf Examples}
(This is an excerpt from a complete example in the BSV Reference Guide.)
\begin{verbatim}
   function Rules incReg(Reg#(CounterType) a);
     return( rules
       rule addOne;
          a <= a + 1;
       endrule
     endrules);
   endfunction

      // Add incReg rule to increment the counter
      addRules(incReg(asReg(counter)));
\end{verbatim}




% ================================================================
% \subsubsection{Module}

% Every variable and every expression in {\BSV} has a \emph{type}.  A
% module  expression has type \te{Module}.

% \index{Module@\te{Module} (type)}

% The \te{Module} type  belongs to the \te{Monad} typeclass.
% \begin{libverbatim}
% instance Monad #(Module);
% \end{libverbatim}




%================================================================
\subsection{Operations on Numeric Types}

\subsubsection{Size Relationship/Provisos}

\index{provisos}
\index{Add@\te{Add} (type provisos)}
\index{Max@\te{Max} (type provisos)}
\index{Min@\te{Min} (type provisos)}
\index{Log@\te{Log} (type provisos)}
\index{Mul@\te{Mul} (type provisos)}
\index{Div@\te{Div} (type provisos)}



These classes are  used in provisos to express
constraints between the sizes of types.  

\begin{center}
\begin{tabular}{|p {.5 in}|p{1.5 in}| p{3.0 in}|}
\hline
 Class& Proviso& Description\\
\hline
\hline
\te{Add}&\verb'Add#(n1,n2,n3)'&Assert $n1 + n2 = n3$\\
\hline
\te{Mul}&\verb'Mul#(n1,n2,n3)'&Assert $n1 * n2 = n3$\\
\hline
\te{Div}&\verb'Div#(n1,n2,n3)'&Assert ceiling $n1 / n2 = n3$\\
\hline
\te{Max}&\verb'Max#(n1,n2,n3)'&Assert $\max(n1,n2) = n3$\\
\hline
\te{Min}&\verb'Min#(n1,n2,n3)'&Assert $\min(n1,n2) = n3$\\
\hline
\te{Log}&\verb'Log#(n1,n2)'&Assert ceiling ${\log_{2}(n1)}=n2$.  \\
\hline
\end{tabular}
\end{center}

Examples of Provisos using size relationships:
\begin{verbatim}
     instance Bits #( Vector#(vsize, element_type), tsize)
        provisos (Bits#(element_type, sizea), 
                  Mul#(vsize, sizea, tsize));       // vsize * sizea = tsize

     function Vector#(vsize1, element_type)
          cons (element_type elem, Vector#(vsize, element_type) vect)
       provisos (Add#(1, vsize, vsize1));           // 1 + vsize = vsize1

     function Vector#(mvsize,element_type)
           concat(Vector#(m,Vector#(n,element_type)) xss)
       provisos (Mul#(m,n,mvsize));                 // m * n = mvsize
\end{verbatim}


\subsubsection{Size Relationship Type Functions}

\index{TAdd@\te{TAdd} (type functions)}
\index{TSub@\te{TSub} (type functions)}
\index{TLog@\te{TLog} (type functions)}
\index{TExp@\te{TExp} (type functions)}
\index{TMul@\te{TMul} (type functions)}
\index{TDiv@\te{TDiv} (type functions)}
\index{TMax@\te{TMax} (type functions)}
\index{TMin@\te{TMin} (type functions)}
\index[function]{Prelude!TAdd}
\index[function]{Prelude!TSub}
\index[function]{Prelude!TLog}
\index[function]{Prelude!TExp}
\index[function]{Prelude!TMul}
\index[function]{Prelude!TMax}
\index[function]{Prelude!TMin}
\index[function]{Prelude!TDiv}

These type functions are used when ``defining'' size relationships between data
types, where the defined value need not (or cannot) be named in a proviso.  They may be used in datatype definition statements when the
size of the datatype may be calculated from other parameters.

\begin{center}
\begin{tabular}{|p {1 in}|p{1.5 in}| p{2.0 in}|}
\hline
Type Function& Size Relationship& Description\\
\hline
\hline
\te{TAdd}&\verb'TAdd#(n1,n2)'&Calculate $n1 + n2$\\
\hline
\te{TSub}&\verb'TSub#(n1,n2)'&Calculate $n1 - n2$\\
\hline
\te{TMul}&\verb'TMul#(n1,n2)'&Calculate $n1 * n2$\\
\hline
\te{TDiv}&\verb'TDiv#(n1,n2)'&Calculate ceiling $n1 / n2 $\\
\hline
\te{TLog}&\verb'TLog#(n1)'&Calculate ceiling ${\log_{2}(n1)}$ \\
\hline
\te{TExp}&\verb'TExp#(n1)'&Calculate $2^{n1}$\\
\hline
\te{TMax}&\verb'TMax#(n1,n2)'&Calculate $max(n1,n2)$\\
\hline
\te{TMin}&\verb'TMin#(n1,n2)'&Calculate $min(n1,n2)$\\
\hline
\end{tabular}
\end{center}


Examples using other arithmetic functions:
\begin{verbatim}
   Int#(TAdd#(5,n));                        // defines a signed integer n+5 bits wide
                                            // n must be in scope somewhere
   
   typedef TAdd#(vsize, 8) Bigsize#(numeric type vsize);
                                            // defines a new type Bigsize which
                                            // is 8 bits wider than vsize
   
   typedef Bit#(TLog#(n)) CBToken#(numeric type n); 
                                            // defines a new parameterized type, 
                                            // CBToken, which is log(n) bits wide.

   typedef 8 Wordsize;                      // Blocksize is based on Wordsize      
   typedef TAdd#(Wordsize, 1) Blocksize;
\end{verbatim}


% ================================================================
\subsubsection{valueOf and SizeOf pseudo-functions}
\index{valueof@\texttt{valueof} (pseudo-function of size types)}
\index{valueOf@\texttt{valueOf} (pseudo-function of size types)}
\index{SizeOf@\texttt{SizeOf} (pseudo-function on types)}
\index[function]{Prelude!valueOf}
\index[function]{Prelude!SizeOf}

Prelude provides these pseudo-functions to convert between 
types and numeric values.  The pseudo-function \te{valueof} (or
\te{valueOf}) is used to convert a numeric type into the corresponding
Integer  value.
The pseudo-function \te{SizeOf} is used to convert a type {t} into
the numeric type representing its bit size.  

\begin{center}
\begin{tabular}{|p{1 in}|p{4.6 in}|}
\hline
& \\
\te{valueof}&Converts a numeric type into its Integer value.\\
\te{valueOf}&  \\
\cline{2-2}
&\begin{libverbatim}
function Integer valueOf (t) ;
\end{libverbatim}
\\
\hline
\end{tabular}
\end{center}

{\bf Examples}
\begin{libverbatim}
   module mkFoo (Foo#(n));
     UInt#(n) x;
     Integer y = valueOf(n);
   endmodule
\end{libverbatim}



\begin{center}
\begin{tabular}{|p{1 in}|p{4.6 in}|}
\hline
& \\
\te{SizeOf} & Converts a type into a numeric type representing its
bit size.\\
&  \\
\cline{2-2}
&\begin{libverbatim}
function t SizeOf#(any_type)
   provisos (Bits#(any_type, sa)) ;
\end{libverbatim}
\\
\hline
\end{tabular}
\end{center}

{\bf Examples}
\begin{libverbatim}
   any_type x = structIn;
   Bit#(SizeOf#(any_type)) = pack(structIn);
\end{libverbatim}


% ================================================================
\subsection{Registers and Wires}
\label{prelude-register}


\begin{center}
\begin{tabular}{|p{.8 in}|p{.5in}|p{4 in}|}
\hline
\multicolumn{3}{|c|}{Register and Wire Interfaces and Modules}\\
\hline
Name & Section & Description \\
\hline
\hline
\te{Reg} & \ref{lib-registers} & Register interface \\
\hline
\te{CReg} & \ref{lib-creg} & Implementation of a register with an array
of \te{Reg} interfaces that sequence concurrently \\
\hline
\te{RWire} & \ref{lib-rwire} & Similar to the \te{Reg} interface with output wrapped
in a \te{Maybe} type to indicate validity \\
\hline
\te{Wire} & \ref{lib-wire} & Interchangeable with a \te{Reg} interface,
validity of the data is implicit \\
\hline
\te{BypassWire} & \ref{lib-bypasswire} & Implementation of the \te{Wire}
interface where the \te{\_write method} is \te{always\_enabled} \\
\hline
\te{DWire} &\ref{lib-dwire} & Implementation of the \te{Wire} interface where
the \te{\_read} method is \te{always\_ready} by providing a default value \\
\hline
\te{PulseWire} & \ref{lib-pulsewire} & Similar to the \te{RWire} interface without any data \\
\hline
\te{ReadOnly} & \ref{readonly} & Interface which provides a value \\
\hline
\te{WriteOnly} & \ref{writeonly} & Interface which writes a value \\
\hline
\end{tabular}
\end{center}

%-----------------------------------------

\subsubsection{Reg}
\index{Reg@\te{Reg} (type)}
\index{\_write@\te{\_write} (\te{Reg} interface method)}
\index{\_read@\te{\_read} (\te{Reg} interface method)}
\index{mkReg@\te{mkReg} (module)}
\index{mkRegU@\te{mkRegU} (module)}
\index{mkRegA@\te{mkRegA} (module)}
\index{asReg@\te{asReg} (\te{Reg} function)}
\label{lib-registers}
\index[function]{Prelude!mkReg}
\index[function]{Prelude!mkRegU}
\index[function]{Prelude!mkRegA}
\index[function]{Prelude!asReg}
\index[function]{Prelude!readReg}
\index{readReg@\te{readReg} (\te{Reg} function)}
\index{writeReg@\te{writeReg} (\te{Reg} function)}
\index[function]{Prelude!writeReg}

The most elementary module available in {\BSV} is the register, which
has a  \te{Reg} interface.  Registers are polymorphic, i.e., in
principle they can hold a value of any type but, of course, ultimately
registers store bits.  Thus, the provisos on register modules indicate that
the type of the value stored in the register must be in the \te{Bits}
type class, i.e., the operations \te{pack} and \te{unpack} are defined
on the type to convert into bits and back.

Note that all Bluespec registers are considered
atomic units, which means that even if one bit is updated
(written), then all the bits are considered updated. This
prevents multiple rules from updating register fields in an
inconsistent manner.

When scheduling register modules, reads occur before writes.  That is,
any rule which reads from a register
must be scheduled earlier than any other rule which writes to the
register. The value read from the register is the value written in the
previous clock cycle.

{\bf Interfaces and Methods}

The \te{Reg} interface contains two methods, \te{\_write} and \te{\_read}.

\begin{libverbatim}
   interface Reg #(type a_type);
       method Action _write(a_type x1);
       method a_type _read();
   endinterface: Reg
\end{libverbatim}

The \te{\_write} and \te{\_read} methods are rarely used.  Instead, for
writes, one uses the
non-blocking assignment notation and, for reads, one just mentions the
register interface in an expression.  % Both these notations are described
% in more detail in the BSV Reference Guide.

\begin{center}
\begin{tabular}{|p{.5in}|p{.7in}|p{1.5 in}|p{.4in}|p{1.5 in}|}
\hline
\multicolumn{5}{|c|}{\te{Reg} Interface}\\
\hline
\multicolumn{3}{|c|}{Method}&\multicolumn{2}{|c|}{Arguments}\\
\hline
Name & Type & Description& Name &\multicolumn{1}{|c|}{Description} \\
\hline
\hline 
\te{\_write}&\te{Action}&writes a value \te{x1} &\te{x1}&data
to be written \\
\hline
\te{\_read}&\te{a\_type}&returns the value of the register&&\\
\hline

\end{tabular}
\end{center}

{\bf Modules}

Prelude provides three modules to create a register: \te{mkReg} creates a
register with a given reset value, \te{mkRegU} creates a register
without any reset, and  \te{mkRegA} creates a register with a given
reset value and with asynchronous reset logic.  


\begin{center}
\begin{tabular}{|p{1.2 in}|p{4.4 in}|}
\hline
\te{mkReg} &Make a register with a given reset value.  Reset logic is
synchronous. \\
\cline{2-2}
&\begin{libverbatim}
module mkReg#(parameter a_type resetval)(Reg#(a_type))
  provisos (Bits#(a_type, sizea));
\end{libverbatim}
\\
\hline
\end{tabular}
\end{center}
\begin{center}
\begin{tabular}{|p{1.2 in}|p{4.4 in}|}
\hline
\te{mkRegU}&Make a register without any reset; initial simulation
value is alternating 01 bits.\\
\cline{2-2}
&\begin{libverbatim}
module mkRegU(Reg#(a_type))
  provisos (Bits#(a_type, sizea));
\end{libverbatim}
\\
\hline
\end{tabular}
\end{center}
\begin{center}
\begin{tabular}{|p{1.2 in}|p{4.4 in}|}
\hline
\te{mkRegA}&Make a register with a given reset value.  Reset logic is
asynchronous.\\
\cline{2-2}
&\begin{libverbatim}
module mkRegA#(parameter a_type resetval)(Reg#(a_type))
  provisos (Bits#(a_type, sizea));
\end{libverbatim}
\\ \hline
\hline
\end{tabular}
\end{center}


{\bf Scheduling}

When scheduling register modules, reads occur before writes.  That is,
any rule which reads from a register
must be scheduled earlier than any other rule which writes to the
register.  The value read from the register is the value written in the
previous clock cycle.
Multiple rules can write to a register in a given clock cycle, with the
effect that a later rule overwrites the value written by earlier rules.

\begin{center}
\begin{tabular}{|p{.75 in}|c|c|}
\hline
\multicolumn{3}{|c|}{Scheduling Annotations}\\
\multicolumn{3}{|c|}{mkReg, mkRegU, mkRegA}\\
\hline
&{read}&{write}\\
\hline
\hline
{read}&CF&SB\\
\hline
{write}&SA& SBR\\
\hline
\hline
\end{tabular}
\end{center}



{\bf Functions}

Three functions are provided for using registers: \te{asReg} returns the register
interface instead of the value of the register; \te{readReg} reads the
value of a register, useful when managing vectors or lists of
registers; and \te{writeReg} to write a value into a register, also
useful when managing vectors or lists of registers.

\begin{center}
\begin{tabular}{|p{1.2 in}|p{4.4 in}|}
\hline
\te{asReg}&Treat a register as a register, i.e., suppress the normal behavior
where the interface name implicitly represents the value that the
register contains (the \te{\_read} value).  This
function returns
the register interface, not the value of the register.\\
\cline{2-2}
&\begin{libverbatim}
function Reg#(a_type) asReg(Reg#(a_type) regIfc);
\end{libverbatim}
\\
\hline
\end{tabular}
\end{center}
\begin{center}
\begin{tabular}{|p{1.2 in}|p{4.4 in}|}
\hline
\te{readReg}&Read the value out of a register. Useful for giving  as
the  argument to higher-order vector and list functions.\\
\cline{2-2}
&\begin{libverbatim}
function a_type readReg(Reg#(a_type) regIfc);
\end{libverbatim}
\\
\hline
\end{tabular}
\end{center}
\begin{center}
\begin{tabular}{|p{1.1 in}|p{4.5 in}|}
\hline
\te{writeReg}&Write a value into a register.  Useful for giving  as
the argument to higher-order vector and list functions.\\  
\cline{2-2}
&\begin{libverbatim}
function Action writeReg(Reg#(a_atype) regIfc, a_type din);
\end{libverbatim}
\\ \hline
\end{tabular}
\end{center}

{\bf Examples}

\begin{verbatim}
   Reg#(ta) res <- mkReg(0);

   // create board[x][y]
   Reg#(ta) pipe[depth];
   for (Integer i=0; i<depth; i=i+1) begin
      Reg#(ta) c();
      mkReg#(0) xinst(c);
      pipe[i] = asReg(c);
   end

   function a readReg(Reg#(a) r);
      return(r);
   endfunction
\end{verbatim}


%-----------------------------------------

\subsubsection{CReg}
\index{mkCReg@\te{mkCReg} (module)}
\index{mkCRegU@\te{mkCRegU} (module)}
\index{mkCRegA@\te{mkCRegA} (module)}
\label{lib-creg}
\index[function]{Prelude!mkCReg}
\index[function]{Prelude!mkCRegU}
\index[function]{Prelude!mkCRegA}

The basic register modules described in \ref{lib-registers} have scheduling
annotations that do not allow two rules to read and write a register
concurrently (that is, sequentially in the same clock cycle).  Implementing
this concurrency requires bypassing, so that a value written by one rule is
visible to the next rule that wants to read the register.  This can create
long paths, and so explicit input from the designer is preferred.  Therefore,
the basic registers do not support this bypassing.  If a designer wants
concurrent register access between rules, they must explicitly request
and manage this, by instantiating one of the \te{CReg} family of modules.

These concurrent registers are also known as EHRs (Ephemeral History Registers)
in work by Arvind and Rosenband\footnote{Daniel~L. Rosenband.
\newblock {The Ephemeral History Register: Flexible Scheduling for Rule-Based Designs}.
\newblock In {\em Proc. MEMOCODE'04}, June 2004.}.

{\bf Modules}

Prelude provides three modules to create a concurrent register: \te{mkCReg}
creates a concurrent register with a given reset value, \te{mkCRegU} creates
a concurrent register without any reset, and  \te{mkCRegA} creates a
concurrent register with a given reset value and with asynchronous reset logic.  

\begin{center}
\begin{tabular}{|p{1.2 in}|p{4.4 in}|}
\hline
\te{mkCReg} & Make a concurrent register with a given number of ports
and with a given reset value.  Reset logic is synchronous. \\
\cline{2-2}
&\begin{libverbatim}
module mkCReg#(parameter Integer n,
               parameter a_type resetval)
             (Reg#(a_type) ifc[])
  provisos (Bits#(a_type, sizea));
\end{libverbatim}
\\
\hline
\end{tabular}
\end{center}
\begin{center}
\begin{tabular}{|p{1.2 in}|p{4.4 in}|}
\hline
\te{mkCRegU} & Make a concurrent register with a given number of ports
and without any reset. Initial simulation value is alternating 01 bits. \\
\cline{2-2}
&\begin{libverbatim}
module mkCRegU#(parameter Integer n) (Reg#(a_type) ifc[])
  provisos (Bits#(a_type, sizea));
\end{libverbatim}
\\
\hline
\end{tabular}
\end{center}
\begin{center}
\begin{tabular}{|p{1.2 in}|p{4.4 in}|}
\hline
\te{mkCRegA} & Make a concurrent register with a given number of ports
and with a given reset value. Reset logic is asynchronous. \\
\cline{2-2}
&\begin{libverbatim}
module mkCRegA#(parameter Integer n,
                parameter a_type resetval) 
              (Reg#(a_type) ifc[])
  provisos (Bits#(a_type, sizea));
\end{libverbatim}
\\ \hline
\hline
\end{tabular}
\end{center}

As indicated by the bracket notation in the interface type, these
modules provide an array of \te{Reg} interfaces, whose methods can be
called concurrently in separate actions.  The modules take a size
parameter \te{n}, that indicates the size of the array to return.  The
minimum size is $0$.  An implementation may specify a maximum size (5,
in the current implementation).

{\bf Scheduling}

When a concurrent register is instantiated, it returns an array of \te{Reg}
interfaces.  The scheduling relationships for \te{\_read} and \te{\_write} in
one \te{Reg} interface are the same as for the basic register modules in
\ref{lib-registers}.  The methods of lower-numbered interfaces sequence
before the methods of higher-numbered interfaces.

The designer manages the concurrency of the register accesses by choosing
which interfaces to assign to which rules.

\begin{center}
\begin{tabular}{|p{.75 in}||c|c||c|c||}
\hline
\multicolumn{5}{|c|}{Scheduling Annotations}\\
\multicolumn{5}{|c|}{mkCReg, mkCRegU, mkCRegA}\\
\multicolumn{5}{|c|}{for $j < k$}\\
\hline
& {read$_j$} & {write$_j$} & {read$_k$} & {write$_k$} \\
\hline
\hline
{read$_j$} & CF & SB & SBR & SBR \\
\hline
{write$_j$} & SA & SBR & SBR & SBR \\
\hline
\hline
\end{tabular}
\end{center}

The \te{Reg} interfaces execute in sequence starting with element $0$ of the
array up through element $n-1$.  That is, the \te{\_read} method of the first
interface will return the value stored in the register from the previous
clock cycle; if the \te{\_write} method of the first interface is called,
the \te{\_read} method of the second interface will return the written value,
otherwise it returns the registered value.  And so on, with each \te{\_read}
method returning the last value written to any of the lower-numbered
interfaces, or the register value if none of the lower-numbered \te{\_write}
methods were called.  The value registered at the end of clock cycle is the
value written by the highest-numbered write method that was called.

{\bf Examples}

In the following example, the two rules can be scheduled concurrently in
the same clock cycle, which would not have been possible if the register
\te{byteCount} had been instantiated as a basic \te{mkReg} register.
Further, if both rules execute in a given clock cycle, the value read in
rule \te{doRecv} is the updated value that was written in rule \te{doSend}.

\begin{verbatim}
   Reg#(Bool) byteCount[2] <- mkCReg(2, True);

   rule doSend (canSend);
      ...
      byteCount[0] <= byteCount[0] + len;
   endrule

   rule doRecv (canRecv);
      ...
      byteCount[1] <= byteCount[1] - len;
   endrule
\end{verbatim}

\index{array!named type!example}
\index{Array@\te{Array} (type)!example}
In the above example, \te{mkCReg} returns an array of \te{Reg} interfaces.
This type is expressed implicitly using the bracket syntax.
The type can also be explicitly expressed with the name \te{Array}, when
necessary. (See Section \ref{sec-array} for more information on the \te{Array}
type.)  One place where this can be necessary is when the array type is a
component of a larger type.  A common example of this would be when
instatiating a \te{Vector} of \te{mkCReg} modules:

\begin{verbatim}
   Vector#(N, Array#(Reg#(T))) regs <- replicateM(mkCReg(3,0));
\end{verbatim}

The above instantiation can also be achieved using multidimensional arrays.
Instead of a vector of arrays, one can use an array of arrays:

\begin{verbatim}
   Integer n = valueOf(N);
   Reg#(T) regs[n][3];
   for (Integer i=0; i<n; i=i+1)
      regs[i] <- mkCReg(3,0);
\end{verbatim}


%-----------------------------------------

\subsubsection{RWire}

\index{RWire@\te{RWire}}
\index{wset@\te{wset} (\te{RWire} interface method)}
\index{wget@\te{wget} (\te{RWire} interface method)}

\label{lib-rwire}

An \te{RWire} is a primitive stateless module whose purpose is to allow     
data transfer between methods and rules without    
the cycle latency of a register.  That is, a \te{RWire} may be written
in a cycle and that value can be read out in the same cycle; values
are not stored across clock cycles. 

When scheduling wire modules, since the value is read in the same
cycle in which it is written, writes must occur before reads.  That is, any
rule which writes to a wire  must be scheduled earlier than any other
rule which reads from the 
wire.  This is the reverse of how registers are scheduled.


{\bf Interfaces and Methods}
                                                                        
The \te{RWire} interface is conceptually similar to a register's interface,  
but the output value is wrapped in a \te{Maybe} type.  The \te{wset}
method places a value on the wire and sets the valid signal.  The read-like method, 
\te{wget},  returns the value and a valid signal in a 
\te{Maybe} type. The output is   
only \te{Valid} if a write has a occurred in the same clock     
cycle, otherwise the output is \te{Invalid}.

\begin{center}
\begin{tabular}{|p{.5in}|p{.7in}|p{2.0 in}|p{.5in}|p{1 in}|}
\hline
\multicolumn{5}{|c|}{\te{RWire} Interface}\\
\hline
\multicolumn{3}{|c|}{Method}&\multicolumn{2}{|c|}{Arguments}\\
\hline
Name & Type & Description& Name &\multicolumn{1}{|c|}{Description} \\
\hline
\hline 
\te{wset}&\te{Action}&writes a value and sets the valid signal&\te{datain}&data
to be sent on the wire \\
\hline
\te{wget}&\te{Maybe}&returns the value and the valid signal&&\\
\hline
\end{tabular}
\end{center}




\begin{verbatim}
   interface RWire#(type element_type) ;
      method Action wset(element_type datain) ;
      method Maybe#(element_type) wget() ;
   endinterface: RWire
\end{verbatim}

{\bf Modules}

\index{mkRWireSBR@\te{mkRWireSBR} (\te{RWire} module)}
\index[function]{Prelude!mkRWireSBR}

The \te{mkRWireSBR},  \te{mkRWire}, and \te{mkUnSafeRWire} modules are provided to create an
RWire.  %The \te{mkRWireSBR} is  recommended.
The difference between the  \te{RWire} modules is the scheduling
annotations.  In  \te{mkRWireSBR} the \te{wset} is SBR with itself,
allowing multiple \te{wset}s in the same clock cycle (though not, of
course, in the same rule).

\begin{center}
\begin{tabular}{|p{1 in}|p{4 in}|}
\hline
\te{mkRWireSBR}&Creates an \te{RWire}. Output is only valid if a write
has occurred in the same clock cycle.  This is the recommended module
to use to create an \te{RWire}. \\
\cline{2-2}
&\begin{libverbatim}
module mkRWireSBR(RWire#(element_type))
   provisos (Bits#(element_type, element_width)) ;\end{libverbatim}
\\
\hline
\end{tabular}
\end{center}

\index{mkRWire@\te{mkRWire} (\te{RWire} module)}
\index[function]{Prelude!mkRWire}


\begin{center}
\begin{tabular}{|p{1 in}|p{4 in}|}
\hline
\te{mkRWire}&Creates an \te{RWire}. Output is only valid if a write
has occurred in the same clock cycle.   The write (\te{wset}) must be
sequenced before the read (\te{wget}) and they must be in different rules.   \\
\cline{2-2}
&\begin{libverbatim}
module mkRWire(RWire#(element_type))
   provisos (Bits#(element_type, element_width)) ;\end{libverbatim}
\\
\hline
\end{tabular}
\end{center}

\index{mkUnsafeRWire@\te{mkUnsafeRWire} (\te{RWire} module)}
\index[function]{Prelude!mkUnsafeRWire}


\begin{center}
\begin{tabular}{|p{1 in}|p{4 in}|}
\hline
\te{mkUnsafeRWire}&Creates an \te{RWire}. Output is only valid if a write
has occurred in the same clock cycle.  The write (\te{wset}) must be
sequenced before the read (\te{wget}) but they can  be in the same rule.\\
\cline{2-2}
&\begin{libverbatim}
module mkUnsafeRWire(RWire#(element_type))
   provisos (Bits#(element_type, element_width)) ;\end{libverbatim}
\\
\hline
\end{tabular}
\end{center}

{\bf Scheduling}

When scheduling wire modules, since the value is read in the same
cycle in which it is written, writes must occur before reads.  That is, any
rule which writes to a wire  must be scheduled earlier than any other
rule which reads from the 
wire.  This is the reverse of how registers are scheduled.

\begin{center}
\begin{tabular}{|c|c|c|}
\hline
\multicolumn{3}{|c|}{Scheduling Annotations}\\
\multicolumn{3}{|c|}{mkRWire}\\
\hline
&wget&wset\\
\hline
\hline
wget&CF&SAR\\
\hline
wset&SBR&C\\
\hline
\hline
\end{tabular}
\hm
\begin{tabular}{|c|c|c|}
\hline
\multicolumn{3}{|c|}{Scheduling Annotations}\\
\multicolumn{3}{|c|}{mkRWireSBR}\\
\hline
&wget&wset\\
\hline
\hline
wget&CF&SAR\\
\hline
wset&SBR&SBR\\
\hline
\hline
\end{tabular}
\hm
\begin{tabular}{|c|c|c|}
\hline
\multicolumn{3}{|c|}{Scheduling Annotations}\\
\multicolumn{3}{|c|}{mkUnsafeRWire}\\
\hline
&wget&wset\\
\hline
\hline
wget&CF&SA\\
\hline
wset&SB&C\\
\hline
\hline
\end{tabular}
\end{center}

{\bf Examples}

\begin{verbatim}
   RWire#(int) rw_incr <- mkRWire();  

   rule doit;
      Maybe#(int) mbi = rw_incr.wget();
      int   di        = fromMaybe (?, mbi);
      if      ((! isValid (mbi))
         noAction;
      else // (   isValid (mbi))
         value2 <= value2 + di ;
   endrule

   rule doitdifferently;
      case (rw_incr.wget()) matches
        { tagged Invalid  } : noAction;
        { tagged Valid .di} : value <= value + di;
      endcase
   endrule

   method Action increment (int di);
      rw_incr.wset (di);
   endmethod
\end{verbatim}


\subsubsection{Wire}
\label{lib-wire}

\index{Wire@\te{Wire} (interface)}
The \te{Wire} interface and module are similar to \te{RWire}, but the valid
bit is hidden from the user and the validity of the read is considered
an implicit condition.  The \te{Wire} interface works like
the \te{Reg} interface, so mentioning the name of the wire gets (reads) its
contents whenever they're valid, and using \te{<=} writes the
wire.  \te{Wire} is an \te{RWire} that is designed to be
interchangeable with \te{Reg}.   You can replace a \te{Reg} with a
\te{Wire} without changing the syntax.


{\bf Interfaces and Methods}

\begin{verbatim}
   typedef Reg#(element_type) Wire#(type element_type);
\end{verbatim}


\begin{center}
\begin{tabular}{|p{.5in}|p{.7in}|p{1.5 in}|p{.4in}|p{1.5 in}|}
\hline
\multicolumn{5}{|c|}{\te{Wire} Interface}\\
\hline
\multicolumn{3}{|c|}{Method}&\multicolumn{2}{|c|}{Arguments}\\
\hline
Name & Type & Description& Name &\multicolumn{1}{|c|}{Description} \\
\hline
\hline 
\te{\_write}&\te{Action}&writes a value \te{x1} &\te{x1}&data
to be written \\
\hline
\te{\_read}&\te{a\_type}&returns the value of the wire&&\\
\hline

\end{tabular}
\end{center}


{\bf Modules}

The \te{mkWire} and \te{mkUnsafeWire} modules are provided to create a
\te{Wire}.  The only difference between the two modules are the
scheduling annotations.  The \te{mkWire} version requires that the the
write and the read be in different rules.

\index{mkWire@\te{mkWire} (module)}
\index[function]{Prelude!mkWire}
\index{mkUnsafeWire@\te{mkUnsafeWire} (module)}
\index[function]{Prelude!mkUnsafeWire}


\begin{center}
\begin{tabular}{|p{1 in}|p{4 in}|}
\hline
\te{mkWire}&Creates a \te{Wire}. Validity of the output is
automatically checked as an implicit condition of the read method.
The write and the read methods must be in different rules.\\
\cline{2-2}
& \begin{libverbatim}
module mkWire(Wire#(element_type)) 
   provisos (Bits#(element_type, element_width));
\end{libverbatim}
\\
\hline
\end{tabular}
\end{center}

\begin{center}
\begin{tabular}{|p{1 in}|p{4 in}|}
\hline
\te{mkUnsafeWire}&Creates a \te{Wire}. Validity of the output is
automatically checked as an implicit condition of the read method.
The write and the read methods can be in the same rule.\\
\cline{2-2}
& \begin{libverbatim}
module mkUnsafeWire(Wire#(element_type)) 
   provisos (Bits#(element_type, element_width));
\end{libverbatim}
\\
\hline
\end{tabular}
\end{center}

\begin{center}
\begin{tabular}{|p{.75 in}|c|c|}
\hline
\multicolumn{3}{|c|}{Scheduling Annotations}\\
\multicolumn{3}{|c|}{mkWire}\\
\hline
&{\_read}&{\_write}\\
\hline
\hline
{\_read}&CF&SAR\\
\hline
{\_write}&SBR& C\\
\hline
\hline
\end{tabular}
\hmm\begin{tabular}{|p{.75 in}|c|c|}
\hline
\multicolumn{3}{|c|}{Scheduling Annotations}\\
\multicolumn{3}{|c|}{mkUnsafeWire}\\
\hline
&{\_read}&{\_write}\\
\hline
\hline
{\_read}&CF&SA\\
\hline
{\_write}&SB& C\\
\hline
\hline
\end{tabular}
\end{center}

{\bf Examples}

\begin{verbatim}
module mkCounter_v2 (Counter);
   Reg#(int) value2 <- mkReg(0);  
   Wire#(int) w_incr <- mkWire(); 

   rule r_incr;
      value2 <= value2 + w_incr;
   endrule

   method int read();
      return value2;
   endmethod

   method Action increment (int di);
      w_incr <= di;
   endmethod
endmodule
\end{verbatim}


\subsubsection{BypassWire}
\index{BypassWire@\te{BypassWire} (interface)}
\index{mkBypassWire@\te{mkBypassWire} (module)}
\index[function]{Prelude!mkBypassWire}
\label{lib-bypasswire}

BypassWire is an implementation of the Wire interface where
the \te{\_write} method is an \te{always\_enabled} method. The compiler 
will issue a warning if the method does not appear to be called
every clock cycle. The advantage of this tradeoff is that 
the \te{\_read} method of this interface does not carry any implicit
condition (so it can satisfy a \te{no\_implicit\_conditions} assertion
or an \te{always\_ready} method). %  See the BSV Reference Guide for
% more discussion on the \te{always\_ready}, \te{always\_enabled}, and
% \te{no\_implicit\_conditions} attributes.

\begin{center}
\begin{tabular}{|p{1 in}|p{4 in}|}
\hline
\te{mkBypassWire}&Creates a \te{BypassWire}. The write method is always\_enabled.\\
\cline{2-2}
& \begin{libverbatim}
module mkBypassWire(Wire#(element_type)) 
   provisos (Bits#(element_type, element_width));
\end{libverbatim}
\\
\hline
\end{tabular}
\end{center}

\begin{center}
\begin{tabular}{|p{.75 in}|c|c|}
\hline
\multicolumn{3}{|c|}{Scheduling Annotations}\\
\multicolumn{3}{|c|}{mkBypassWire}\\
\hline
&{\_read}&{\_write}\\
\hline
\hline
{\_read}&CF&SAR\\
\hline
{\_write}&SBR& C\\
\hline
\hline
\end{tabular}
\end{center}


{\bf Examples}

\begin{verbatim}
module mkCounter_v2 (Counter);
   Reg#(int) value2 <- mkReg(0);  
   Wire#(int) w_incr <- mkBypassWire();  

   rule r_incr;
      value2 <= value2 + w_incr;
   endrule

   method int read();
      return value2;
   endmethod

   method Action increment (int di);
      w_incr <= di;
   endmethod
endmodule
\end{verbatim}


\subsubsection{DWire}
\index{DWire@\te{DWire} (interface)}
\index{mkDWire@\te{mkDWire} (module)}
\index[function]{Prelude!mkDWire}
\index{mkUnsafeDWire@\te{mkUnsafeDWire} (module)}
\index[function]{Prelude!mkUnsafeDWire}
\label{lib-dwire}


DWire is an implementation of the Wire interface where the \te{\_read}
method is an \te{always\_ready} method and thus has no implicit
conditions. Unlike the BypassWire however, the \te{\_write method}
need not be always enabled. On cycles when a DWire is written to, the
\te{\_read} method returns that value.  On cycles when no value is
written, the \te{\_read} method instead returns a default value that
is specified as an argument during instantiation.

There are two modules to create a \te{DWire}; the only difference
being the scheduling annotations.  A write is always scheduled before
a read, however the \te{mkDWire} module requires
that the write and  read be in different rules.

\begin{center}
\begin{tabular}{|p{1 in}|p{4.4 in}|}
\hline
\te{mkDWire}&Creates a \te{DWire}. The read method is always\_ready.\\
\cline{2-2}
& \begin{libverbatim}
module mkDWire#(a_type defaultval)(Wire#(element_type)) 
   provisos (Bits#(element_type, element_width));
\end{libverbatim}
\\
\hline
\end{tabular}
\end{center}

\begin{center}
\begin{tabular}{|p{1 in}|p{4.4 in}|}
\hline
\te{mkUnsafeDWire}&Creates a \te{DWire}. The read method is always\_ready.\\
\cline{2-2}
& \begin{libverbatim}
module mkUnsafeDWire#(a_type defaultval)(Wire#(element_type)) 
   provisos (Bits#(element_type, element_width));
\end{libverbatim}
\\
\hline
\end{tabular}
\end{center}

\begin{center}
\begin{tabular}{|p{.75 in}|c|c|}
\hline
\multicolumn{3}{|c|}{Scheduling Annotations}\\
\multicolumn{3}{|c|}{mkDWire}\\
\hline
&{\_read}&{\_write}\\
\hline
\hline
{\_read}&CF&SAR\\
\hline
{\_write}&SBR& C\\
\hline
\hline
\end{tabular}
\hmm\begin{tabular}{|p{.75 in}|c|c|}
\hline
\multicolumn{3}{|c|}{Scheduling Annotations}\\
\multicolumn{3}{|c|}{mkUnsafeDWire}\\
\hline
&{\_read}&{\_write}\\
\hline
\hline
{\_read}&CF&SA\\
\hline
{\_write}&SB& C\\
\hline
\hline
\end{tabular}
\end{center}

{\bf Examples}
\begin{verbatim}
module mkCounter_v2 (Counter);
   Reg#(int) value2 <- mkReg(0); 
   Wire#(int) w_incr <- mkDWire (0); 

   rule r_incr;
      value2 <= value2 + w_incr;
   endrule

   method int read();
      return value2;
   endmethod

   method Action increment (int di);
      w_incr <= di;
   endmethod
endmodule
\end{verbatim}

\subsubsection{PulseWire}

\index{PulseWire@\te{PulseWire} (interface)}
\index{mkPulseWire@\te{mkPulseWire} (module)}
\index{send@\te{send} (\te{PulseWire} interface method)}
\index{\_read@\te{\_read} (\te{PulseWire} interface method)}
\index{mkPulseWireOR@\te{mkPulseWireOR} (module)}
\index[function]{Prelude!mkPulseWire}
\index[function]{Prelude!mkPulseWireOR}

\index{mkUnsafePulseWire@\te{mkUnsafePulseWire} (module)}
\index{mkUnsafePulseWireOR@\te{mkUnsafePulseWireOR} (module)}
\index[function]{Prelude!mkUnsafePulseWire}
\index[function]{Prelude!mkUnsafePulseWireOR}



\label{lib-pulsewire}


{\bf Interfaces and Methods}

The \te{PulseWire} interface is an \te{RWire} without any data.  It is useful
within rules and action methods to signal other methods or rules in
the same clock cycle.  Note that because the read method is called
\te{\_read}, the register shorthand can be used to get its value without
mentioning the method \te{\_read} (it is implicitly added).


\begin{center}
\begin{tabular}{|p{.5in}|p{.7in}|p{2 in}|}
\hline
\multicolumn{3}{|c|}{\te{PulseWire} Interface}\\
\hline
Name & Type & Description  \\
\hline
\hline 
\te{send}&Action&sends a signal down the wire \\
\hline
\te{\_read}&Bool&returns the valid signal\\
\hline

\end{tabular}
\end{center}


\begin{verbatim}
   interface PulseWire;
     method Action send();
     method Bool _read();
   endinterface
\end{verbatim}


{\bf Modules}

Four modules are provided to create a \te{PulseWire}, the only
difference being the scheduling annotations.  
In the \te{OR} versions 
the \te{send} method does not conflict with itself.  Calling the
\te{send} method for a \te{mkPulseWire} from 2 rules causes 
the two rules to conflict while in the \te{mkPulseWireOR} there is no
conflict.  In other words, the \te{mkPulseWireOR} acts a logical
"OR". 
The \te{Unsafe}
versions allow the \te{send} and \te{\_read} methods to be in the same rule.



\begin{center}
\begin{tabular}{|p{1.4 in}|p{4 in}|}
\hline
\te{mkPulseWire}& The writing to this type of wire is used in rules
and action methods to send a single bit to signal other
methods or rules in the same clock cycle.\\
\cline{2-2}
& \begin{libverbatim}
module mkPulseWire(PulseWire);
\end{libverbatim}
\\
\hline
\end{tabular}
\end{center}


\begin{center}
\begin{tabular}{|p{1.4 in}|p{4 in}|}
\hline
\te{mkPulseWireOR}&Returns a \te{PulseWire} which acts like a logical
"Or".  The \te{send} method of the same wire can be used in two
different rules without conflict.\\
\cline{2-2}
& \begin{libverbatim}
module mkPulseWireOR(PulseWire);
\end{libverbatim}
\\
\hline
\end{tabular}
\end{center}

\begin{center}
\begin{tabular}{|p{1.4 in}|p{4 in}|}
\hline
\te{mkUnsafePulseWire}& The writing to this type of wire is used in rules
and action methods to send a single bit to signal other
methods or rules in the same clock cycle.  The \te{send} and
\te{\_read} methods can be in the same rule.\\
\cline{2-2}
& \begin{libverbatim}
module mkUnsafePulseWire(PulseWire);
\end{libverbatim}
\\
\hline
\end{tabular}
\end{center}


\begin{center}
\begin{tabular}{|p{1.4 in}|p{4 in}|}
\hline
\te{mkUnsafePulseWireOR}&Returns a \te{PulseWire} which acts like a logical
"Or".  The \te{send} method of the same wire can be used in two
different rules without conflict.  The \te{send} and \te{\_read}
methods can be in the same rule.\\
\cline{2-2}
& \begin{libverbatim}
module mkUnsafePulseWireOR(PulseWire);
\end{libverbatim}
\\
\hline
\end{tabular}
\end{center}

\begin{center}
\begin{tabular}{|p{.75 in}|c|c|}
\hline
\multicolumn{3}{|c|}{Scheduling Annotations}\\
\multicolumn{3}{|c|}{mkPulseWire}\\
\hline
&{\_read}&{send}\\
\hline
\hline
{\_read}&CF&SAR\\
\hline
{send}&SBR& C\\
\hline
\hline
\end{tabular}
\hmm\begin{tabular}{|p{.75 in}|c|c|}
\hline
\multicolumn{3}{|c|}{Scheduling Annotations}\\
\multicolumn{3}{|c|}{mkUnsafePulseWire}\\
\hline
&{\_read}&{send}\\
\hline
\hline
{\_read}&CF&SA\\
\hline
{send}&SB& C\\
\hline
\hline
\end{tabular}
\end{center}

\begin{center}
\begin{tabular}{|p{.75 in}|c|c|}
\hline
\multicolumn{3}{|c|}{Scheduling Annotations}\\
\multicolumn{3}{|c|}{mkPulseWireOR}\\
\hline
&{\_read}&{send}\\
\hline
\hline
{\_read}&CF&SAR\\
\hline
{send}&SBR&SBR\\
\hline
\hline
\end{tabular}
\hmm\begin{tabular}{|p{.75 in}|c|c|}
\hline
\multicolumn{3}{|c|}{Scheduling Annotations}\\
\multicolumn{3}{|c|}{mkUnsafePulseWireOR}\\
\hline
&{\_read}&{send}\\
\hline
\hline
{\_read}&CF&SA\\
\hline
{send}&SB&SBR\\
\hline
\hline
\end{tabular}
\end{center}



{\bf Counter Example - Using \te{Reg} and \te{PulseWire} }
\begin{verbatim}
interface Counter#(type size_t);
    method Bit#(size_t) read();
    method Action load(Bit#(size_t) newval);
    method Action increment();
    method Action decrement();
endinterface

module mkCounter(Counter#(size_t));
    Reg#(Bit#(size_t)) value <- mkReg(0);          // define a Reg

    PulseWire increment_called <- mkPulseWire();   // define the PulseWires used 
    PulseWire decrement_called <- mkPulseWire();   // to signal other methods or rules

    // whether rules fire is based on values of PulseWires
    rule do_increment(increment_called && !decrement_called);
        value <= value + 1;
    endrule

    rule do_decrement(!increment_called && decrement_called);
        value <= value - 1;
    endrule

    method Bit#(size_t) read();                 // read the register
        return value;
    endmethod

    method Action load(Bit#(size_t) newval);    // load the register
        value <= newval;                        // with a new value
    endmethod

    method Action increment();                  // sends the signal on the
        increment_called.send();                // PulseWire increment_called
    endmethod

    method Action decrement();                  /  sends the signal on the
        decrement_called.send();                // PulseWire decrement_called  
    endmethod
endmodule
\end{verbatim}

\subsubsection{ReadOnly}
\index{ReadOnly@\te{ReadOnly} (interface)}
\label{readonly}

\te{ReadOnly} is an interface which provides a value. The \te{\_read}
        shorthand can be used to read the value.

{\bf Interfaces and Methods}
\begin{center}
\begin{tabular}{|p{.75in}|p{1 in}|p{2 in}|}
\hline
\multicolumn{3}{|c|}{\te{ReadOnly} Interface}\\
\hline
\multicolumn{3}{|c|}{Method}\\
\hline
Name & Type & Description\\
\hline
\hline 
\te{\_read}&\te{a\_type}&Reads the data\\
\hline
\end{tabular}
\end{center}

\begin{libverbatim}
     interface ReadOnly #( type a_type ) ;
        method a_type _read() ;
     endinterface
\end{libverbatim}


{\bf Functions}

\index[function]{Prelude!regToReadOnly}
\index{regToReadOnly@\te{regToReadOnly} (function)}

\begin{center}
\begin{tabular}{|p{1 in}|p{4.6 in}|}
\hline
\te{regToReadOnly}&Converts a \te{Reg} interface into a \te{ReadOnly}
interface.  Useful for giving  as
the  argument to higher-order vector and list functions.\\
\cline{2-2}
&\begin{libverbatim}
function ReadOnly#(a_type) regToReadOnly(Reg#(a_type) regIfc);
\end{libverbatim}
\\
\hline
\end{tabular}
\end{center}

\index[function]{Prelude!pulseWireToReadOnly}
\index{pulseWireToReadOnly@\te{pulseWireToReadOnly} (function)}

\begin{center}
\begin{tabular}{|p{1.4 in}|p{4.2 in}|}
\hline
\te{pulseWireToReadOnly}&Converts a \te{PulseWire} interface into a
\te{ReadOnly} interface. \\
\cline{2-2}
&\begin{libverbatim}
function ReadOnly#(Bool) pulseWireToReadOnly(PulseWire ifc);
\end{libverbatim}
\\
\hline
\end{tabular}
\end{center}


\index[function]{Prelude!readReadOnly}
\index{readReadOnly@\te{readReadOnly} (function)}

\begin{center}
\begin{tabular}{|p{1.4 in}|p{4.2 in}|}
\hline
\te{readReadOnly}&Takes a \te{ReadOnly} interface and returns a
value.  \\
\cline{2-2}
&\begin{libverbatim}
function a_type readReadOnly(ReadOnly#(a_type) r);
\end{libverbatim}
\\
\hline
\end{tabular}
\end{center}

{\bf Examples}

\begin{verbatim}
   interface AHBSlaveIFC;
      interface AHBSlave              bus;
      interface Put#(AHBResponse)     response;
      interface ReadOnly#(AHBRequest) request;
   endinterface
   ...
      interface ReadOnly request;
         method AHBRequest _read;
            let ctrl = AhbCtrl {command:  write_wire,
                                size:     size_wire,
                                burst:    burst_wire,
                                transfer: transfer_wire,
                                prot:     prot_wire,
                                addr:     addr_wire} ;
            let value = AHBRequest {ctrl: ctrl, data: wdata_wire};
            return value;
         endmethod
      endinterface
\end{verbatim}

%=============================================================

\subsubsection{WriteOnly}
\index{WriteOnly@\te{WriteOnly} (interface)}
\label{writeonly}

\te{WriteOnly} is an interface which writes a value. The \te{\_write}
        shorthand is used to write the value.

{\bf Interfaces and Methods}
\begin{center}
\begin{tabular}{|p{.5in}|p{.7in}|p{1.2 in}|p{.4in}|p{1.4 in}|}
\hline
\multicolumn{5}{|c|}{\te{WriteOnly} Interface}\\
\hline
\multicolumn{3}{|c|}{Method}&\multicolumn{2}{|c|}{Arguments}\\
\hline
Name & Type & Description& Name &\multicolumn{1}{|c|}{Description} \\
\hline
\hline 
\te{\_write}&Action&Writes the data&\te{x}&Value to be written, of
datatype \te{a\_type}.\\
\hline
\end{tabular}
\end{center}

\begin{libverbatim}
interface WriteOnly #( type a_type ) ;
   method Action _write (a_type x) ;
endinterface
\end{libverbatim}

{\bf Examples}

\begin{verbatim}
interface WriteOnly#(type a);
   method Action _write(a v);
endinterface

// module with an always-enabled port to tie to a default value
   import "BVI" AlwaysWrite = 
      module mkAlwaysWrite(WriteOnly#(a)) provisos(Bits#(a,sa));
        no_reset;
        parameter width = valueof(sa);
        method _write(D_IN) enable((*inhigh*)EN);
        schedule _write C _write;
   endmodule

   module mkDefaultValue1();
      WriteOnly#(UInt#(7)) d1 <- mkAlwaysWrite(clocked_by primMakeDisabledClock);
      rule handle_d1;
          d1 <= 5;
      endrule
   endmodule
\end{verbatim}

%=============================================================

\subsection{Miscellaneous Functions }

% \begin{center}
% \begin{tabular}{|p{1 in}|p{4 in}|}
% \hline
% \te{}& \\
% \cline{2-2}
% & 
% \\
% \hline
% \end{tabular}
% \end{center}

\subsubsection{Compile-time Messages}

\index{error@\te{error} (forced error)}
\index[function]{Prelude!error}
\begin{center}
\begin{tabular}{|p{1 in}|p{4 in}|}
\hline
\te{error}&Generate a compile-time error message, {\tt s},  and halt compilation. \\
\cline{2-2}
& \begin{libverbatim}
function a_type error(String s);
\end{libverbatim}
\\
\hline
\end{tabular}
\end{center}


\index{warning@\te{warning} (forced warning)}
\index[function]{Prelude!warning}
\begin{center}
\begin{tabular}{|p{1 in}|p{4 in}|}
\hline
\te{warning}&When applied to a value {\tt v} of type {\tt a}, generate a compile-time
warning message, {\tt s}, and continue compilation, returning {\tt v}. \\
\cline{2-2}
& \begin{libverbatim}
function a_type warning(String s, a_type v);
\end{libverbatim}
\\
\hline
\end{tabular}
\end{center}




\index{message@\te{message} (compilation message)}
\index[function]{Prelude!message}
\begin{center}
\begin{tabular}{|p{1 in}|p{4 in}|}
\hline
\te{message}&When applied to a value {\tt v} of type {\tt a}, generate a compile-time
informative message, {\tt s}, and continue compilation, returning {\tt v}. \\
\cline{2-2}
& \begin{libverbatim}
function a_type message(String s, a_type v);
\end{libverbatim}
\\
\hline
\end{tabular}
\end{center}


\index{errorM@\te{errorM} (forced error)}
\index[function]{Prelude!errorM}
\begin{center}
\begin{tabular}{|p{1 in}|p{4 in}|}
\hline
\te{errorM}&Generate a compile-time error message, {\tt s},  and halt
compilation in a monad. \\
\cline{2-2}
& \begin{libverbatim}
function m#(void) errorM(String s)
  provisos (Monad#(m));
\end{libverbatim}
\\
\hline
\end{tabular}
\end{center}


\index{warningM@\te{warningM} (forced warning)}
\index[function]{Prelude!warningM}
\begin{center}
\begin{tabular}{|p{1 in}|p{4 in}|}
\hline
\te{warningM}&Generate a compilation warning in a monad. \\
\cline{2-2}
& \begin{libverbatim}
function m#(void) warningM(String s)
  provisos (Monad#(m));
\end{libverbatim}
\\
\hline
\end{tabular}
\end{center}



\index{messageM@\te{messageM} (compilation message)}
\index[function]{Prelude!messageM}
\begin{center}
\begin{tabular}{|p{1 in}|p{4 in}|}
\hline
\te{messageM}&Generate a compilation message in a monad.\\
\cline{2-2}
& \begin{libverbatim}
function m#(void) messageM(String s)
  provisos (Monad#(m));
\end{libverbatim}
\\
\hline
\end{tabular}
\end{center}



% \index{messageM@\te{messageM} (compilation message)}

% \begin{center}
% \begin{tabular}{|p{1 in}|p{4 in}|}
% \hline
% \te{messageM}&Generate a compilation message in a monad. \\
% \cline{2-2}
% & \begin{libverbatim}
% function m#(void) messageM(String s)
%   provisos (Monad#(m));
% \end{libverbatim}
% \\
% \hline
% \end{tabular}
% \end{center}






\subsubsection{Arithmetic Functions}

\index{max@\te{max} (function)}
\index[function]{Prelude!max}

\begin{center}
\begin{tabular}{|p{1 in}|p{4 in}|}
\hline
\te{max}&Returns the maximum of two values, x and y. \\
\cline{2-2}
& \begin{libverbatim}
function a_type max(a_type x, a_type y)
  provisos (Ord#(a_type));
\end{libverbatim}
\\
\hline
\end{tabular}
\end{center}



\index{min@\te{min} (function)}
\index[function]{Prelude!min}

\begin{center}
\begin{tabular}{|p{1 in}|p{4 in}|}
\hline
\te{min}& Returns the minimum of two values, x and y.\\
\cline{2-2}
& \begin{libverbatim}
function a_type min(a_type x, a_type y)
  provisos (Ord#(a_type));
\end{libverbatim}
\\
\hline
\end{tabular}
\end{center}


\index{abs@\te{abs} (function)}
\index[function]{Prelude!abs}
\begin{center}
\begin{tabular}{|p{1 in}|p{4 in}|}
\hline
\te{abs}&Returns the absolute value of x. \\
\cline{2-2}
&\begin{libverbatim}
function a_type abs(a_type x)
  provisos (Arith#(a_type), Ord#(a_type));
\end{libverbatim}
\\
\hline
\end{tabular}
\end{center}

\index{signedMul@\te{signedMul} (function)}
\index[function]{Prelude!signedMul}

\begin{center}
\begin{tabular}{|p{1 in}|p{4 in}|}
\hline
\te{signedMul}&Performs full precision multiplication on  two Int\#(n)
operands of different sizes. \\
\cline{2-2}
&\begin{libverbatim}
function Int#(m) signedMul(Int#(n) x, Int#(k) y)
  provisos (Add#(n,k,m));
\end{libverbatim}
\\
\hline
\end{tabular}
\end{center}

\index{unsignedMul@\te{unsignedMul} (function)}
\index[function]{Prelude!unsignedMul}

\begin{center}
\begin{tabular}{|p{1 in}|p{4 in}|}
\hline
\te{unsignedMul}&Performs full precision multiplication on  two
unsigned UInt\#(n) operands of different sizes. \\
\cline{2-2}
&\begin{libverbatim}
function UInt#(m) unsignedMul(UInt#(n) x, UInt#(k) y)
  provisos (Add#(n,k,m));
\end{libverbatim}
\\
\hline
\end{tabular}
\end{center}


\index{signedQuot@\te{signedQuot} (function)}
\index[function]{Prelude!signedQuot}

\begin{center}
\begin{tabular}{|p{1 in}|p{4 in}|}
\hline
\te{signedQuot}&Performs full precision division on  two Int\#(n)
operands of different sizes. \\
\cline{2-2}
&\begin{libverbatim}
function Int#(m) signedQuot(Int#(n) x, Int#(k) y) ;
\end{libverbatim}
\\
\hline
\end{tabular}
\end{center}

\index{unsignedQuot@\te{unsignedQuot} (function)}
\index[function]{Prelude!unsignedQuot}

\begin{center}
\begin{tabular}{|p{1 in}|p{4 in}|}
\hline
\te{unsignedQuot}&Performs full precision division on  two unsigned UInt\#(n)
operands of different sizes. \\
\cline{2-2}
&\begin{libverbatim}
function UInt#(m) unsignedQuot(UInt#(n) x, UInt#(k) y) ;
\end{libverbatim}
\\
\hline
\end{tabular}
\end{center}


\subsubsection{Operations on Functions}


 Higher order functions are functions which take functions as
arguments and/or return functions as results.  These are often useful with
list and vector functions. 


\index{compose@\te{compose} (function)}
\index[function]{Prelude!compose}

\begin{center}
\begin{tabular}{|p{1 in}|p{4 in}|}
\hline
\te{compose}&Creates a new function, {\tt c}, made up of
functions, {\tt f} and {\tt g}. That is,  \verb'c(a) = f(g(a))' \\
\cline{2-2}
& \begin{libverbatim}
function (function c_type (a_type x0)) 
     compose(function c_type f(b_type x1), 
             function b_type g(a_type x2));
\end{libverbatim}
\\
\hline
\end{tabular}
\end{center}

\index{composeM@\te{composeM} (function)}
\index[function]{Prelude!composeM}

\begin{center}
\begin{tabular}{|p{1 in}|p{4 in}|}
\hline
\te{composeM}&Creates a new monadic function, \te{m\#(c)}, made up of
functions, {\tt f} and {\tt g}.  That is, \verb'c(a) = f(g(a))' \\
\cline{2-2}
& \begin{libverbatim}
function (function m#(c_type) (a_type x0)) 
      composeM(function m#(c_type) f(b_type x1), 
               function m#(b_type) g(a_type x2))
   provisos # (Monad#(m));
\end{libverbatim}
\\
\hline
\end{tabular}
\end{center}



\index{id@\te{id} (function)}
\index[function]{Prelude!id}

\begin{center}
\begin{tabular}{|p{1 in}|p{4 in}|}
\hline
\te{id}&Identity function, returns {\tt x} when given {\tt x}.  This
function is useful when the argument requires a function which
doesn't do anything.\\
\cline{2-2}
& \begin{libverbatim}
function a_type id(a_type x);
\end{libverbatim}
\\
\hline
\end{tabular}
\end{center}


\index{constFn@\te{constFn} (function)}
\index[function]{Prelude!constFn}

\begin{center}
\begin{tabular}{|p{1 in}|p{4 in}|}
\hline
\te{constFn}&Constant function, returns {\tt x}. \\
\cline{2-2}
&\begin{libverbatim}
function a_type constFn(a_type x, b_type y);
\end{libverbatim}
\\
\hline
\end{tabular}
\end{center}



\index{flip@\te{flip} (function)}
\index[function]{Prelude!flip}
\begin{center}
\begin{tabular}{|p{1 in}|p{4 in}|}
\hline
\te{flip}&Flips the arguments {\tt x} and {\tt y}, returning a new function. \\
\cline{2-2}
& \begin{libverbatim}
function (function c_type new (b_type y, a_type x))
     flip (function c_type old (a_type x, b_type y));
\end{libverbatim}
\\
\hline
\end{tabular}
\end{center}



\index{curry@\te{curry} (function)}
\index[function]{Prelude!curry}

\begin{center}
\begin{tabular}{|p{1 in}|p{4 in}|}
\hline
\te{curry}&This function converts a function on a pair (Tuple2) of
arguments into a function which takes the arguments separately.  The
phrase \verb|t0 f(t1 x, t2 y)| is the function returned by \te{curry}\\
\cline{2-2}
& \begin{libverbatim}
function (function t0 f(t1 x, t2 y)) 
     curry (function t0 g(Tuple2#(t1, t2) x));
\end{libverbatim}
\\
\hline
\end{tabular}
\end{center}

\index{uncurry@\te{uncurry} (function)}
\index[function]{Prelude!curry}

\begin{center}
\begin{tabular}{|p{1 in}|p{4 in}|}
\hline
\te{uncurry}&This function does the reverse of \te{curry}; it converts
a function of two arguments into a function which takes a single
argument, a pair (Tuple2).
\\
\cline{2-2}
& \begin{libverbatim}
function (function t0 g(Tuple2#(t1, t2) x))
     uncurry (function t0 f(t1 x, t2 y));
\end{libverbatim}
\\
\hline
\end{tabular}
\end{center}


{\bf Examples}

\begin{libverbatim}
 //using constFn to set the initial values of the registers in a list
   List#(Reg#(Resource)) items <- mapM( constFn(mkReg(initRes)),upto(1,numAdd) );
 
   return(pack(map(compose(head0,toList),state)));

   xs <- mapM(constFn(mkReg(False)),genList);
\end{libverbatim}


\subsubsection{Bit Functions}

The following functions operate on Bit\#(n) variables.

% \index{msb@\te{msb} (function)}
% \index[function]{Prelude!msb}
% \begin{center}
% \begin{tabular}{|p{1 in}|p{4 in}|}
% \hline
% \te{msb}&Returns the most significant bit of \te{x} \\
% \cline{2-2}
% & \begin{libverbatim}
% function Bit#(1) msb(Bit#(n) x) 
%    provisos(Add#(1,k,n));
% \end{libverbatim}
% \\
% \hline
% \end{tabular}
% \end{center}
% replaced by typeclass Bitwise msb function

\index{parity@\te{parity} (function)}
\index[function]{Prelude!parity}
\begin{center}
\begin{tabular}{|p{1 in}|p{4 in}|}
\hline
\te{parity}&Returns the parity of the bit argument \te{v}.  Example: parity( 5'b1) = 1,
parity( 5'b3) = 0; \\
\cline{2-2}
& \begin{libverbatim}
function Bit#(1) parity(Bit#(n) v); 
\end{libverbatim}
\\
\hline
\end{tabular}
\end{center}

\index{reverseBits@\te{reverseBits} (function)}
\index[function]{Prelude!reverseBits}
\begin{center}
\begin{tabular}{|p{1 in}|p{4 in}|}
\hline
\te{reverseBits}&Reverses the order of the bits in the argument \te{x}. \\
\cline{2-2}
& \begin{libverbatim}
function Bit#(n) reverseBits(Bit#(n) x);
\end{libverbatim}
\\
\hline
\end{tabular}
\end{center}

\index{countOnes@\te{countOnes} (function)}
\index[function]{Prelude!countOnes}
\begin{center}
\begin{tabular}{|p{1 in}|p{4 in}|}
\hline
\te{countOnes}&Returns the count of the number of 1's in the bit
vector \te{bin}. \\
\cline{2-2}
& \begin{libverbatim}
function UInt#(lgn1) countOnes ( Bit#(n) bin )
  provisos (Add#(1, n, n1), Log#(n1, lgn1), 
            Add#(1, xx, lgn1) );
\end{libverbatim}
\\
\hline
\end{tabular}
\end{center}

\index{countZerosMSB@\te{countZerosMSB} (function)}
\index[function]{Prelude!countZerosMSB}
\begin{center}
\begin{tabular}{|p{1 in}|p{4 in}|}
\hline
\te{countZerosMSB}&For the bit vector \te{bin}, count the number of 0s until
the first 1, starting from the most significant bit (MSB). \\
\cline{2-2}
& \begin{libverbatim}
function UInt#(lgn1) countZerosMSB ( Bit#(n) bin )
  provisos (Add#(1, n, n1), Log#(n1, lgn1) );
\end{libverbatim}
\\
\hline
\end{tabular}
\end{center}

\index{countZerosLSB@\te{countZerosLSB} (function)}
\index[function]{Prelude!countZerosLSB}
\begin{center}
\begin{tabular}{|p{1 in}|p{4 in}|}
\hline
\te{countZerosLSB}&For the bit vector \te{bin}, count the number of 0s until
the first 1, starting from the least significant bit (LSB). \\
\cline{2-2}
& \begin{libverbatim}
function UInt#(lgn1) countZerosLSB ( Bit#(n) bin )
  provisos (Add#(1, n, n1), Log#(n1, lgn1) );
\end{libverbatim}
\\
\hline
\end{tabular}
\end{center}

\index{truncateLSB@\te{truncateLSB} (function)}
\index[function]{Prelude!truncateLSB}
\begin{center}
\begin{tabular}{|p{1 in}|p{4 in}|}
\hline
\te{truncateLSB}&Truncates a \te{Bit\#(m)} to a \te{Bit\#(n)}  by
dropping bits starting with the LSB.  \\
\cline{2-2}
& \begin{libverbatim}
function Bit#(n) truncateLSB(Bit#(m) x)
   provisos(Add#(n,k,m));
\end{libverbatim}
\\
\hline
\end{tabular}
\end{center}

{\bf Examples}

\begin{verbatim}
   Bit#(6) f6 = truncateLSB(f);
  
   let cmem=countZerosLSB(cfg.memoryAllocate);

   let n = countOnes(neighbors);
\end{verbatim}

\subsubsection{Integer Functions}

The following functions can only be used
for static elaboration.

\index{gcd@\te{gcd} (function)}
\index[function]{Prelude!gcd}
\begin{center}
\begin{tabular}{|p{1 in}|p{4 in}|}
\hline
\te{gcd}&Calculate the greatest common divisor of two Integers. \\
\cline{2-2}
& \begin{libverbatim}
function Integer gcd(Integer a, Integer b);
\end{libverbatim}
\\
\hline
\end{tabular}
\end{center}

\index{lcm@\te{lcm} (function)}
\index[function]{Prelude!lcm}
\begin{center}
\begin{tabular}{|p{1 in}|p{4 in}|}
\hline
\te{lcm}&Calculate the least common multiple of two Integers. \\
\cline{2-2}
& \begin{libverbatim}
function Integer lcm(Integer a, Integer b);
\end{libverbatim}
\\
\hline
\end{tabular}
\end{center}


\subsubsection{Control Flow Function}

\index{while@\te{while} (function)}
\index[function]{Prelude!while}
\begin{center}
\begin{tabular}{|p{1 in}|p{4 in}|}
\hline
\te{while}&Repeat a function while a predicate holds. \\
\cline{2-2}
& \begin{libverbatim}
function a_type while(function Bool pred(a_type x1), 
            function a_type f(a_type x1), a_type x);
\end{libverbatim}
\\
\hline
\end{tabular}
\end{center}


\index{when@\te{when} (function)}
\index[function]{Prelude!when}
\begin{center}
\begin{tabular}{|p{1 in}|p{4 in}|}
\hline
\te{when}&Adds an implicit condition onto an expression.\\
\cline{2-2}
& \begin{libverbatim}
function a when(Bool condition, a arg); 
\end{libverbatim}
\\
\hline
\end{tabular}
\end{center}

{\bf Example - adding the implicit condition readCount==0 to the
action}

\begin{verbatim}
   function Action startReadSequence (BAddr startAddr,
                                      UInt#(6) count);
      return  when ((readCount == 0), // implicit condition of the action
      (action
          readAddr    <= startAddr ;
          readCount   <= count ;
       endaction));
   endfunction

   rule readSeq;         // readCount==0 is an implicit condition
       startReadSequence (addr, count);
   endrule
\end{verbatim}

% \begin{center}
% \begin{tabular}{|p{1 in}|p{4 in}|}
% \hline
% \te{asTypeOf}&Forces the type of the first argument, x,  to be the
% same as the second argument, y. 
% \index{asTypeOf@\te{asTypeOf} (Prelude function)} \\
% \cline{2-2}
% & \begin{libverbatim}
% function a_type asTypeOf(a_type x, a_type y);
% \end{libverbatim}
% \\
% \hline
% \end{tabular}
% \end{center}



% \index{liftM@\te{liftM} (Prelude function)}

% \begin{center}
% \begin{tabular}{|p{1 in}|p{4 in}|}
% \hline
% \te{liftM}&Any function can be lifted into a monadic version. \\
% \cline{2-2}
% & \begin{libverbatim}
% function m#(b) liftM(function b func(a_type x1), 
%                                      m#(a_type) x)
%   provisos (Monad#(m));
% \end{libverbatim}
% \\
% \hline
% \end{tabular}
% \end{center}



\subsection{Environment Values}

%\index{Environment@\te{Environment} (package)}
The \te{Environment} section of the Prelude contains some value definitions that remain
static within a compilation, but may vary between compilations.


Test whether the compiler is generating C.

\index{genC@\te{genC}}
\index[function]{Prelude!genC}
\begin{center}
\begin{tabular}{|p{1.2 in}|p{4 in}|}
\hline
\te{genC}&Returns \te{True} if the compiler is generating C.\\
\cline{2-2}
&\begin{libverbatim}
function Bool genC();
\end{libverbatim}
\\
\hline
\end{tabular}
\end{center}

\index{genVerilog@\te{genVerilog}}
\index[function]{Prelude!genVerilog}

Test whether the  compiler is generating {\veri}.

\begin{center}
\begin{tabular}{|p{1.2 in}|p{4 in}|}
\hline
\te{genVerilog}&Returns \te{True} if the compiler is generating Verilog.\\
\cline{2-2}
&\begin{libverbatim}
function Bool genVerilog();
\end{libverbatim}
\\
\hline
\end{tabular}
\end{center}

{\bf Examples}
\begin{verbatim}
 if (genVerilog)
	 return (t + fromInteger(adj));
\end{verbatim}

The following two variables provide access to the names of the package
being compiled and the module being synthesized as strings.

\index{genPackageName@\te{genPackageName}}
\index[function]{Prelude!genPackageName}

\begin{center}
\begin{tabular}{|p{1.2 in}|p{4 in}|}
\hline
\te{genPackageName}&Returns a \te{String} containing the name of the
package being compiled. \\
\cline{2-2}
&\begin{libverbatim}
function String genPackageName;
\end{libverbatim}
\\
\hline
\end{tabular}
\end{center}

\index{genModuleName@\te{genModuleName}}
\index[function]{Prelude!genModuleName}

\begin{center}
\begin{tabular}{|p{1.2 in}|p{4 in}|}
\hline
\te{genModuleName}&Returns a \te{String} containing the name of the
module  being synthesized. \\
\cline{2-2}
&\begin{libverbatim}
function String genModuleName;
\end{libverbatim}
\\
\hline
\end{tabular}
\end{center}




\index{compilerVersion@\te{compilerVersion}}
\index[function]{Prelude!compilerVersion}
Return the version of the compiler.

\begin{center}
\begin{tabular}{|p{1.2 in}|p{4 in}|}
\hline
\te{compilerVersion}&Returns a \te{String} containing the compiler
version.  This is the same string used with the \te{-v} flag.\\
\cline{2-2}
&\begin{libverbatim}
String compilerVersion;
\end{libverbatim}
\\
\hline
\end{tabular}
\end{center}
\hmm Example:
\begin{libverbatim}
      The statement:
           $display("compiler version = %s", compilerVersion);
      produces this output:
           compiler version = version 3.8.56 (build 7084, 2005-07-22)
\end{libverbatim}

\index{buildVersion@\te{buildVersion}}
\index[function]{Prelude!buildVersion}
Return the build number of the version of the compiler.

\begin{center}
\begin{tabular}{|p{1.2 in}|p{4 in}|}
\hline
\te{buildVersion}&Returns a \te{Bit\#(32)} containing the build number
portion of the compiler version. \\
\cline{2-2}
&\begin{libverbatim}
Bit#(32) buildVersion;
\end{libverbatim}
\\
\hline
\end{tabular}
\end{center}

\hmm Example:
\begin{libverbatim}
      The statement:
           $display("The build version of the compiler is %d", buildVersion);
      produces this output:
           "The build version of the compiler is 12345"
\end{libverbatim}

\index{date@\te{date}}
\index[function]{Prelude!date}
Get the current date and time.

\begin{center}
\begin{tabular}{|p{1.2 in}|p{4 in}|}
\hline
\te{date}&Returns a \te{String} containing the date. \\
\cline{2-2}
&\begin{libverbatim}
String date;
\end{libverbatim}
\\
\hline
\end{tabular}
\end{center}

\hmm Example:
\begin{libverbatim}
      The statement: 
           $display("date = %s", date); 
      produces this output:
           "date = Mon Feb 6 08:39:59 EST 2006"
\end{libverbatim}


\index{epochTime@\te{epochTime}}
\index[function]{Prelude!epochTime}
Returns the number of seconds from the epoch (1970-01-01 00:00:00) to now.

\begin{center}
\begin{tabular}{|p{1.2 in}|p{4 in}|}
\hline
\te{epochTime}&Returns a \te{Bit\#(32)} containing the number of
seconds since the epoch, which is defined as 1970-01-01 00:00:00. \\
\cline{2-2}
&\begin{libverbatim}
Bit#(32) epochTime;
\end{libverbatim}
\\
\hline
\end{tabular}
\end{center}

\hmm Example:
\begin{libverbatim}
      The statement: 
           $display("Current epoch is %d", epochTime); 
      produces this output:
           "Current epoch is 1235481642"
\end{libverbatim}


\subsection{Compile-time IO}

These functions control file IO  during elaboration.
The functions are expressed as modules and can only be used as
statements inside a \te{module...endmodule} block.

The type \te{Handle} is a primitive type for a file handle.  The value
is returned when you open a file and is used to specify the file by
the other functions.

The flag \te{-fdir}, described in the {\em User Guide},  can be used to specify where relative file paths
should be based from.

\index{openFile@\te{openFile}}
\index[function]{Prelude!openFile}

The type \te{IOMode} is an enumerated type with three values:
\te{ReadMode}, \te{WriteMode}, and \te{AppendMode}:

\begin{verbatim}
    typedef enum { ReadMode, WriteMode, AppendMode } IOMode;
\end{verbatim}


When opening a file you specify the mode (\te{IOMode}) and the
filename.    Opening  a file in write mode  creates a new
file; in append mode it adds to an existing file.

\begin{center}
\begin{tabular}{|p{1 in}|p{4.3 in}|}
\hline
\te{openFile}& Opens a file and returns the  type
\te{Handle}.   \\
\cline{2-2}
&\begin{libverbatim}
module openFile #(String filename, IOMode mode) (Handle);
\end{libverbatim}
\\
\hline
\end{tabular}
\end{center}



\index{hClose@\te{hClose}}
\index[function]{Prelude!hClose}

The function \te{hClose} explicitly closes the file with the specified
handle. The compiler will close any handles that are still open at
the end of elaboration, or upon exiting with an error, but you
shouldn't rely on this.  Buffered files will be flushed when the file
is closed.


\begin{center}
\begin{tabular}{|p{1 in}|p{4.3 in}|}
\hline
\te{hClose}& Closes the file with the specified handle.  \\
\cline{2-2}
&\begin{libverbatim}
module hClose #(Handle hdl) ();
\end{libverbatim}
\\
\hline
\end{tabular}
\end{center}
The following functions provide query functions for handles.


\index{hIsEOF@\te{hIsEOF}}
\index[function]{Prelude!hIsEOF}

\begin{center}
\begin{tabular}{|p{1 in}|p{4.3 in}|}
\hline
\te{hIsEOF}& Returns a Bool indicating if the end of file has been
reached for the specified handle.  \\
\cline{2-2}
&\begin{libverbatim}
function Bool hIsEOF (Handle hdl);
\end{libverbatim}
\\
\hline
\end{tabular}
\end{center}

\index{hIsOpen@\te{hIsOpen}}
\index[function]{Prelude!hIsOpen}

\begin{center}
\begin{tabular}{|p{1 in}|p{4.3 in}|}
\hline
\te{hIsOpen}&  Returns true if the  the handle
\te{hdl} is open.\\
\cline{2-2}
&\begin{libverbatim}
function Bool hIsOpen (Handle hdl);
\end{libverbatim}
\\
\hline
\end{tabular}
\end{center}

\index{hIsClosed@\te{hIsClosed}}
\index[function]{Prelude!hIsClosed}

\begin{center}
\begin{tabular}{|p{1 in}|p{4.3 in}|}
\hline
\te{hIsClosed}& Returns true if the handle
\te{hdl} is closed.\\
\cline{2-2}
&\begin{libverbatim}
function Bool hIsClosed (Handle hdl);
\end{libverbatim}
\\
\hline
\end{tabular}
\end{center}



\index{hIsReadable@\te{hIsReadable}}
\index[function]{Prelude!hIsReadable}

\begin{center}
\begin{tabular}{|p{1 in}|p{4.3 in}|}
\hline
\te{hIsReadable}& Returns true if the  handle has been opened in
Readable mode and can be read from.\\
\cline{2-2}
&\begin{libverbatim}
function Bool hIsReadable (Handle hdl);
\end{libverbatim}
\\
\hline
\end{tabular}
\end{center}

\index{hIsWriteable@\te{hIsWriteable}}
\index[function]{Prelude!hIsWriteable}

\begin{center}
\begin{tabular}{|p{1 in}|p{4.3 in}|}
\hline
\te{hIsWriteable}& Returns true if handle has been opened in Writeable
mode and can be written to.\\
\cline{2-2}
&\begin{libverbatim}
function Bool hIsWriteable (Handle hdl);
\end{libverbatim}
\\
\hline
\end{tabular}
\end{center}

  The default buffering of files is determined by your
 system.  If the system is buffering, you may not see any output until
 the handle is flushed or closed.  You can override this by setting the
 buffering policy of the handle, so that writes are not buffered, or
 are line buffered.  The file handle functions \te{hFlush},
 \te{hGetBuffering}, and \te{hSetBuffering} allow you to 
 control buffering.

At the end of elaboration, or upon exiting with an error, the compiler
closes any file handles that were not otherwised closed.  Any
buffered files will be flushed at this point.

\index{BufferMode@\te{BufferMode} (type)}

The data type \te{BufferMode} indicates the type of buffering.
\begin{verbatim}
   typedef union tagged {
      void NoBuffering;
      void LineBuffering;
      Maybe#(Integer) BlockBuffering;
   } BufferMode;
\end{verbatim}


\index{hFlush@\te{hFlush}}
\index[function]{Prelude!hFlush}

\begin{center}
\begin{tabular}{|p{1 in}|p{4.3 in}|}
\hline
\te{hFlush}& Explicitly flushes the buffer with the specified handle. \\
\cline{2-2}
&\begin{libverbatim}
function Action hFlush(Handle hdl);
\end{libverbatim}
\\
\hline
\end{tabular}
\end{center}

\index{hGetBuffering@\te{hGetBuffering}}
\index[function]{Prelude!hGetBuffering}

\begin{center}
\begin{tabular}{|p{1 in}|p{4.3 in}|}
\hline
\te{hGetBuffering}& Returns the buffering policy of the file with the specified handle. \\
\cline{2-2}
&\begin{libverbatim}
function ActionValue#(BufferMode) hGetBuffering(Handle hdl);
\end{libverbatim}
\\
\hline
\end{tabular}
\end{center}

\index{hSetBuffering@\te{hSetBuffering}}
\index[function]{Prelude!hSetBuffering}

\begin{center}
\begin{tabular}{|p{1 in}|p{4.3 in}|}
\hline
\te{hSetBuffering}& Sets the  buffering mode for the file with the
specified handle if the file system
supports it.\\
\cline{2-2}
&\begin{libverbatim}
function Action hSetBuffering(Handle hdl, BufferMode mode);
\end{libverbatim}
\\
\hline
\end{tabular}
\end{center}



The functions \te{hPutStr} and \te{hPutStrLn} write strings to a
file.  The function \te{hPutStrLn} adds a newline to the end of the string.

\index{hPutStr@\te{hPutStr}}
\index[function]{Prelude!hPutStr}

\begin{center}
\begin{tabular}{|p{1 in}|p{4.2 in}|}
\hline
\te{hPutStr}&Writes the string to the file with the specified handle.\\
\cline{2-2}
&\begin{libverbatim}
module hPutStr #(Handle hdl, String str) ();
\end{libverbatim}
\\
\hline
\end{tabular}
\end{center}

\index{hPutStrLn@\te{hPutStrLn}}
\index[function]{Prelude!hPutStrLn}

\begin{center}
\begin{tabular}{|p{1 in}|p{4.2 in}|}
\hline
\te{hPutStrLn}&Writes the string to the file with the specified handle
and appends a newline to the end of the string.\\
\cline{2-2}
&\begin{libverbatim}
module hPutStr #(Handle hdl, String str) ();
\end{libverbatim}
\\
\hline
\end{tabular}
\end{center}

\index{hPutChar@\te{hPutChar}}
\index[function]{Prelude!hPutChar}

\begin{center}
\begin{tabular}{|p{1 in}|p{4.2 in}|}
\hline
\te{hPutChar}&Writes the character to the file with the specified handle.\\
\cline{2-2}
&\begin{libverbatim}
module hPutChar #(Handle hdl, Char c) ();
\end{libverbatim}
\\
\hline
\end{tabular}
\end{center}


\index{hGetChar@\te{hGetChar}}
\index[function]{Prelude!hGetChar}

\begin{center}
\begin{tabular}{|p{1 in}|p{4.2 in}|}
\hline
\te{hGetChar}&Reads the character from the file with the specified handle.\\
\cline{2-2}
&\begin{libverbatim}
module hGetChar #(Handle hdl) (Char);
\end{libverbatim}
\\
\hline
\end{tabular}
\end{center}

\index{hGetLine@\te{hGetLine}}
\index[function]{Prelude!hGetLine}

\begin{center}
\begin{tabular}{|p{1 in}|p{4.2 in}|}
\hline
\te{hGetLine}&Reads a line from the file with the specified handle.\\
\cline{2-2}
&\begin{libverbatim}
module hGetLine #(Handle hdl) (String);
\end{libverbatim}
\\
\hline
\end{tabular}
\end{center}

\hmm Example: Creates  a file named \te{sysBasicWrite.log} containing
the line \te{"Hello World"}.

\begin{verbatim}
      String fname = "sysBasicWrite.log";

      module sysBasicWrite ();
         Handle hdl <- openFile(fname, WriteMode);
         hPutStr(hdl, "Hello");
         hClose(hdl);
         mkSub;
      endmodule

      module mkSub ();
         Handle hdl <- openFile(fname, AppendMode);
         hPutStrLn(hdl, " World");
         hClose(hdl);
      endmodule
\end{verbatim}


\subsection{Generics}

\index{Generic@\te{Generic}}
\index[function]{Prelude!to}
\index[function]{Prelude!from}

Generics is a mechanism permitting users to derive instances of their own custom type classes.
The design of generics in \BS{} is based on the
\href{https://hackage.haskell.org/package/base-4.19.0.0/docs/GHC-Generics.html}{GHC.Generics}
library in Haskell.
Generics provides a way of converting arbitrary struct and tagged union/data types
to and from a generic representation. In this representation product types are represented as tuples/\te{PrimPair},
and sum types as \te{Either}. The representation types are also tagged with various metadata
about the types, fields and constructors, such as their name, arity and index.
Users can implement a default instance for their type class, using a helper type class over the generic representation.

Due to the complexity of the types involved, writing generic instances in \BSVFull{} can be rather tedious.
Thus the documentation here is instead given using the \BHFull{} syntax.

\subsubsection{The \te{Generic} type class}

The \te{Generic} type class defines a means of converting
values of a datatype \te{a} into a generic representation \te{r}.
The type \te{r} is determined by the type \te{a} as a functional dependency.
The function \te{from} converts a value into its generic representation,
and \te{to} converts a generic representation back into a value.

\begin{verbatim}
   class Generic a r | a -> r where
      from :: a -> r
      to :: r -> a
\end{verbatim}

BSC automatically derives an instance of \te{Generic} for all types that
don't have an explicit instance.  For libraries that export a type
abstractly (without exporting its internals), an explicit instance is
needed, to avoid exposing the internal implementation; see the source of the \te{Vector}
library for an example of this.

\subsubsection{Representation types}

The following types are used in generic representations:

\begin{center}
   \begin{tabular}{|p{2 in}|p{3 in}|}
   \hline
   \begin{libverbatim}
data Either a b
  = Left a | Right b
   \end{libverbatim}
   &
   The standard \te{Either} type is used to represent sum types,
   {\it i.e.} tagged unions/data with multiple constructors. \\
   \hline
   \begin{libverbatim}
interface PrimPair a b =
  fst :: a
  snd :: b
   \end{libverbatim}
   &
   The standard \te{PrimPair} type is used to represent product types,
   {\it i.e.} structs/interfaces/data constructors with multiple fields.
   Since this is also the same underlying representation as tuples,
   tuple syntax (as described in the \BH ref guide) may be used for products in generic instances.\\
   \hline
   \begin{libverbatim}
interface PrimUnit = { }
   \end{libverbatim}
   &
   The standard \te{PrimUnit} type is used to represent types containing no data,
   {\it i.e.} empty structs/interfaces/data constructors.\\
   \hline
   \begin{libverbatim}
data (Vector :: # -> * -> *)
  n a
   \end{libverbatim}
   &
   The standard \te{Vector} type is used to represent fixed-size collection types -- \te{ListN} and \te{Vector}.\\
   \hline
   \begin{libverbatim}
data Conc a = Conc a
   \end{libverbatim}
   &
   The \te{Conc} type is used to wrap the (non-generic) types of fields in the sum of products. \\
   \hline
   \begin{libverbatim}
data ConcPrim a = ConcPrim a
   \end{libverbatim}
   &
   The \te{ConcPrim} type is used in \te{Generic} instances for primitive types, {\it e.g.} \te{Bit}.
   This type is used instead of \te{Conc} to avoid infinite recursion through a \te{Conc} instance
   of the generic version of a type class that defaults back to the non-generic one.
   Users of generics typically do not need to define instances for \te{ConcPrim}; this exists to help supply
   generic instances for several type classes that are used internally by BSC.\\
   \hline
   \begin{libverbatim}
data ConcPoly a = ConcPoly a
   \end{libverbatim}
   &
   The \te{ConcPoly} type exists as a workaround for a limitation of generics in dealing with higher-rank data types;
   see below for more details.  Users typically should not need to define instances for \te{ConcPoly}. \\
   \hline
   \begin{libverbatim}
data Meta m r = Meta r
   \end{libverbatim}
   &
   \te{Meta} is used to tag a representation type with additional type-level metadata.
   The \te{m} type parameter must be one of the below metadata types \\
   \hline
   \end{tabular}
\end{center}

{\bf Examples}

Ignoring metadata, the derived \te{Generic} instance for the \te{PrimPair} type is

\begin{libverbatim}
   instance Generic (PrimPair a b) (Conc a, Conc b) where
      from x = (Conc x.fst, Conc x.snd)
      to (Conc x, Conc y) = PrimPair { fst=x; snd=y; }
\end{libverbatim}

The derived \te{Generic} instance for the \te{List} type is

\begin{libverbatim}
   instance Generic (List a) (Either () (Conc a, Conc (List a))) where
      from Nil = Left ()
      from (Cons x y) = Right (Conc x, Conc y)
      to (Left ()) = Nil
      to (Right (Conc x, Conc y)) = Cons x y
\end{libverbatim}

In the generic representation for data/tagged unions with more than two constructors,
the \te{Either} types are aranged to form a left-biased, balanced binary tree.
This makes it possible to directly convert between nested left/right constructors,
and a binary tag value corresponding to the constructor index.
An example of this, in implementing a \te{CustomBits} type class, is given below.

{\bf Higher-rank data}

Generics is not able to fully handle {\it higher-rank} data, {\it i.e.} structs/interfaces/data constructors
containing type variables that are not bound as type parameters.  For example, the following type is higher-rank:

\begin{libverbatim}
   struct Foo =
      x :: a -> a -- Higher rank
      y :: Int 8
\end{libverbatim}

If a \te{Generic} instance were derived for this type in the usual fasion,
then the representation would contain \te{Conc (a -> a)}.
The presence of a type variable in the representation means that it is not uniquely determined by the type \te{Foo},
as required by the functional dependency.

Instead, when deriving an instance for a higher-rank struct or data, a ``wrapper'' struct is generated
for each higher-rank field.  This is wrapped in the \te{ConcPoly} constructor, to indicate that this
is not the real type of the field:

\begin{libverbatim}
   struct Foo_x =
     val :: a -> a

   instance Generic Foo (ConcPoly Foo_x, Conc (Int 8))) where
     from a = (ConcPoly (Foo_x { val = a.x; }), Conc a.y)
     to (ConcPoly x, Conc y) = Foo { x = x.val; y = y; }
\end{libverbatim}

Users can omit an instance for \te{ConcPoly} to not support higher-rank data,
or define some useful default behavior.
For example, the \te{CShow} library defines an instance for ConcPoly
to return a placeholder string for higher-rank fields.

\subsubsection{Metadata types}

The following types are used to represent metadata in generic representations.
Note that these only appear at the type level tagging a \te{Meta} type;
values of these types are not constructed.

\begin{center}
   \begin{tabular}{|p{2.9 in}|p{2.7 in}|}
   \hline
   \begin{libverbatim}
data (MetaData :: $ -> $ -> * -> # -> *)
    name pkg tyargs ncons
   \end{libverbatim}
   &
   Indicates that a representation is for a type (e.g. struct/data) with a name,
   package, tuple of type arguments and number of constructors.
   Types of kind \te{*}, \te{\#} or \te{\$} appearing in the type arguments are wrapped
   in one of the following type constructors:
   \vspace{0.1in}
   \begin{libverbatim}
   data (StarArg :: * -> *) i
   data (NumArg :: # -> *) i
   data (StrArg :: $ -> *) i
   data ConArg
   \end{libverbatim}
   \vspace{-0.15in}
   Constructor-kinded types arguments cannot be handled in general and
   are omitted from the \te{ConArg} representation type.
   \\
   \hline
   \begin{libverbatim}
data (MetaConsNamed :: $ -> # -> # -> *)
    name idx nfields
   \end{libverbatim}
   &
   Indicates that a representation is for a constructor with named fields,
   with a name, index in the data's constructors, and number of fields. \\
   \hline
   \begin{libverbatim}
data (MetaConsAnon :: $ -> # -> # -> *)
    name idx nfields
   \end{libverbatim}
   &
   Indicates that a representation is for a data constructor with anonymous fields,
   with a name, index in the data's constructors, and number of fields. \\
   \hline
   \begin{libverbatim}
data (MetaField :: $ -> # -> *) name idx
   \end{libverbatim}
   &
   Indicates that a representation is for a field, with a field name (either the
   given name for a named field or the generated field name for an anonymous
   field) and index in the constructor's fields. \\
   \hline
   \end{tabular}
\end{center}

{\bf Examples}

Including metadata, the derived \te{Generic} instance for the \te{PrimPair} type is

\begin{libverbatim}
   instance Generic (PrimPair a b)
      (Meta (MetaData "PrimPair" "Prelude" (StarArg a, StarArg b) 1)
      (Meta (MetaConsNamed "PrimPair" 0 2)
      (Meta (MetaField "fst" 0) (Conc a),
         Meta (MetaField "snd" 1) (Conc b)))) where
   from x = Meta (Meta (Meta (Conc x.fst), Meta (Conc x.snd)))
   to (Meta (Meta (PrimPair (Meta (Conc a1)) (Meta (Conc a2))))) = 
      PrimPair { fst = a1; snd = a2; }
\end{libverbatim}

The derived \te{Generic} instance for the \te{List} type is

\begin{libverbatim}
   instance Generic (List a)
      (Meta (MetaData "List" "Prelude" (StarArg a) 2)
      (Either (Meta (MetaConsAnon "Nil" 0 0) ())
      (Meta (MetaConsAnon "Cons" 1 2)
         (Meta (MetaField "_1" 0) (Conc a),
         Meta (MetaField "_2" 1) (Conc (List a)))))) where
   from Nil = Meta (Left (Meta ()))
   from (Cons x y) =
      Meta (Right (Meta (Meta (Conc x), Meta (Conc y))))
   to (Meta (Left (Meta ()))) = Nil
   to (Meta (Right (Meta ((Meta (Conc x)), (Meta (Conc y)))))) = Cons x y
\end{libverbatim}

\subsubsection{Defining generic instances}

The typical way for users to define a generic implementation for their type class is to define a helper type class
that works over the generic representation, and then define a default instance for the original type class
using \te{Generic} to convert to and from the generic representation.  
For example, one can use generics to implement a custom version of the \te{Bits} type class:

\begin{libverbatim}
   class MyBits a n | a -> n where
     mypack   :: a -> Bit n
     myunpack :: Bit n -> a

   -- Explicit instances for primitive types
   instance MyBits (Bit n) n where
     mypack = id
     myunpack = id

   -- Generic default instance
   instance (Generic a r, MyBits' r n) => MyBits a n where
     mypack   x  = mypack' $ from x
     myunpack bs = to $ myunpack' bs

   class MyBits' r n | r -> n where
     mypack'   :: r -> Bit n
     myunpack' :: Bit n -> r

   -- Instance for sum types
   instance (MyBits' r1 n1, MyBits' r2 n2, Max n1 n2 c, Add 1 c n,
             Add p1 n1 c, Add p2 n2 c) =>
      MyBits' (Either r1 r2) n where
     mypack' (Left x) = 1'b0 ++ extend (mypack' x)
     mypack' (Right x) = 1'b1 ++ extend (mypack' x)
     myunpack' bs =
        let (tag, content) = (split bs) :: (Bit 1, Bit c)
        in case tag of
           0 -> Left $ myunpack' $ truncate content
           1 -> Right $ myunpack' $ truncate content

   -- Instance for product types
   instance (MyBits' r1 n1, MyBits' r2 n2, Add n1 n2 n) =>
      MyBits' (r1, r2) n where
     mypack' (x, y) = mypack' x ++ mypack' y
     myunpack' bs = let (bs1, bs2) = split bs
                  in (myunpack' bs1, myunpack' bs2)

   instance  MyBits' () 0 where
     mypack' () = 0'b0
     myunpack' _ = ()

   instance (MyBits' a m, Bits (Vector n (Bit m)) l) =>
      MyBits' (Vector n a) l where
     mypack' v = pack $ map mypack' v
     myunpack' = map myunpack' `compose` unpack

   -- Ignore all types of metadata
   instance (MyBits' r n) => MyBits' (Meta m r) n where
     mypack' (Meta x) = mypack' x
     myunpack' bs = Meta $ myunpack' bs

   -- Conc instance calls back to the non-generic MyBits class
   instance (MyBits a n) => MyBits' (Conc a) n where
     mypack' (Conc x) = mypack x
     myunpack' bs = Conc $ myunpack bs
\end{libverbatim}

A more sophisticated use of generics, making use of metadata,
can be found in the implementation of the \te{CShow} library.
