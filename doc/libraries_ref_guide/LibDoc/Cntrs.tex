\subsubsection{Cntrs}
\label{sec-Cntrs}


{\bf Package}
\index{Cntrs (package)}

\begin{verbatim}
import Cntrs :: * ;
\end{verbatim}

{\bf Description}

The Cntrs package provides  interfaces and modules to implement
 typed and  untyped up/down counters.  

The \te{Count} interface and associated \te{mkCount} module provides
an 
up/down counter which allows atomic simultaneous increment and
decrement operations.  The scheduled order of operations in a single cycle is:
\begin{verbatim}
 read SB update SB (incr,decr) SB write
\end{verbatim}

If there are simultaneous \te{update}, \te{incr}, and \te{decr}
operations, the final result will be:
\begin{verbatim}
update_val + incr_val - decr_val
\end{verbatim}
A \te{write} sets the new value to \te{write\_val} regardless of the
other methods called in the  cycle.

The \te{UCount} interface and associated \te{mkUCount} module provide
an untyped version of an up/down counter; that is the counter width
can be determined at elaboration time rather than type check time.
The value of the counter is represented by an \te{UInt\#(n)} where the
width of the counter is determined by the \te{maxValue} ($0 \leq
maxValue <  2^{32}$) argument.  There are no methods to access the
counter value directly; you can only access the value through the
comparison operations.

{\bf Interfaces and Methods}

The Cntrs package provides two interfaces; \te{Count} which is a typed
interface and \te{UCount} which is an untyped interface.  

\begin{center}
\begin{tabular}{|p {1 in}|p{1 in}|p{3 in}|}
\hline
\multicolumn{3}{|c|}{Count Interface Methods}\\
\hline
Name & Type & Description\\
\hline
\hline 
\te{incr}  & Action &Increments the counter by \te{incr\_val}\\
\hline
\te{decr} & Action& Decrements the counter by \te{decr\_val} \\
\hline
\te{update} & Action& Sets the value to \te{update\_val}.  Final value
will include increment and decrement values. \\
\hline
\te{\_write} &Action & Sets the final value to \te{write\_val}
regardless of other operations in the cycle. \\
\hline
\te{\_read} & \te{t} &Returns the value of the counter. \\
\hline
\end{tabular}
\end{center}


\begin{libverbatim}
interface Count#(type t);
   method Action incr    (t incr_val);
   method Action decr    (t decr_val);
   method Action update  (t update_val);
   method Action _write  (t write_val);
   method t      _read;
endinterface
\end{libverbatim}

\begin{center}
\begin{tabular}{|p {1 in}|p{1 in}|p{3 in}|}
\hline
\multicolumn{3}{|c|}{UCount Interface Methods}\\
\hline
Name & Type & Description\\
\hline
\hline 
\te{incr}  & Action &Increments the counter by \te{incr\_val}\\
\hline
\te{decr} & Action& Decrements the counter by \te{decr\_val} \\
\hline
\te{update} & Action& Sets the value to \te{update\_val}.  Final value
will include increment and decrement values. \\
\hline
\te{\_write} &Action & Sets the final value to \te{write\_val} 
regardless of other operations in the cycle. \\
\hline
\te{isEqual} & \te{Bool} &Returns true if \te{val} is equal to  the
 value of the counter. \\
\hline
\te{isLessThan} & \te{Bool} &Returns true if \te{val} is less than the
 value of the counter. \\
\hline
\te{isGreaterThan} & \te{Bool} &Returns true if \te{val} is greater than the
 value of the counter. \\
\hline
\end{tabular}
\end{center}

\begin{libverbatim}
interface UCount;
   method Action update (Integer update_val);
   method Action _write (Integer write_val);
   method Action incr   (Integer incr_val);
   method Action decr   (Integer decr_val);
   method Bool isEqual  (Integer val);
   method Bool isLessThan    (Integer val);
   method Bool isGreaterThan (Integer val);
endinterface

\end{libverbatim}




{\bf Modules}

\index{mkCounter@\te{mkCounter} (module)}
\index[function]{Counter!mkCounter}



\begin{tabular}{|p{1.4 in}|p{4.6 in}|}
\hline
& \\
\te{mkCounter} &Instantiates a Counter where read precedes update
precedes an increment or decrement precedes a write. The \te{ModArith}
provisos limits its use to module 2 arithmetic types: \te{UInt}, \te{Int},
and \te{Bit}.   Widths of size 0 are supported.\\
\cline{2-2}
& \begin{libverbatim}
module mkCount #(t resetVal) (Count#(t))
   provisos ( Arith#(t)
             ,ModArith#(t)
             ,Bits#(t,st));
\end{libverbatim}
\\
\hline
\end{tabular}

\begin{tabular}{|p{1.4 in}|p{4.6 in}|}
\hline
& \\
\te{mkUCount} &Instantiates a counter which can count from 0 to
\te{maxVal} inclusive.  \te{maxVal} must be known at compile time.
The \te{initValue} and \te{maxValue} must be Integers.\\
\cline{2-2}
& \begin{libverbatim}
module mkUCount#(Integer initValue, Integer maxValue) (UCount);
\end{libverbatim}
\\
\hline
\end{tabular}

{\bf Verilog Modules}

\te{mkCounter} and \te{mkUCount}  corresponds to the following {\V}
module, which are found in the BSC {\V} library,
\te{\$BLUESPECDIR/Verilog/}.

\begin{center}
\begin{tabular} {|p{1.8in}|p{1.5in}|p{1in}|}
\hline
&&\\
BSV Module Name & Verilog Module Name&Defined in File \\
&&\\
\hline
\hline
\te{mkCounter} & vCount & Counter.v \\
\te{mkUCount} && \\
\hline
\end{tabular}
\end{center}
