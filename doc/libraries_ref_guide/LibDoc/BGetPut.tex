\subsubsection{BGetPut}
\index{BGetPut@\te{BGetPut} (package)|textbf}

{\bf Package Name}

import BGetPut :: * ;

{\bf Description}

The \te{BGetPut} package should be used when
the two ends have different clocks.  In all other circumstances, the
\te{CGetPut} package will probably be preferable since the
BGetPut protocol is very slow. 

The interfaces \te{BGet} and \te{BPut} are similar to
\te{Get} and \te{Put}, but the interconnection of them
(via \te{Connectable} or in {\veri}) is implemented with a simple protocol
that allows all inputs and outputs to be directly connected.   
All wires go to registers and have no {\blue} handshaking.
The protocol makes no assumptions about setup time and hold time for the
registers at each end;  so these interfaces may be used when the two ends
have different clocks. 

The protocol consists of the sender putting the value to be sent on the
\te{pvalue} output, and then toggling the \te{ppresent} wire.
The receiver acknowledges the receipt by toggling the \te{gcredit} wire.
Both \te{ppresent} and \te{gcredit} start out low.

{\bf Interfaces and methods}

\begin{tabular}{|p{1.2in}|p{.8in}|p{2in}|p{1.3in}|}
 \hline
\multicolumn{4}{|c|}{Interfaces}\\
\hline
Interface Name   & Parameter name & Parameter Description & Restrictions \\
\hline
\hline
\te{BGet} & \it{element\_type} & type of the element & must be in
                          \te{Bits} class\\
& &being retrieved by the \te{BGet} &\\
\hline
\te{BPut} & \it{element\_type} & type of the element & must be in
\te{Bits} class\\
& &being added by the \te{BPut} &\\
\hline
\te{BGetPut}&\it{element\_type}&type of the element&must be in the
\te{Bits} class\\
\hline
\te{GetBPut}&\it{element\_type}&type of the element&must be in the
\te{Bits} class\\
\hline
\end{tabular}



\begin{itemize}

\item{\te{BGet}}

\begin{center}
\begin{tabular}{|p{.7in}|p{.8 in}|p{1.5 in}|p{.4in}|p{1.4 in}|}
\hline
\multicolumn{5}{|c|}{BGet}\\
\hline
\multicolumn{3}{|c|}{Method}&\multicolumn{2}{|c|}{Argument}\\
\hline
Name & Type & Description& Name &\multicolumn{1}{|c|}{Description} \\
\hline
\hline 
\te{gvalue}&\it{element\_type}&returns an item from the interface&&\\
\hline
\te{gpresent}&\te{Bool}&toggles when new data is available && \\
\hline
\te{gcredit}&Action&toggles when ready for new data&x1&\te{Bool}\\
\hline
\end{tabular}
\end{center}


\begin{libverbatim}
interface BGet #(type element_type);
    method element_type gvalue();
    method Bool gpresent();
    method Action gcredit(Bool x1);
endinterface: BGet
\end{libverbatim}

\item{\te{BPut}}

\begin{center}
\begin{tabular}{|p{.7in}|p{.8 in}|p{1.5 in}|p{.4in}|p{1.4 in}|}
\hline
\multicolumn{5}{|c|}{BPut}\\
\hline
\multicolumn{3}{|c|}{Method}&\multicolumn{2}{|c|}{Argument}\\
\hline
Name & Type & Description& Name &\multicolumn{1}{|c|}{Description} \\
\hline
\hline 
\te{pvalue}&Action&new data&&\\
\hline
\te{ppresent}&\te{Bool}&toggles when new data is available && \\
\hline
\te{pcredit}&Action&toggles when ready for new data&x1&\te{Bool}\\
\hline
\end{tabular}
\end{center}

\begin{libverbatim}
interface BPut #(type element_type);
    method Action pvalue();
    method Bool ppresent();
    method Action pcredit(Bool x1);
endinterface: BPut
\end{libverbatim}

\item{\te{BGetPut}}
\begin{libverbatim}
typedef Tuple2 #(BGet#(element_type), Put#(element_type)) BGetPut #(type element_type);
\end{libverbatim}

\item{\te{GetBPut}}
\begin{libverbatim}
typedef Tuple2 #(Get#(element_type), BPut#(element_type)) GetBPut #(type element_type);
\end{libverbatim}

\item{Connectables}

The \te{BGet} and \te{BPut} interface are connectable.
\begin{libverbatim}
instance Connectable #(BGet#(sa), BPut#(sa));
\end{libverbatim}
\lineup
\begin{libverbatim}
instance Connectable #(BPut#(sa), BGet#(sa));
\end{libverbatim}

\item{\te{BClient} and \te{BServer}}

\begin{libverbatim}
interface BClient #(type req_type, type resp_type);
interface BServer #(type req_type, type resp_type);

typedef Tuple2 #(BClient#(req_type, resp_type),Server#(req_type, resp_type)) 
                 BClientServer #(type req_type, type resp_type);
typedef Tuple2 #(Client#(req_type, resp_type), BServer#(req_type, resp_type)) 
                 ClientBServer #(type req_type, type resp_type);
\end{libverbatim}
A \te{BClient} can be connected to a \te{BServer}
and vice versa.
\begin{libverbatim}
instance Connectable #(BClientS#(req_type, resp_type), BServerS#(req_type, resp_type));
instance Connectable #(BServerS#(req_type, resp_type), BClientS#(req_type, resp_type));
\end{libverbatim}


\end{itemize}

{\bf Modules}
\index{mkBGetPut@\te{mkBGetPut} (function)|textbf}
\index{mkGetBPut@\te{mkGetBPut} (function)|textbf}

\begin{center}
\begin{tabular}{|p{1 in}|p{4.5 in}|}
 \hline
&\\
\te{mkBGetPut} & Create one end of the buffer.  Access to it is via a \te{Put} interface.\\
\cline{2-2}
& \begin{libverbatim}
module mkBGetPut(Tuple2 #(BGet#(sa), Put#(a)))
  provisos (Bits#(a, sa));
\end{libverbatim}
\\ \hline
\end{tabular}
\end{center}

\begin{center}
\begin{tabular}{|p{1 in}|p{4.5 in}|}
 \hline
& \\
\te{mkGetBPut}&Create the other end of the buffer.  Access to it is via a \te{Get}
interface. \\
\cline{2-2}
&\begin{libverbatim}
module mkGetBPut(Tuple2 #(Get#(a), BPut#(sa)))
  provisos (Bits#(a, sa));
\end{libverbatim} 
\\
\hline
\end{tabular}
\end{center}



\index{mkClientBServer@\te{mkClientBServer} (function)|textbf}


\begin{center}
\begin{tabular}{|p{1 in}|p{4.5 in}|}
 \hline
&\\
\te{mkClientBServer}&\\
\cline{2-2}
&\begin{libverbatim}
module mkClientBServer(Tuple2 #(Client#(req_type, resp_type), 
                                BServer#(req_type, resp_type)))
  provisos (Bits#(req_type), Bits#(resp_type));
\end{libverbatim}
\\
\hline
\end{tabular}
\end{center}

\index{mkBClientServer@\te{mkBClientServer} (function)|textbf}
\begin{center}
\begin{tabular}{|p{1 in}|p{4.5 in}|}
 \hline
&\\
\te{mkBClientServer}&\\
\cline{2-2}
&\begin{libverbatim}
module mkBClientServer(Tuple2 #(BClientS#(req_type, resp_type), 
                                Server#(req_type, resp_type)))
  provisos (Bits#(req_type), Bits#(resp_type));
\end{libverbatim}
\\
\hline
\end{tabular}
\end{center}



